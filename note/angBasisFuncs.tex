\subsection{Basis Functions} \label{sec:angBasisFuncs}
The angular dependence of a decay is governed by the directions of the decay product spins. One way
of deriving this dependence is to apply the \emph{helicity formalism}
\cite{JacobWick,Chung,Richman,Kutschke}. In this
description the projection of the spin in the rest frame of the particle is given by its helicity
in the rest frame of its mother particle. After a boost from the mother frame into the daughter
frame, the spin projection in the direction of the boost is equal to the helicity. Assuming a
two-body decay, the difference of the daughter particle helicities is equal to the spin projection
of the mother particle along the decay axis.

The rotation of a spin state into the direction of the decay axis is specified by the three Euler
angles $\phi$, $\theta$ and $\phi'$. The first two Euler angles are equal to, respectively, the
azimuthal and polar angles that specify this direction. These angles are defined in a spherical
coordinate system in the mother rest frame. The $z$-axis of this frame points in the direction of
the original spin projection of the mother particle. In the \emph{Jacob--Wick convention}, the
third angle $\phi'$ is defined to be equal to $-\phi$. This convention will be applied throughout
this work. A shorthand notation will be used to specify the set of rotation angles:
$\EulSet=(\phi,\,\theta,\,-\phi)$.

The amplitude for a mother particle to have a particular spin projection along its decay axis is
given by a \emph{Wigner D-matrix}: ${D_{m,\,n}^{j}}^*(\EulSet)$. For a spin $j$ particle with spin
projection $m$ in the $z$-direction, this is the matrix element for a spin projection $n$ along the
decay axis. A representation of the D-matrices can be given in terms of the matrices
$d_{m,\,n}^{j}(\theta)$, which have real values:
\begin{equation}
  D_{m,\,n}^{j}(\phi,\theta,\phi') = e^{-im\phi}\, d_{m,\,n}^{j}(\theta)\, e^{-in\phi'}
\end{equation}

\begin{figure}[htb]
  \centering
  \resizebox{0.9\textwidth}{!}{\input{figs/helFormFrames.pdftex_t}}
  \caption{Helicity frames for the decay $B\to a(\to P_1P_2)\ b(\to P_3P_4)$ in a general
    configuration where the coordinate systems in the $a$ and $b$ rest frames are aligned (a) and
    with the Jacob--Wick convention \cite{JacobWick}, in which the coordinate system in the $b$
    rest frame is rotated (b). A definition of the helicity angle \phihel{} in the Jacob--Wick
    convention is shown in (c).}
  \label{fig:helFormFrames}
\end{figure}

The decay of a particle $B$ into four particles $P_1$ through $P_4$ via intermediate states $a$ and
$b$ may be considered as a sequence of three two-body decays. Coordinate systems in the rest
frames of mother particles $B$, $a$ and $b$ are depicted in Figure~\ref{fig:helFormFrames}. In
part (a) of the figure, the $a$ and $b$ coordinate systems are aligned. Both $z$-axes are pointing
in the same direction along the $B$ decay axis. Usually, an alternative system is defined for the
decay of the $b$. Part (b) of the figure shows the Jacob--Wick convention, in which the
$z_a$ and $z_b$-axes point in the opposite directions and the $y_a$- and $y_b$-axes are aligned.

Rotating the $b$ coordinate system introduces an additional D-matrix. This matrix gives the
amplitude for state $b$ having a spin projection $\lambda_b$ in the new $z_b$ direction, while it
had a spin projection $-\lambda_b$ in the original $z_b$ direction.\footnote{Note that the boost
from the $B$ system into the $b$ system is in the negative $z_b$ direction in
Figure~\ref{fig:helFormFrames}a. Therefore, a helicity $\lambda_b$ gives a spin projection
$-\lambda_b$ along the $z_b$ axis. In the coordinate system of part (b) of the figure, this is a
spin projection $\lambda_b$ in the $z_b$ direction.} The system is rotated by 180$^\circ$ about
the $y_b$ axis ($\theta=\pi$). The two other Euler angles are arbitrary, as long as they are equal:
$\phi=\phi'$. This gives the following factor in the amplitude:\footnote{In the
\emph{Jackson convention}, the $x$-axes of the $a$ and $b$ systems are aligned instead of the
$y$-axes. This involves a rotation by angles $(\gamma,\,\pi,\,\gamma-\pi)$, which gives a factor
${D_{-\lambda_b,\,\lambda_b}^{j_b}}^*(\gamma,\,\pi,\,\gamma-\pi)=(-1)^{j_b}$.}
\begin{equation}
  {D_{-\lambda_b,\,\lambda_b}^{j_b}}^*(\gamma,\,\pi,\,\gamma)
    = e^{-i\lambda_b\gamma}\ d_{-\lambda_b,\,\lambda_b}^{j_b}(\pi)\ e^{+i\lambda_b\gamma}
    = (-1)^{j_b-\lambda_b}
\end{equation}

The amplitude for the decay $B\to a(\to P_1P_2)\ b(\to P_3P_4)$ is proportional to the
product of the three decay amplitudes for $B\to ab$ ($\mathcal{A}$), $a\to P_1P_2$ ($\mathcal{B}$)
and $b\to P_3P_4$ ($\mathcal{C}$). These amplitudes depend on the spins of the particles and
intermediate states. Since the states $a$ and $b$ are not being observed, their spins must be
summed over. For each of the three sub-decays, a D-matrix gives the amplitude for the spin
projection. This combination of amplitudes is given by:
\begin{align}\label{eq:amplitude}
  A &\propto \sum_{j_a,\,j_b}\,
       \sqrt{\frac{2j_B+1}{4\pi}\; \frac{2j_{a}+1}{4\pi}\; \frac{2j_{b}+1}{4\pi}}
       \sum_{\lambda_a,\,\lambda_b}\,
       \mathcal{A}_{\lambda_a\,\lambda_b}^{j_B\,j_a\,j_b}\,
       {D_{\lambda_B,\,\lambda_a-\lambda_b}^{j_B}}^*(\EulSet[B])
       \times \mathcal{B}_{\lambda_1\,\lambda_2}^{j_a}
       {D_{\lambda_a,\,\lambda_1-\lambda_2}^{j_a}}^*(\EulSet[a])
       \nonumber\\&\qquad\qquad\qquad\qquad\qquad\qquad\qquad\qquad\qquad
       \times (-1)^{j_b-\lambda_b}\ \mathcal{C}_{\lambda_3\,\lambda_4}^{j_b}
       {D_{\lambda_b,\,\lambda_3-\lambda_4}^{j_b}}^*(\EulSet[b])
\end{align}
The factors $\sqrt{\frac{2j_i+1}{4\pi}}$ follow from normalization of the amplitude.

Since only decays of spinless particles will be considered here, $j_B$ and
$\lambda_B=\lambda_a-\lambda_b$ are always zero. Therefore, the superscript $j_B$ on the amplitude
$\mathcal{A}$ will be dropped and the D-matrix of the $B$ decay reduces to
${D_{0,\,0}^{0}}^*(\EulSet[B]) = d_{0,\,0}^{0}(\theta_B) = 1$ (see also
Equation~\ref{eq:PYDDefinitions}). Introducing the definitions $\lambda_a=\lambda_b\equiv\lambda$,
$\lambda_1-\lambda_2\equiv\alpha$ and $\lambda_3-\lambda_4\equiv\beta$ and combining the three
separate amplitudes into one amplitude $A$, Equation~\ref{eq:amplitude} can be written as:
\begin{align}\label{eq:amplitude1}
  A &\propto \frac{1}{(4\pi)^{3/2}}\,
       \sum_{j_a,\,j_b}\,
       \sqrt{(2j_a+1)(2j_b+1)}\,
       \sum_\lambda\,
       (-1)^{j_b-\lambda}\,
       \mathcal{A}_\lambda^{j_a\,j_b}\, \mathcal{B}_\alpha^{j_a}\, \mathcal{C}_\beta^{j_b}\
       {D_{\lambda,\,\alpha}^{j_a}}^*(\EulSet[a])\,
       {D_{\lambda,\,\beta}^{j_b}}^*(\EulSet[b])
       \nonumber\\
    &\propto \frac{1}{(4\pi)^{3/2}}\,
       \sum_{j_a,\,j_b}\,
        \sqrt{(2j_a+1)(2j_b+1)}\,
       \sum_\lambda\,
       (-1)^{j_b-\lambda}\,
       A_{\lambda\,\alpha\,\beta}^{j_a\,j_b}\,
       {D_{\lambda,\,\alpha}^{j_a}}^*(\EulSet[a])\,
       {D_{\lambda,\,\beta}^{j_b}}^*(\EulSet[b])
\end{align}
%
To find an expression for $|A|^2$, the following relations are used:
%
\begin{subequations}\begin{align}
  &\qquad\qquad\qquad\qquad
    {D_{m,\,n}^{j}}^*(\EulSet) = (-1)^{m-n}\ D_{-m,\,-n}^{j}(\EulSet) \\
  &D_{m_1,\,n_1}^{j_1}(\EulSet)\,D_{m_2,\,n_2}^{j_2}(\EulSet)
    = \sum_{j_3=|j_1-j_2|}^{j_1+j_2}
    \langle j_1\,m_1,\,j_2\,m_2|j_3\ m_1+m_2 \rangle
    \nonumber\\&\qquad\qquad\qquad\qquad\qquad\qquad\quad\times
    \langle j_1\,n_1,\,j_2\,n_2|j_3\ n_1+n_2 \rangle\,
    D_{m_1+m_2,\,n_1+n_2}^{j_3}(\EulSet) \\
  &\qquad
    \langle j_1\,m_1,\,j_2\,m_2|j_3\ m_3 \rangle
    = (-1)^{j_1-j_2+m_3}\,\sqrt{2\,j_3+1}\
    \begin{pmatrix}
      \mss{j_1} & \mss{j_2} & \mss{j_3} \\
      \mss{m_1} & \mss{m_2} & \mss{-m_3}
    \end{pmatrix}
\end{align}\end{subequations}
%
where $\langle j_1\,m_1,\,j_2\,m_2|j_3\ m_3 \rangle$ is a Clebsch-Gordan coefficient and
$\begin{pmatrix} \mss{j_1} & \mss{j_2} & \mss{j_3} \\ \mss{m_1} & \mss{m_2} & \mss{-m_3}
\end{pmatrix}$ the corresponding Wigner 3j-symbol. The product of a D-matrix and a complex
conjugate D-matrix with equal indices $n$ can now be written as:
%
\begin{align}
  &D_{m_1,\,n}^{j_1}(\EulSet)\,{D_{m_2,\,n}^{j_2}}^*(\EulSet) \nonumber\\
  &\qquad= (-1)^{m_2-n}\ D_{m_1,\,n}^{j_1}(\EulSet)\,D_{-m_2,\,-n}^{j_2}(\EulSet) \nonumber\\
  &\qquad= (-1)^{m_2-n} \sum_{j_3=|j_1-j_2|}^{j_1+j_2}
    \langle j_1\,m_1,\,j_2\,-m_2|j_3\ m_1-m_2 \rangle\
    \langle j_1\,n,\,j_2\,-n|j_3\,0 \rangle\
    D_{m_1-m_2,\,0}^{j_3}(\EulSet) \nonumber\\
  &\qquad= (-1)^{m_1-n} \sum_{j_3=|j_1-j_2|}^{j_1+j_2} \left(2j_3+1\right)\
    \begin{pmatrix}
      \mss{j_1} & \mss{j_2} & \mss{j_3} \\
      \mss{m_1} & \mss{-m_2} & \mss{-m_1+m_2}
    \end{pmatrix}
    \begin{pmatrix}
      \mss{j_1} & \mss{j_2} & \mss{j_3} \\
      \mss{n} & \mss{-n} & \mss{0}
    \end{pmatrix}
    D_{m_1-m_2,\,0}^{j_3}(\EulSet)
\end{align}
%
With these relations, the square of the absolute value of the amplitude
(Equation~\ref{eq:amplitude1}) is given by:
%
\begin{subequations}\label{eq:ampSq}\begin{align}\label{eq:ampSqFirst}
  |A|^2 &\propto \frac{1}{(4\pi)^3}\,
           \sum_{j_a,\,j_a',\,j_b,\,j_b'}\
           \sqrt{(2j_a+1)(2j_a'+1)(2j_b+1)(2j_b'+1)}
           \nonumber\\&\quad\times
           \sum_{\lambda,\,\lambda'}\,
           (-1)^{j_b+j_b'-\lambda-\lambda'}\,
           {A_{\lambda\,\alpha\,\beta}^{j_a\,j_b}}^*\, A_{\lambda'\,\alpha\,\beta}^{j_a'\,j_b'}\
           D_{\lambda,\,\alpha}^{j_a}(\EulSet[a])\, {D_{\lambda',\,\alpha}^{j_a'}}^*(\EulSet[a])\
           D_{\lambda,\,\beta}^{j_b}(\EulSet[b])\, {D_{\lambda',\,\beta}^{j_b'}}^*(\EulSet[b])
           \\
        &\propto \frac{1}{(4\pi)^3}\,
           \sum_{j_a,\,j_a',\,j_b,\,j_b'}\
           \sqrt{(2j_a+1)(2j_a'+1)(2j_b+1)(2j_b'+1)}
           \nonumber\\&\quad\times
           \sum_{\lambda,\,\lambda'}\,
           (-1)^{j_b+j_b'-\lambda-\lambda'}\,
           {A_{\lambda\,\alpha\,\beta}^{j_a\,j_b}}^*\,A_{\lambda'\,\alpha\,\beta}^{j_a'\,j_b'}\,
           \nonumber\\&\quad\times
           (-1)^{\lambda-\alpha}
           \sum_{J_a=|j_a-j_a'|}^{j_a+j_a'} \left(2J_a+1\right)
           \begin{pmatrix}
             \mss{j_a} & \mss{j_a'} & \mss{J_a}\\
             \mss{\lambda} & \mss{-\lambda'} & \mss{-\lambda+\lambda'}
           \end{pmatrix}\,
           \begin{pmatrix}
             \mss{j_a} & \mss{j_a'} & \mss{J_a}\\
             \mss{\alpha} & \mss{-\alpha} & \mss{0}
           \end{pmatrix}\,
           D_{\lambda-\lambda',\,0}^{J_a}(\EulSet[a])
           \nonumber\\&\quad\times
           (-1)^{\lambda-\beta}
           \sum_{J_b=|j_b-j_b'|}^{j_b+j_b'} \left(2J_b+1\right)
           \begin{pmatrix}
             \mss{j_b} & \mss{j_b'} & \mss{J_b} \\
             \mss{\lambda} & \mss{-\lambda'} & \mss{-\lambda+\lambda'}
           \end{pmatrix}\,
           \begin{pmatrix}
             \mss{j_b} & \mss{j_b'} & \mss{J_b}\\
             \mss{\beta} & \mss{-\beta} & \mss{0}
           \end{pmatrix}\,
           D_{\lambda-\lambda',\,0}^{J_b}(\EulSet[b])
           \\ \label{eq:ampSqLast}
        &\propto \frac{1}{(4\pi)^3}\,
           \sum_{j_a,\,j_a',\,j_b,\,j_b',\,\lambda,\,\lambda'}\
           \sum_{J_a=|j_a-j_a'|}^{j_a+j_a'}\
           \sum_{J_b=|j_b-j_b'|}^{j_b+j_b'}\
           (-1)^{j_b+j_b'+m-\alpha-\beta}
           \nonumber\\&\quad\times
           \left(2J_a+1\right)\, \left(2J_b+1\right)\, \sqrt{(2j_a+1)(2j_a'+1)(2j_b+1)(2j_b'+1)}
           \nonumber\\&\quad\times
           \begin{pmatrix}
             \mss{j_a} & \mss{j_a'} & \mss{J_a}\\
             \mss{\lambda} & \mss{-\lambda'} & \mss{-m}
           \end{pmatrix}\,
           \begin{pmatrix}
             \mss{j_b} & \mss{j_b'} & \mss{J_b} \\
             \mss{\lambda} & \mss{-\lambda'} & \mss{-m}
           \end{pmatrix}\,
           \begin{pmatrix}
             \mss{j_a} & \mss{j_a'} & \mss{J_a}\\
             \mss{\alpha} & \mss{-\alpha} & \mss{0}
           \end{pmatrix}\,
           \begin{pmatrix}
             \mss{j_b} & \mss{j_b'} & \mss{J_b}\\
             \mss{\beta} & \mss{-\beta} & \mss{0}
           \end{pmatrix}
           \nonumber\\&\quad\times
           {A_{\lambda\,\alpha\,\beta}^{j_a\,j_b}}^*\, A_{\lambda'\,\alpha\,\beta}^{j_a'\,j_b'}\
           D_{m,\,0}^{J_a}(\EulSet[a])\, D_{m,\,0}^{J_b}(\EulSet[b])
  \quad,
\end{align}\end{subequations}
where $m\equiv\lambda-\lambda'$.

For each term in this sum there is a corresponding term for which the primed and unprimed indices
are swapped (except for the ``diagonal'' terms in which the primed and unprimed indices are equal).
The $(-1)^{j_b+\ldots}$ and the $\sqrt{(2j_a+1)\ldots\ }$ factors are equal for these two
corresponding terms. In the 3j-symbols, the first and second column are swapped and the signs of
the elements in the bottom row are flipped. These operations both give a factor $(-1)^{j+j'+J}$, so
also the corresponding 3j-symbols are equal. Combination of the ``off-diagonal terms'' yields
(without writing the common factors):
\begin{equation}\begin{split}\label{eq:angOffDiagonal}
  &{A_{\lambda\,\alpha\,\beta}^{j_a\,j_b}}^*\, A_{\lambda'\,\alpha\,\beta}^{j_a'\,j_b'}\
    D_{m,\,0}^{J_a}(\EulSet[a])\, D_{m,\,0}^{J_b}(\EulSet[b])
    + {A_{\lambda'\,\alpha\,\beta}^{j_a'\,j_b'}}^*\, A_{\lambda\,\alpha\,\beta}^{j_a\,j_b}\
    D_{-m,\,0}^{J_a}(\EulSet[a])\, D_{-m,\,0}^{J_b}(\EulSet[b])
  \\&\qquad\quad=
  {A_{\lambda\,\alpha\,\beta}^{j_a\,j_b}}^*\, A_{\lambda'\,\alpha\,\beta}^{j_a'\,j_b'}\
    D_{m,\,0}^{J_a}(\EulSet[a])\, D_{m,\,0}^{J_b}(\EulSet[b])
    + \left(
      {A_{\lambda\,\alpha\,\beta}^{j_a\,j_b}}^*\, A_{\lambda'\,\alpha\,\beta}^{j_a'\,j_b'}\
      D_{m,\,0}^{J_a}(\EulSet[a])\, D_{m,\,0}^{J_b}(\EulSet[b])
    \right)^*
  \\&\qquad\quad=
  2\,\Re\!\left(
      {A_{\lambda\,\alpha\,\beta}^{j_a\,j_b}}^*\, A_{\lambda'\,\alpha\,\beta}^{j_a'\,j_b'}\
      D_{m,\,0}^{J_a}(\EulSet[a])\, D_{m,\,0}^{J_b}(\EulSet[b])
    \right)
  \\&\qquad\quad=
  2\,\Re\!\left(
      {A_{\lambda\,\alpha\,\beta}^{j_a\,j_b}}^*\, A_{\lambda'\,\alpha\,\beta}^{j_a'\,j_b'}
    \right)\Re\!\left(
      D_{m,\,0}^{J_a}(\EulSet[a])\, D_{m,\,0}^{J_b}(\EulSet[b])
    \right)
  \\&\qquad\qquad\qquad\quad-
  2\,\Im\!\left(
      {A_{\lambda\,\alpha\,\beta}^{j_a\,j_b}}^*\, A_{\lambda'\,\alpha\,\beta}^{j_a'\,j_b'}
    \right)\Im\!\left(
      D_{m,\,0}^{J_a}(\EulSet[a])\, D_{m,\,0}^{J_b}(\EulSet[b])
    \right)
\end{split}\end{equation}
%
For the ``diagonal'' terms, the last line of Equation~\ref{eq:ampSqLast} can be written as
($m_\text{diagonal} = \lambda_\text{diagonal}-\lambda'_\text{diagonal} = 0$):
%
\begin{equation}\label{eq:angDiagonal}
  \left|A_{\lambda\,\alpha\,\beta}^{j_a\,j_b}\right|^2\,
  D_{0,\,0}^{J_a}(\EulSet[a])\, D_{0,\,0}^{J_b}(\EulSet[b])
\end{equation}

The products of D-matrices in Equations~\ref{eq:angOffDiagonal} and \ref{eq:angDiagonal} form the
angular basis functions in the $B$ decay rate. The sets of Euler angles $\EulSet[a]$ and
$\EulSet[b]$ translate into the helicity angles $\thetak$, $\thetal$ and $\phihel$. Only three
degrees of freedom are needed here, since we are only interested in the relative orientation of the
$a$ and $b$ decay products. This is reflected by the fact that the lower indices are equal for both
D-matrices, which enables combination of two of the remaining four angles\footnote{The
$\theta_B$/$\phi_B$ dependence was eliminated by the fact that the $B$ is a spin zero particle
(Equation~\ref{eq:amplitude1}).} ($\phi_a+\phi_b\equiv\phihel$, see also
Figure~\ref{fig:helFormFrames}c):
\begin{equation}\begin{split}
  D_{m,\,n}^{J_a}(\phi_a,\theta_a,-\phi_a)\, D_{m,\,n}^{J_b}(\phi_b,\theta_b,-\phi_b)
    &= e^{-im\phi_a}\, d_{m,\,n}^{J_a}(\theta_a)\, e^{in\phi_a}\
       e^{-im\phi_b}\, d_{m,\,n}^{J_b}(\theta_b)\, e^{in\phi_b} \\
    &= d_{m,\,n}^{J_a}(\theta_a)\ e^{-im\phihel}\, d_{m,\,n}^{J_b}(\theta_b)\, e^{in\phihel} \\
    &= d_{m,\,n}^{J_a}(\theta_a)\ D_{m,\,n}^{J_b}(\phihel,\theta_b,-\phihel)
\end{split}\end{equation}
The remaining two polar angles are identified with the helicity angles $\thetak$ and $\thetal$:
$\thetak\equiv\theta_a$ and $\thetal\equiv\theta_b$.

Contributions with different outgoing particle spins are not summed coherently in the amplitude.
Therefore, the second lower indices are equal for both pairs of D-matrices in
Equation~\ref{eq:ampSqFirst} ($\alpha$ and $\beta$). As a result, the second lower D-matrix indices
in Equations~\ref{eq:angOffDiagonal} and \ref{eq:angDiagonal} are zero. This makes it possible to
express the angular basis functions in terms of Legendre polynomials ($P_k$), associated Legendre
polynomials ($P_j^m$), spherical harmonics ($Y_j^m$) and real-valued spherical harmonics
($Y_{j\,m}$). The following definitions are applied:
\begin{subequations}\label{eq:PYDDefinitions}\begin{gather}
  P_j(x) = \frac{1}{2^j j!} \frac{\ud^j}{\ud x^j} \left(x^2-1\right)^j
    \qquad \text{with } j\ge0 \\
  P_j^m(x) = \left\{\begin{array}{cl}
    (-1)^m\,\left(1-x^2\right)^{\tfrac{1}{2}m}\,\frac{\ud^m}{\ud x^m}P_j(x)
      &\quad(m\ge0) \\
    \tfrac{(j-|m|)!}{(j+|m|)!}\,\left(1-x^2\right)^{\tfrac{1}{2}|m|}\,
      \frac{\ud^{|m|}}{\ud x^{|m|}}P_j(x)
      &\quad(m<0)
   \end{array}\right.\\
  Y_j^m(\theta,\phi) = \sqrt{\tfrac{2j+1}{4\pi}\tfrac{(j-m)!}{(j+m)!}}\
    P_l^m(\cos\theta)\, e^{im\phi} \\
  Y_{j,\,m}(\theta,\phi) = \left\{\begin{array}{cl}
     Y_j^0(\theta,\phi)
       &\quad(m=0) \\
     \sqrt{2}\ \Re\!\left(Y_j^m(\theta,\phi)\right)
       &\quad(m>0) \\
     \sqrt{2}\ \Im\!\left(Y_j^{|m|}(\theta,\phi)\right)
       &\quad(m<0)
   \end{array}\right. \\
  d_{m,\,0}^j(\theta) = \sqrt{\tfrac{(j-m)!}{(j+m)!}}\ P_j^m(\cos\theta) \\
  D_{m,\,0}^j(\phi,\theta,\gamma) = \sqrt{\tfrac{4\pi}{2j+1}}\ {Y_j^m}^*(\theta,\phi)
\end{gather}\end{subequations}
Angular functions in terms of associated Legendre functions and real-valued spherical harmonics and
their integrals are listed in Appendix~\ref{sec:angBasisFuncTables}. 

With Equation~\ref{eq:PYDDefinitions} and the transformation from $\EulSet[a]$ and $\EulSet[b]$ to
the helicity angles, the angular basis functions may be expressed as:
\begin{subequations}\begin{alignat}{2}
  &m=0: &&\nonumber\\
  &F^S_{J_a,\,J_b}(\helSet)
    &&\equiv \tfrac{1}{4\pi}\,\sqrt{(2J_a+1)(2J_b+1)}\cdot
      D_{0,\,0}^{J_a}(\EulSet[a])\, D_{0,\,0}^{J_b}(\EulSet[b])
    \nonumber\\
    &&&= \tfrac{1}{4\pi}\,\sqrt{(2J_a+1)(2J_b+1)}\
      d_{0\,0}^{J_a}(\thetak)\ D_{0\,0}^{J_b}(\phihel,\thetal,-\phihel)
    \nonumber\\
    &&&= \sqrt{\tfrac{2J_a+1}{4\pi}}\ P_{J_a}^0(\cthetak)\ Y_{J_b,\,0}(\thetal,\phihel) \\
  &F^\Re_{J_a,\,J_b,\,0}(\helSet)
    &&\equiv \tfrac{1}{4\pi}\,\sqrt{(2J_a+1)(2J_b+1)}\cdot
      2\,\Re\!\left(D_{0,\,0}^{J_a}(\EulSet[a])\, D_{0,\,0}^{J_b}(\EulSet[b])\right)
    \nonumber\\
    &&&= \tfrac{2}{4\pi}\,\sqrt{(2J_a+1)(2J_b+1)}\
      \Re\!\left(d_{0,\,0}^{J_a}(\thetak)\, D_{0,\,0}^{J_b}(\phihel,\thetal,-\phihel)\right)
    \nonumber\\
    &&&= 2\,\sqrt{\tfrac{2J_a+1}{4\pi}}\
      P_{J_a}^0(\cthetak)\ Y_{J_b,\,0}(\thetal,\phihel) \\
  &F^\Im_{J_a,\,J_b,\,0}(\helSet)
    &&\equiv \tfrac{1}{4\pi}\,\sqrt{(2J_a+1)(2J_b+1)}\cdot
      -2\,\Im\!\left(D_{0,\,0}^{J_a}(\EulSet[a])\, D_{0,\,0}^{J_b}(\EulSet[b])\right)
    \nonumber\\
  &&&= 0 \\
  &&&\nonumber\\&m>0: &&\nonumber\\
  &F^\Re_{J_a,\,J_b,\,m}(\helSet)
    &&\equiv \tfrac{1}{4\pi}\,\sqrt{(2J_a+1)(2J_b+1)}\cdot
      2\,\Re\!\left(D_{m,\,0}^{J_a}(\EulSet[a])\, D_{m,\,0}^{J_b}(\EulSet[b])\right)
    \nonumber\\
    &&&= \tfrac{2}{4\pi}\,\sqrt{(2J_a+1)(2J_b+1)}\
      \Re\!\left(d_{m,\,0}^{J_a}(\thetak)\, D_{m,\,0}^{J_b}(\phihel,\thetal,-\phihel)\right)
    \nonumber\\
    &&&= \sqrt{2}\,\sqrt{\tfrac{2J_a+1}{4\pi}\tfrac{(J_a-m)!}{(J_a+m)!}}\
      P_{J_a}^m(\cthetak)\ Y_{J_b,\,m}(\thetal,\phihel) \\
  &F^\Im_{J_a,\,J_b,\,m}(\helSet)
    &&\equiv \tfrac{1}{4\pi}\,\sqrt{(2J_a+1)(2J_b+1)}\cdot
      -2\,\Im\!\left(D_{m,\,0}^{J_a}(\EulSet[a])\, D_{m,\,0}^{J_b}(\EulSet[b])\right)
    \nonumber\\
  &&&= \sqrt{2}\,\sqrt{\tfrac{2J_a+1}{4\pi}\tfrac{(J_a-m)!}{(J_a+m)!}}\
      P_{J_a}^m(\cthetak)\ Y_{J_b,\,-m}(\thetal,\phihel) \\
  &&&\nonumber\\&m<0: &&\nonumber\\
  &F^\Re_{J_a,\,J_b,\,m}(\helSet)
    &&\equiv \tfrac{1}{4\pi}\,\sqrt{(2J_a+1)(2J_b+1)}\cdot
      2\,\Re\!\left({D_{|m|,\,0}^{J_a}}^*(\EulSet[a])\, {D_{|m|,\,0}^{J_b}}^*(\EulSet[b])\right)
    \nonumber\\
    &&&= 2\, \sqrt{\tfrac{2J_a+1}{4\pi}\tfrac{(J_a-|m|)!}{(J_a+|m|)!}}\
      P_{J_a}^{|m|}(\cthetak)\
      \Re\!\left(Y_{J_b}^{|m|}(\thetal,\phihel)\right)
    \nonumber\\
    &&&= \sqrt{2}\,\sqrt{\tfrac{2J_a+1}{4\pi}\tfrac{(J_a-|m|)!}{(J_a+|m|)!}}\
      P_{J_a}^{|m|}(\cthetak)\ Y_{J_b,\,|m|}(\thetal,\phihel) \\
  &F^\Im_{J_a,\,J_b,\,m}(\helSet)
    &&\equiv \tfrac{1}{4\pi}\,\sqrt{(2J_a+1)(2J_b+1)}\cdot
      -2\,\Im\!\left({D_{|m|,\,0}^{J_a}}^*(\EulSet[a])\, {D_{|m|,\,0}^{J_b}}^*(\EulSet[b])\right)
    \nonumber\\
  &&&= -\sqrt{2}\,\sqrt{\tfrac{2J_a+1}{4\pi}\tfrac{(J_a-|m|)!}{(J_a+|m|)!}}\
      P_{J_a}^{|m|}(\cthetak)\ Y_{J_b,\,-|m|}(\thetal,\phihel)
\end{alignat}\end{subequations}
%
This yields the following angular basis functions for the different combinations of amplitudes
$|A_i|^2$ (\ref{eq:angBasisFuncsS}), $\Re(A_i^*A_j)$ (\ref{eq:angBasisFuncsR}) and $\Im(A_i^*A_j)$
(\ref{eq:angBasisFuncsI}):
\begin{subequations} \label{eq:angBasisFuncs} \begin{align}
  \label{eq:angBasisFuncsS}
  F^S_{J_a,\,J_b}(\helSet)
    &= \sqrt{\tfrac{2J_a+1}{4\pi}}\ P_{J_a}^0(\cthetak)\ Y_{J_b,\,0}(\thetal,\phihel)
  \\ \label{eq:angBasisFuncsR}
  F^\Re_{J_a,\,J_b,\,m}(\helSet)
    &= \left\{\begin{array}{cl}
      2\,\sqrt{\tfrac{2J_a+1}{4\pi}}\ P_{J_a}^0(\cthetak)\ Y_{J_b,\,0}(\thetal,\phihel)
        & (m=0) \\
      \sqrt{2}\,\sqrt{\tfrac{2J_a+1}{4\pi}\tfrac{(J_a-|m|)!}{(J_a+|m|)!}}\
        P_{J_a}^{|m|}(\cthetak)\ Y_{J_b,\,|m|}(\thetal,\phihel)
        & (m\ne0)
    \end{array}\right.\\ \label{eq:angBasisFuncsI}
  F^\Im_{J_a,\,J_b,\,m}(\helSet)
    &= \left\{\begin{array}{cl}
      0 & (m=0) \\
      \tfrac{m}{|m|}\,\sqrt{2}\,\sqrt{\tfrac{2J_a+1}{4\pi}\tfrac{(J_a-|m|)!}{(J_a+|m|)!}}\
        P_{J_a}^{|m|}(\cthetak)\ Y_{J_b,\,-|m|}(\thetal,\phihel)
        & (m\ne0)
    \end{array}\right.
\end{align}\end{subequations}

It follows from the sums in Equation~\ref{eq:ampSq} that each combination of amplitudes enters with
a sum of different angular basis functions for different values of $J_a$ and $J_b$. Each of these
terms has a coefficient, which depends on $j_a$, $j_a'$, $j_b$, $j_b'$, $J_a$, $J_b$, $\lambda$,
$\lambda'$, $\alpha$ and $\beta$.  After multiplying the right-hand side of Equation~\ref{eq:ampSq}
with $\tfrac{1}{2} (4\pi)^2$ to obtain a convenient normalization
($\Gamma=|A_i|^2+|A_j|^2+\ldots$), the coefficients are given by:
\begin{equation} \label{eq:angCoefs} \begin{split}
  C_{j_a,\,j_a',\,j_b,\,j_b',\,J_a,\,J_b}^{\lambda,\,\lambda',\,\alpha,\,\beta}
  &= \tfrac{1}{2}\, (-1)^{j_b+j_b'+m-\alpha-\beta}
  \\&\quad\times
  \sqrt{(2j_a+1)(2j_a'+1)(2j_b+1)(2j_b'+1)(2J_a+1)(2J_b+1)}
  \\&\quad\times
  \begin{pmatrix}
    \mss{j_a} & \mss{j_a'} & \mss{J_a}\\
    \mss{\lambda} & \mss{-\lambda'} & \mss{-m}
  \end{pmatrix}\,
  \begin{pmatrix}
    \mss{j_b} & \mss{j_b'} & \mss{J_b} \\
    \mss{\lambda} & \mss{-\lambda'} & \mss{-m}
  \end{pmatrix}\,
  \begin{pmatrix}
    \mss{j_a} & \mss{j_a'} & \mss{J_a}\\
    \mss{\alpha} & \mss{-\alpha} & \mss{0}
  \end{pmatrix}\,
  \begin{pmatrix}
    \mss{j_b} & \mss{j_b'} & \mss{J_b}\\
    \mss{\beta} & \mss{-\beta} & \mss{0}
  \end{pmatrix}
\end{split} \end{equation}


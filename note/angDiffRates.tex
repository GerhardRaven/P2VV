\subsection{Differential Decay Rates} \label{sec:angDiffRates}

%%%%%%%%%%%%%%%%%%%%%%%%%%%%%%%%%%%%%%%%%%%%%%%%%%%%%%%%%%%%%%%%%%%%%%%%%%%%%%%%%%%%%%%%%%%%%%%%%%%
\subsubsection{P $\to$ VV} \label{sec:P2VV}
For the decay of a pseudoscalar particle via two vector particles (\PVV), both $j_a$ and $j_b$
(Equation~\ref{eq:amplitude1}) are equal to one. Since $\lambda$ is the spin projection of the
vector particles, it can assume the values 0 and $\pm1$. Thus, three helicity amplitudes are
defined for a given final state with $\lambda_1 - \lambda_2 \equiv \alpha$ and $\lambda_3 -
\lambda_4 \equiv \beta$:
\begin{equation} \label{eq:P2VVHelAmps}
  H_0 \equiv A_{0,\,\alpha,\,\beta}^{1,\,1}
  \qquad\qquad H_+ \equiv A_{+1,\,\alpha,\,\beta}^{1,\,1}
  \qquad\qquad H_- \equiv A_{-1,\,\alpha,\,\beta}^{1,\,1}
\end{equation}

Amplitudes of the \PVV{} decay are often reparameterized in terms of \emph{transversity
amplitudes}:
\begin{equation} \label{eq:P2VVTrAmps}
  A_0 \equiv H_0
  \qquad\qquad A_\parallel \equiv \tfrac{1}{\sqrt{2}}\, (H_+ + H_-)
  \qquad\qquad A_\perp \equiv \tfrac{1}{\sqrt{2}}\, (H_+ - H_-)
\end{equation}
Transversity amplitudes correspond to the linear polarization states of the two vector particles.
The polarization (three-)vectors can be longitudinal or transverse to the decay axis. In the state
in which both vector particles are longitudinal (0), their spins have no component in the direction
of the decay axis ($\lambda=0$). In the transverse states, the polarization vectors may either be
parallel ($\parallel$) or perpendicular ($\perp$) to one another. 

The decay may also be decomposed in terms of partial waves. Each wave corresponds to a state in
which the system of the two vector particles has a definite orbital angular momentum. Amplitudes
corresponding to the S-wave ($L=0$), P-wave ($L=1$) and D-wave ($L=2$) are given by:
\begin{alignat}{2} \label{eq:P2VVPWAmps}
  W_\text{S} &\equiv -\sqrt{\tfrac{1}{3}}\, H_0 + \sqrt{\tfrac{1}{3}}\, (H_+ + H_-)
    && = - \sqrt{\tfrac{1}{3}}\, A_0 + \sqrt{\tfrac{2}{3}}\, A_\parallel
    \nonumber\\
  W_\text{P} &\equiv \sqrt{\tfrac{1}{2}}\, (H_+ - H_-)
    && = A_\perp
    \\
  W_\text{D} &\equiv \sqrt{\tfrac{2}{3}}\, H_0 + \sqrt{\tfrac{1}{6}}\, (H_+ + H_-)
    && = \sqrt{\tfrac{2}{3}}\, A_0 + \sqrt{\tfrac{1}{3}}\, A_\parallel
    \nonumber
\end{alignat}

Differential decay rates in terms of helicity angles and helicity amplitudes are found by
evaluating the coefficients of the angular basis functions (Equations~\ref{eq:angBasisFuncs} and
\ref{eq:angCoefs}). The angular functions for different combinations of amplitudes are listed in
Tables~\ref{tab:angDistP2VV0Hel} and \ref{tab:angDistP2VV1Hel}. Functions for transversity
amplitudes are found by applying the definitions of Equation~\ref{eq:P2VVTrAmps}. Corresponding
tables with transversity amplitudes and transversity angles can be found in
Appendix~\ref{sec:angTransFuncTables}. See Table~\ref{tab:helTransConv} in
Appendix~\ref{sec:angBasisFuncTables} for the transformation between helicity and transversity
angles of some relevant functions.
\begin{table}[htbp]
  \centering \footnotesize
  \begin{tabular}{| c | c | c |}
    \hline
    amplitudes                              &
      $f(\helSet) \times 16\sqrt{\pi}$      &
      $f(\helSet) \times \tfrac{32\pi}{9}$  \\

    \hline\hline

    \AmpSq[H]{0}  &
      $4\, (P_0^0 + 2\, P_2^0)\,
        (Y_{0,\,0} + 2\tfrac{1}{\sqrt{5}}\, Y_{2,\,0})$  &
      $4\, \cos^2\thetak\, \cos^2\thetal$  \\
    \hline

    \AmpSq[H]{\pm}  &
      $4\, (P_0^0 - P_2^0)\,
        (Y_{0,\,0} - \tfrac{1}{\sqrt{5}}\, Y_{2,\,0})$  &
      $\sin^2\thetak\, \sin^2\thetal$  \\
      &
      $\qquad = 2\, P_2^2\,
        (Y_{0,\,0} - \tfrac{1}{\sqrt{5}}\, Y_{2,\,0})$  &
      \\
    \hline

    \ReAmp[H][H]{0}{+}  &
      $-4\sqrt{\tfrac{3}{5}}\, P_2^1\, Y_{2,\,+1}$  &
      $-\sin2\thetak\, \sin2\thetal\, \cos\phihel$  \\
    \hline

    \ImAmp[H][H]{0}{+}  &
      $+4\sqrt{\tfrac{3}{5}}\, P_2^1\, Y_{2,\,-1}$  &
      $+\sin2\thetak\, \sin2\thetal\, \sin\phihel$  \\
    \hline

    \ReAmp[H][H]{0}{-}  &
      $-4\sqrt{\tfrac{3}{5}}\, P_2^1\, Y_{2,\,+1}$  &
      $-\sin2\thetak\, \sin2\thetal\, \cos\phihel$  \\
    \hline

    \ImAmp[H][H]{0}{-}  &
      $-4\sqrt{\tfrac{3}{5}}\, P_2^1\, Y_{2,\,-1}$  &
      $-\sin2\thetak\, \sin2\thetal\, \sin\phihel$  \\
    \hline

    \ReAmp[H][H]{+}{-}  &
      $+4\sqrt{\tfrac{3}{5}}\, P_2^2\, Y_{2,\,+2}$  &
      $+2\, \sin^2\thetak\, \sin^2\thetal\, \cos2\phihel$  \\
    \hline

    \ImAmp[H][H]{+}{-}  &
      $+4\sqrt{\tfrac{3}{5}}\, P_2^2\, Y_{2,\,-2}$  &
      $+2\, \sin^2\thetak\, \sin^2\thetal\, \sin2\phihel$  \\
    \hline\hline

    \AmpSq{0}  &
      $4\, (P_0^0 + 2\, P_2^0)\,
        (Y_{0,\,0} + 2\tfrac{1}{\sqrt{5}}\, Y_{2,\,0})$  &
      $4\, \cos^2\thetak\, \cos^2\thetal$  \\
    &
      $= 2\, (P_0^0 + 2\, P_2^0)\,
        (3\, Y_{0,\,0} - Y_{0,\,0} + 4\tfrac{1}{\sqrt{5}}\, Y_{2,\,0})$  &
      $= 2\, \cos^2\thetak\, (1 + \cos2\thetal)$  \\
    \hline

    \AmpSq{\parallel}  &
      $2\, P_2^2\,
        (Y_{0,\,0} - \tfrac{1}{\sqrt{5}}\, Y_{2,\,0} + \sqrt{\tfrac{3}{5}}\, Y_{2,\,+2})$  &
      $2\, \sin^2\thetak\, \sin^2\thetal\, \cos^2\phihel$  \\
    &
      $= \tfrac{1}{2}\, P_2^2\,
        (3\, Y_{0,\,0} + Y_{0,\,0} - 4\tfrac{1}{\sqrt{5}}\, Y_{2,\,0}
        + 4\sqrt{\tfrac{3}{5}}\, Y_{2,\,+2})$  &
      $= \tfrac{1}{2}\, \sin^2\thetak\, (1 - \cos2\thetal + 2\, \sin^2\thetal\, \cos2\phihel)$  \\
    \hline

    \AmpSq{\perp}  &
      $2\, P_2^2\,
        (Y_{0,\,0} - \tfrac{1}{\sqrt{5}}\, Y_{2,\,0} - \sqrt{\tfrac{3}{5}}\, Y_{2,\,+2})$  &
      $2\, \sin^2\thetak\, \sin^2\thetal\, \sin^2\phihel$  \\
    &
      $= \tfrac{1}{2}\, P_2^2\,
        (3\, Y_{0,\,0} + Y_{0,\,0} - 4\tfrac{1}{\sqrt{5}}\, Y_{2,\,0}
        - 4\sqrt{\tfrac{3}{5}}\, Y_{2,\,+2})$  &
      $= \tfrac{1}{2}\, \sin^2\thetak\, (1 - \cos2\thetal - 2\, \sin^2\thetal\, \cos2\phihel)$  \\
    \hline

    \ReAmp{0}{\parallel}  &
      $-4\sqrt{2}\sqrt{\tfrac{3}{5}}\, P_2^1\, Y_{2,\,+1}$  &
      $-\sqrt{2}\, \sin2\thetak\, \sin2\thetal\, \cos\phihel$  \\
    \hline

    \ImAmp{0}{\parallel}  &
      0  &
      0  \\
    \hline

    \ReAmp{0}{\perp}  &
      0  &
      0  \\
    \hline

    \ImAmp{0}{\perp}  &
      $+4\sqrt{2}\sqrt{\tfrac{3}{5}}\, P_2^1\, Y_{2,\,-1}$  &
      $+\sqrt{2}\, \sin2\thetak\, \sin2\thetal\, \sin\phihel$  \\
    \hline

    \ReAmp{\parallel}{\perp}  &
      0  &
      0  \\
    \hline

    \ImAmp{\parallel}{\perp}  &
      $-4\sqrt{\tfrac{3}{5}}\, P_2^2\, Y_{2,\,-2}$  &
      $-2\, \sin^2\thetak\, \sin^2\thetal\, \sin2\phihel$  \\
    \hline
  \end{tabular}

  \caption{Angular functions in helicity angles for a \PVV{} decay in which the difference of the
    outgoing particle helicities is zero on both sides ($\alpha=\beta=0$). (See
    Table~\ref{tab:angDistP2VV0Tr} on page~\pageref{tab:angDistP2VV0Tr} for these functions
    in transversity angles.)}
  \label{tab:angDistP2VV0Hel}
\end{table}
%
\begin{table}[htbp]
  \centering \footnotesize
  \begin{tabular}{| c | c | c |}
    \hline
    amplitudes                              &
      $f(\helSet) \times 16\sqrt{\pi}$      &
      $f(\helSet) \times \tfrac{32\pi}{9}$  \\

    \hline\hline

    \AmpSq[H]{0}  &
      $4\, (P_0^0 + 2\, P_2^0)\,
        (Y_{0,\,0} - \tfrac{1}{\sqrt{5}}\, Y_{2,\,0})$  &
      $2\, \cos^2\thetak\, \sin^2\thetal$  \\
    \hline

    \AmpSq[H]{+}  &
      $2\, (P_0^0 - P_2^0)\,
        (2\, Y_{0,\,0} + \tfrac{1}{\sqrt{5}}\, Y_{2,\,0}
        \pm \sqrt{3}\, Y_{1,\,0})$  &
      $\tfrac{1}{2}\, \sin^2\thetak\, (1 \pm \cos\thetal)^2$  \\
      &
      $\qquad\quad = P_2^2\,
        (2\, Y_{0,\,0} + \tfrac{1}{\sqrt{5}}\, Y_{2,\,0}
        \pm \sqrt{3}\, Y_{1,\,0})$  &
      \\
    \hline

    \AmpSq[H]{-}  &
      $2\, (P_0^0 - P_2^0)\,
        (2\, Y_{0,\,0} + \tfrac{1}{\sqrt{5}}\, Y_{2,\,0}
        \mp \sqrt{3}\, Y_{1,\,0})$  &
      $\tfrac{1}{2}\, \sin^2\thetak\, (1 \mp \cos\thetal)^2$  \\
      &
      $\qquad\quad = P_2^2\,
        (2\, Y_{0,\,0} + \tfrac{1}{\sqrt{5}}\, Y_{2,\,0}
        \mp \sqrt{3}\, Y_{1,\,0})$  &
      \\
    \hline

    \ReAmp[H][H]{0}{+}  &
      $+2\sqrt{\tfrac{3}{5}}\, P_2^1 (Y_{2,\,+1} \pm \sqrt{5}\, Y_{1,\,+1})$  &
      $\pm \sin2\thetak\, \sin\thetal\, (1 \pm \cos\thetal) \cos\phihel$  \\
    \hline

    \ImAmp[H][H]{0}{+}  &
      $-2\sqrt{\tfrac{3}{5}}\, P_2^1 (Y_{2,\,-1} \pm \sqrt{5}\, Y_{1,\,-1})$  &
      $\mp \sin2\thetak\, \sin\thetal\, (1 \pm \cos\thetal) \sin\phihel$  \\
    \hline

    \ReAmp[H][H]{0}{-}  &
      $+2\sqrt{\tfrac{3}{5}}\, P_2^1 (Y_{2,\,+1} \mp \sqrt{5}\, Y_{1,\,+1})$  &
      $\mp \sin2\thetak\, \sin\thetal\, (1 \mp \cos\thetal) \cos\phihel$  \\
    \hline

    \ImAmp[H][H]{0}{-}  &
      $+2\sqrt{\tfrac{3}{5}}\, P_2^1 (Y_{2,\,-1} \mp \sqrt{5}\, Y_{1,\,-1})$  &
      $\mp \sin2\thetak\, \sin\thetal\, (1 \mp \cos\thetal) \sin\phihel$  \\
    \hline

    \ReAmp[H][H]{+}{-}  &
      $-2\sqrt{\tfrac{3}{5}}\, P_2^2\, Y_{2,\,+2}$  &
      $-\sin^2\thetak\, \sin^2\thetal\, \cos2\phihel$  \\
    \hline

    \ImAmp[H][H]{+}{-}  &
      $-2\sqrt{\tfrac{3}{5}}\, P_2^2\, Y_{2,\,-2}$  &
      $-\sin^2\thetak\, \sin^2\thetal\, \sin2\phihel$  \\
    \hline\hline

    \AmpSq{0}  &
      $4\, (P_0^0 + 2\, P_2^0)\,
        (Y_{0,\,0} - \tfrac{1}{\sqrt{5}}\, Y_{2,\,0})$  &
      $2\, \cos^2\thetak\, \sin^2\thetal$  \\
    &
      $= (P_0^0 + 2\, P_2^0)\,
        (3\, Y_{0,\,0} + Y_{0,\,0} - 4\tfrac{1}{\sqrt{5}}\, Y_{2,\,0})$  &
      $= \cos^2\thetak\, (1 - \cos2\thetal)$  \\
    \hline

    \AmpSq{\parallel}  &
      $P_2^2\,
        (2\, Y_{0,\,0} + \tfrac{1}{\sqrt{5}}\, Y_{2,\,0} - \sqrt{\tfrac{3}{5}}\, Y_{2,\,+2})$  &
      $\sin^2\thetak\, (1 - \sin^2\thetal\, \cos^2\phihel)$  \\
    &
      $= \tfrac{1}{4}\, P_2^2\,
        (9\, Y_{0,\,0} - Y_{0,\,0} + 4\tfrac{1}{\sqrt{5}}\, Y_{2,\,0}
        - 4\sqrt{\tfrac{3}{5}}\, Y_{2,\,+2})$  &
      $= \tfrac{1}{4}\, \sin^2\thetak\, (3 + \cos2\thetal - 2\, \sin^2\thetal\, \cos2\phihel)$  \\
    \hline

    \AmpSq{\perp}  &
      $P_2^2\,
        (2\, Y_{0,\,0} + \tfrac{1}{\sqrt{5}}\, Y_{2,\,0} + \sqrt{\tfrac{3}{5}}\, Y_{2,\,+2})$  &
      $\sin^2\thetak\, (1 - \sin^2\thetal\, \sin^2\phihel)$  \\
    &
      $= \tfrac{1}{4}\, P_2^2\,
        (9\, Y_{0,\,0} - Y_{0,\,0} + 4\tfrac{1}{\sqrt{5}}\, Y_{2,\,0}
        + 4\sqrt{\tfrac{3}{5}}\, Y_{2,\,+2})$  &
      $= \tfrac{1}{4}\, \sin^2\thetak\, (3 + \cos2\thetal + 2\, \sin^2\thetal\, \cos2\phihel)$  \\
    \hline

    \ReAmp{0}{\parallel}  &
      $+2\sqrt{2}\sqrt{\tfrac{3}{5}}\, P_2^1\, Y_{2,\,+1}$  &
      $+\frac{1}{\sqrt{2}}\, \sin2\thetak\, \sin2\thetal\, \cos\phihel$  \\
    \hline

    \ImAmp{0}{\parallel}  &
      $\mp 2\sqrt{2}\;\sqrt{3}\, P_2^1\, Y_{1,\,-1}$  &
      $\mp \sqrt{2}\, \sin2\thetak\, \sin\thetal\, \sin\phihel$  \\
    \hline

    \ReAmp{0}{\perp}  &
      $\pm 2\sqrt{2}\;\sqrt{3}\, P_2^1\, Y_{1,\,+1}$  &
      $\pm \sqrt{2}\, \sin2\thetak\, \sin\thetal\, \cos\phihel$  \\
    \hline

    \ImAmp{0}{\perp}  &
      $-2\sqrt{2}\sqrt{\tfrac{3}{5}}\, P_2^1\, Y_{2,\,-1}$  &
      $-\frac{1}{\sqrt{2}}\, \sin2\thetak\, \sin2\thetal\, \sin\phihel$  \\
    \hline

    \ReAmp{\parallel}{\perp}  &
      $\pm 2\sqrt{3}\, P_2^2\, Y_{1,\,0}$  &
      $\pm 2\, \sin^2\thetak\, \cos\thetal$  \\
    \hline

    \ImAmp{\parallel}{\perp}  &
      $+2\sqrt{\tfrac{3}{5}}\, P_2^2\, Y_{2,\,-2}$  &
      $+\sin^2\thetak\, \sin^2\thetal\, \sin2\phihel$  \\
    \hline
  \end{tabular}

  \caption{Angular functions in helicity angles for a \PVV{} decay in which the difference of the
    helicities of particles 1 and 2 is zero and the difference of the helicities of particles 3 and
    4 is plus or minus one ($\alpha=0$ and $\beta=\pm1$). (See Table~\ref{tab:angDistP2VV1Tr} on
    page~\pageref{tab:angDistP2VV1Tr} for these functions in transversity angles.)}
  \label{tab:angDistP2VV1Hel}
\end{table}

Table~\ref{tab:angDistP2VV0Hel} shows the angular functions for the case $\alpha=\beta=0$. That is,
the difference of the final state particle helicities is zero on both sides of the decay
($\lambda_1=\lambda_2$ and $\lambda_3=\lambda_4$). Examples of such a process are \PVV{} decays in
which all final state particles are (pseudo)scalars (e.g. $\B[s][0] \to \phi(\to \kaon[+]\kaon[-])\
\phi(\to \kaon[+]\kaon[-])$), but also $\B[][0] \to V(\to\muon[+]\muon[-])\ V'(\to PP')$ when the
(massive) muons have equal helicities.

The functions shown in Table~\ref{tab:angDistP2VV1Hel} are for $\alpha=0$, $\beta=\pm1$. This is
the configuration of a decay with (pseudo)scalars on one side and a lepton/antilepton pair on the
other side. In this case, the leptons are in a ``normal'' configuration with opposite helicities.
Examples are \BdtoJpsimumuKstKpi{} and \BstoJpsimumuphiKK.

%%%%%%%%%%%%%%%%%%%%%%%%%%%%%%%%%%%%%%%%%%%%%%%%%%%%%%%%%%%%%%%%%%%%%%%%%%%%%%%%%%%%%%%%%%%%%%%%%%%
\subsubsection{Spinless Intermediate States} \label{sec:spinlessAB}
The \PVV{} decay can be modified by replacing one of the vector particles by a spinless particle.
Only one amplitude can be defined in this case:
\begin{subequations} \begin{align} \label{eq:spinlessAmp}
  A_{\text{S}_a} &\equiv A_{0,\,\alpha,\,\beta}^{0,\,1}
    \qquad\qquad(\text{spinless }a\text{ and vector }b)  \\
  A_{\text{S}_b} &\equiv A_{0,\,\alpha,\,\beta}^{1,\,0}
    \qquad\qquad(\text{vector }a\text{ and spinless }b)
\end{align} \end{subequations}

\begin{table}[htbp]
  \centering \footnotesize
  \begin{tabular}{| c | c | c |}
    \hline
    amplitudes                              &
      $f(\helSet) \times 16\sqrt{\pi}$      &
      $f(\helSet) \times \tfrac{32\pi}{9}$  \\

    \hline\hline

    \AmpSq{{\text{S}_a}}  &
      $4\, P_0^0\,
        (Y_{0,\,0} + 2\tfrac{1}{\sqrt{5}}\, Y_{2,\,0})$  &
      $\tfrac{4}{3}\, \cos^2\thetal$  \\
    \hline

    \ReAmp[H]{0}{{\text{S}_a}}  &
      $8\sqrt{3}\, P_1^0\, (Y_{0,\,0} + 2\tfrac{1}{\sqrt{5}}\, Y_{2,\,0})$  &
      $\tfrac{8}{3}\sqrt{3}\, \cos\thetak\, \cos^2\thetal$  \\
    \hline

    \ImAmp[H]{0}{{\text{S}_a}}  &
      0  &
      0  \\
    \hline

    \ReAmp[H]{+}{{\text{S}_a}}  &
      $-12\tfrac{1}{\sqrt{5}}\, P_1^1\, Y_{2,\,+1}$  &
      $-\tfrac{2}{3}\sqrt{3}\, \sin\thetak\, \sin2\thetal\, \cos\phihel$  \\
    \hline

    \ImAmp[H]{+}{{\text{S}_a}}  &
      $-12\tfrac{1}{\sqrt{5}}\, P_1^1\, Y_{2,\,-1}$  &
      $-\tfrac{2}{3}\sqrt{3}\, \sin\thetak\, \sin2\thetal\, \sin\phihel$  \\
    \hline

    \ReAmp[H]{-}{{\text{S}_a}}  &
      $-12\tfrac{1}{\sqrt{5}}\, P_1^1\, Y_{2,\,+1}$  &
      $-\tfrac{2}{3}\sqrt{3}\, \sin\thetak\, \sin2\thetal\, \cos\phihel$  \\
    \hline

    \ImAmp[H]{-}{{\text{S}_a}}  &
      $+12\tfrac{1}{\sqrt{5}}\, P_1^1\, Y_{2,\,-1}$  &
      $+\tfrac{2}{3}\sqrt{3}\, \sin\thetak\, \sin2\thetal\, \sin\phihel$  \\
    \hline\hline

    \AmpSq{{\text{S}_a}}  &
      $4\, P_0^0\, (Y_{0,\,0} + 2\tfrac{1}{\sqrt{5}}\, Y_{2,\,0})$  &
      $\tfrac{4}{3}\, \cos^2\thetal$  \\
    \hline

    \ReAmp{0}{{\text{S}_a}}  &
      $8\sqrt{3}\, P_1^0\, (Y_{0,\,0} + 2\tfrac{1}{\sqrt{5}}\, Y_{2,\,0})$  &
      $\tfrac{8}{3}\sqrt{3}\, \cos\thetak\, \cos^2\thetal$  \\
    \hline

    \ImAmp{0}{{\text{S}_a}}  &
      0  &
      0  \\
    \hline

    \ReAmp{\parallel}{{\text{S}_a}}  &
      $-12\sqrt{2}\tfrac{1}{\sqrt{5}}\, P_1^1\, Y_{2,\,+1}$  &
      $-\tfrac{2}{3}\sqrt{6}\, \sin\thetak\, \sin2\thetal\, \cos\phihel$  \\
    \hline

    \ImAmp{\parallel}{{\text{S}_a}}  &
      0  &
      0  \\
    \hline

    \ReAmp{\perp}{{\text{S}_a}}  &
      0  &
      0  \\
    \hline

    \ImAmp{\perp}{{\text{S}_a}}  &
      $-12\sqrt{2}\tfrac{1}{\sqrt{5}}\, P_1^1\, Y_{2,\,-1}$  &
      $-\tfrac{2}{3}\sqrt{6}\, \sin\thetak\, \sin2\thetal\, \sin\phihel$  \\
    \hline
  \end{tabular}

  \caption{Angular functions in helicity angles for a decay in which $a$ is spinless and $b$ is a
    vector particle. The difference of the outgoing particle helicities is zero on both sides
    ($\alpha=\beta=0$). Interference terms with the \PVV{} decay are shown. (See
    Table~\ref{tab:angDistP2SV0Tr} on page~\pageref{tab:angDistP2SV0Tr} for these functions in
    transversity angles.)}
  \label{tab:angDistP2SV0Hel}
\end{table}
%
\begin{table}[htbp]
  \centering \footnotesize
  \begin{tabular}{| c | c | c |}
    \hline
    amplitudes                              &
      $f(\helSet) \times 16\sqrt{\pi}$      &
      $f(\helSet) \times \tfrac{32\pi}{9}$  \\

    \hline\hline

    \AmpSq{{\text{S}_a}}  &
      $4\, P_0^0\,
        (Y_{0,\,0} - \tfrac{1}{\sqrt{5}}\, Y_{2,\,0})$  &
      $\tfrac{2}{3}\, \sin^2\thetal$  \\
    \hline

    \ReAmp[H]{0}{{\text{S}_a}}  &
      $8\sqrt{3}\, P_1^0\, (Y_{0,\,0} - \tfrac{1}{\sqrt{5}}\, Y_{2,\,0})$  &
      $\tfrac{4}{3}\sqrt{3}\, \cos\thetak\, \sin^2\thetal$  \\
    \hline

    \ImAmp[H]{0}{{\text{S}_a}}  &
      0  &
      0  \\
    \hline

    \ReAmp[H]{+}{{\text{S}_a}}  &
      $+6\, P_1^1\, (\tfrac{1}{\sqrt{5}}\, Y_{2,\,+1} \pm Y_{1,\,+1})$  &
      $\pm \tfrac{2}{3}\sqrt{3}\, \sin\thetak\, \sin\thetal\, (1 \pm \cos\thetal)\, \cos\phihel$ \\
    \hline

    \ImAmp[H]{+}{{\text{S}_a}}  &
      $+6\, P_1^1\, (\tfrac{1}{\sqrt{5}}\, Y_{2,\,-1} \pm Y_{1,\,-1})$  &
      $\pm \tfrac{2}{3}\sqrt{3}\, \sin\thetak\, \sin\thetal\, (1 \pm \cos\thetal)\, \sin\phihel$ \\
    \hline

    \ReAmp[H]{-}{{\text{S}_a}}  &
      $+6\, P_1^1\, (\tfrac{1}{\sqrt{5}}\, Y_{2,\,+1} \mp Y_{1,\,+1})$  &
      $\mp \tfrac{2}{3}\sqrt{3}\, \sin\thetak\, \sin\thetal\, (1 \mp \cos\thetal)\, \cos\phihel$ \\
    \hline

    \ImAmp[H]{-}{{\text{S}_a}}  &
      $-6\, P_1^1\, (\tfrac{1}{\sqrt{5}}\, Y_{2,\,-1} \mp Y_{1,\,-1})$  &
      $\pm \tfrac{2}{3}\sqrt{3}\, \sin\thetak\, \sin\thetal\, (1 \mp \cos\thetal)\, \sin\phihel$ \\
    \hline\hline

    \AmpSq{{\text{S}_a}}  &
      $4\, P_0^0\, (Y_{0,\,0} - \tfrac{1}{\sqrt{5}}\, Y_{2,\,0})$  &
      $\tfrac{2}{3}\, \sin^2\thetal$  \\
    \hline

    \ReAmp{0}{{\text{S}_a}}  &
      $8\sqrt{3}\, P_1^0\, (Y_{0,\,0} - \tfrac{1}{\sqrt{5}}\, Y_{2,\,0})$  &
      $\tfrac{4}{3}\sqrt{3}\, \cos\thetak\, \sin^2\thetal$  \\
    \hline

    \ImAmp{0}{{\text{S}_a}}  &
      0  &
      0  \\
    \hline

    \ReAmp{\parallel}{{\text{S}_a}}  &
      $+6\sqrt{2}\tfrac{1}{\sqrt{5}}\, P_1^1\, Y_{2,\,+1}$  &
      $+\tfrac{1}{3}\sqrt{6}\, \sin\thetak\, \sin2\thetal\, \cos\phihel$  \\
    \hline

    \ImAmp{\parallel}{{\text{S}_a}}  &
      $\pm 6\sqrt{2}\, P_1^1\, Y_{1,\,-1}$  &
      $\pm \tfrac{2}{3}\sqrt{6}\, \sin\thetak\, \sin\thetal\, \sin\phihel$  \\
    \hline

    \ReAmp{\perp}{{\text{S}_a}}  &
      $\pm 6\sqrt{2}\, P_1^1\, Y_{1,\,+1}$  &
      $\pm \tfrac{2}{3}\sqrt{6}\, \sin\thetak\, \sin\thetal\, \cos\phihel$  \\
    \hline

    \ImAmp{\perp}{{\text{S}_a}}  &
      $+6\sqrt{2}\tfrac{1}{\sqrt{5}}\, P_1^1\, Y_{2,\,-1}$  &
      $+\tfrac{1}{3}\sqrt{6}\, \sin\thetak\, \sin2\thetal\, \sin\phihel$  \\
    \hline
  \end{tabular}

  \caption{Angular functions in helicity angles for a decay in which $a$ is spinless
    and $b$ a vector particle. The difference of the helicities of particles 1 and 2 is zero and
    the difference of the helicities of particles 3 and 4 is plus or minus one ($\alpha=0$ and
    $\beta=\pm1$).  Interference terms with the \PVV{} decay are shown. (See
    Table~\ref{tab:angDistP2SV1Tr} on page~\pageref{tab:angDistP2SV1Tr} for these functions in
    transversity angles.)}
  \label{tab:angDistP2SV1Hel}
\end{table}

\begin{table}[htbp]
  \centering \footnotesize
  \begin{tabular}{| c | c | c |}
    \hline
    amplitudes                              &
      $f(\helSet) \times 16\sqrt{\pi}$      &
      $f(\helSet) \times \tfrac{32\pi}{9}$  \\

    \hline\hline

    \AmpSq{{\text{S}_b}}  &
      $4\, (P_0^0 + 2\, P_2^0)\, Y_{0,\,0}$  &
      $\tfrac{4}{3}\, \cos^2\thetak$  \\
    \hline

    \ReAmp[H]{0}{{\text{S}_b}}  &
      $-8\, (P_0^0 + 2\, P_2^0)\, Y_{1,\,0}$  &
      $-\tfrac{8}{3}\sqrt{3}\, \cos^2\thetak\, \cos\thetal$  \\
    \hline

    \ImAmp[H]{0}{{\text{S}_b}}  &
      0  &
      0  \\
    \hline

    \ReAmp[H]{+}{{\text{S}_b}}  &
      $+4\, P_2^1\, Y_{1,\,+1}$  &
      $+\tfrac{2}{3}\sqrt{3}\, \sin2\thetak\, \sin\thetal\, \cos\phihel$  \\
    \hline

    \ImAmp[H]{+}{{\text{S}_b}}  &
      $+4\, P_2^1\, Y_{1,\,-1}$  &
      $+\tfrac{2}{3}\sqrt{3}\, \sin2\thetak\, \sin\thetal\, \sin\phihel$  \\
    \hline

    \ReAmp[H]{-}{{\text{S}_b}}  &
      $+4\, P_2^1\, Y_{1,\,+1}$  &
      $+\tfrac{2}{3}\sqrt{3}\, \sin2\thetak\, \sin\thetal\, \cos\phihel$  \\
    \hline

    \ImAmp[H]{-}{{\text{S}_b}}  &
      $-4\, P_2^1\, Y_{1,\,-1}$  &
      $-\tfrac{2}{3}\sqrt{3}\, \sin2\thetak\, \sin\thetal\, \sin\phihel$  \\
    \hline\hline

    \AmpSq{{\text{S}_b}}  &
      $4\, (P_0^0 + 2\, P_2^0)\, Y_{0,\,0}$  &
      $\tfrac{4}{3}\, \cos^2\thetak$  \\
    \hline

    \ReAmp{0}{{\text{S}_b}}  &
      $-8\, (P_0^0 + 2\, P_2^0)\, Y_{1,\,0}$  &
      $-\tfrac{8}{3}\sqrt{3}\, \cos^2\thetak\, \cos\thetal$  \\
    \hline

    \ImAmp{0}{{\text{S}_b}}  &
      0  &
      0  \\
    \hline

    \ReAmp{\parallel}{{\text{S}_b}}  &
      $+4\sqrt{2}\, P_2^1\, Y_{1,\,+1}$  &
      $+\tfrac{2}{3}\sqrt{6}\, \sin2\thetak\, \sin\thetal\, \cos\phihel$  \\
    \hline

    \ImAmp{\parallel}{{\text{S}_b}}  &
      0  &
      0  \\
    \hline

    \ReAmp{\perp}{{\text{S}_b}}  &
      0  &
      0  \\
    \hline

    \ImAmp{\perp}{{\text{S}_b}}  &
      $+4\sqrt{2}\, P_2^1\, Y_{1,\,-1}$  &
      $+\tfrac{2}{3}\sqrt{6}\, \sin2\thetak\, \sin\thetal\, \sin\phihel$  \\
    \hline\hline

    \ReAmp{\text{S}_a}{{\text{S}_b}}  &
      $-8\sqrt{3}\, P_1^0\, Y_{1,\,0}$  &
      $-\tfrac{8}{3}\, \cos\thetak\, \cos\thetal$  \\
    \hline

    \ImAmp{\text{S}_a}{{\text{S}_b}}  &
      0  &
      0  \\
    \hline
  \end{tabular}

  \caption{Angular functions in helicity angles for a decay in which $a$ is a vector particle
    and $b$ is spinless. The difference of the outgoing particle helicities is zero on both sides
    ($\alpha=\beta=0$). Interference terms with both the \PVV{} and ``spinless $a$'' decays are
    shown. (See Table~\ref{tab:angDistP2VSTr} on page~\pageref{tab:angDistP2VSTr} for these
    functions in transversity angles.)}
  \label{tab:angDistP2VSHel}
\end{table}


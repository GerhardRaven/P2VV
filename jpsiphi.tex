\documentclass[a4paper,10pt,twosided]{article}
\usepackage[cm]{fullpage}
\usepackage{rotating}
\usepackage{amsmath}

\setlength{\oddsidemargin}{0in}
\setlength{\evensidemargin}{0in}


\def\thetatr{\theta_\mathrm{tr}}
\def\phitr{\phi_\mathrm{tr}}
\def\psitr{\psi_\mathrm{tr}}
\def\thetaK{\theta_\mathrm{K}}
\def\thetaL{\theta_\mathrm{L}}
\def\phiL{\phi_\mathrm{L}}

\begin{document}


\begin{eqnarray}
   \lambda_f &\equiv& \frac{q}{p}\frac{\overline{\cal A}_f}{{\cal A}_f} = \eta_f\frac{q}{p} \frac{\overline{\cal A}_{\overline{f}}}{{\cal A}_f} \equiv \eta_f \lambda\\
   S &=& -\frac{2 \Im \lambda}{1+|\lambda|^2};\;\; D = -\frac{2 \Re \lambda}{1+|\lambda|^2};\;\; C = \frac{1-|\lambda|^2 }{1+|\lambda|^2}\\
   D^2+S^2 & = & 1 - C^2 \\
   \lambda & = & - \frac{D+iS}{1+C};\;\; \frac{1}{\lambda}  =  -\frac{D-iS}{1-C};\;\; |\lambda|^2 = \frac{ 1-C }{ 1+C }
\end{eqnarray}
\begin{equation}
  \eta_f = (-1)^L = \left\{ \begin{array}{l} +1 \;\; \mathrm{if}\;\; i=0,\parallel  \\ -1\;\; \mathrm{if}\;\; i=\perp,S \end{array} \right. \;\;\;;\;\; \frac{1}{\eta_f}=\eta_f
  \label{eq:etadef}
\end{equation}
For $B_s\rightarrow J/\psi\phi$, assuming the Standard Model,  we have $S=-\sin(2\beta_s),D=-\cos(2\beta_s),C=0$.
In terms of $\phi_s$, we have $S= \sin(\phi_s),D=-\cos(\phi_s),C=0$.

\begin{eqnarray}
   {\cal A}_f(t)            &=& e^{-t/2\tau} \left[ E_+(t) + \eta_f \lambda E_-(t)  \right] a_f  \\
                            &=&     e^{-t/2\tau} \left[ E_+(t) -\eta_f \frac{D+iS}{1+C} E_-(t)  \right] a_f  \\
   {\overline{\cal A}}_f(t) &=& e^{-t/2\tau} \left[  E_+(t) + \frac{1}{\eta_f\lambda} E_-(t)  \right] \eta_f a_f    \\
                            &=&     e^{-t/2\tau} \left[\eta_f E_+(t) - \frac{D-iS}{1-C} E_-(t)  \right] a_f   
\end{eqnarray}
where we have defined that $a_f \equiv {\cal A}_f(t=0)$ and thus that, by definition, $\overline{a}_f = \eta_f a_f$. The above can be
written in a more symmetrical way as:
\begin{eqnarray}
   {\cal A}_f(t)           &=&     \frac{e^{-t/2\tau}}{1+C} \left[ (1+C)E_+(t) -\eta_f (D+iS) E_-(t)  \right] a_f  
%\\                           &=&     \frac{e^{-t/2\tau}}{1-C^2} \left[ (1-C^2)E_+(t)-\eta_f(D+iS)(1-C)E_-(t) \right] a_f\\
\\   {\overline{\cal A}}_f(t) &=&    \frac{e^{-t/2\tau}}{1-C} \left[ (1-C)E_+(t) -\eta_f (D-iS) E_-(t)  \right] \eta_f a_f  
%\\                           &=&     \frac{e^{-t/2\tau}}{1-C^2} \left[ (1-C^2)E_+(t)-\eta_f(D+iS)(1+C)E_-(t) \right] a_f
\end{eqnarray}
where we have used the properties of $\eta_f$ shown in (\ref{eq:etadef}).
%Since the overall normalzation factor $\frac{1}{1-C^2}$ does not depend on any observable, it can be dropped at this point
%(which will simplify the case when we later consider $S=D=0,C=\pm 1$!), resulting in:
%\begin{eqnarray}
%   {\cal A}_f(t)            &=&     e^{-t/2\tau} \left[ (1-C^2)E_+(t)-\eta_f(D+iS)(1-C)E_-(t) \right] a_f\\
%   {\overline{\cal A}}_f(t) &=&     e^{-t/2\tau} \left[ (1-C^2)E_+(t)-\eta_f(D+iS)(1+C)E_-(t) \right] a_f
%\end{eqnarray}
As a consequence, the expressions for $\overline{b}_\mathrm{tag}$ are related to those for $b_\mathrm{tag}$ by the following substitution:
\begin{equation}
b_\mathrm{tag} \leftrightarrow \overline{b}_\mathrm{tag} \;\;\;\Leftrightarrow\;\;\;  ( S \leftrightarrow -S, C\leftrightarrow -C, a_\perp\leftrightarrow -a_\perp ) 
\end{equation}
which can be combined in a single expression by multiplying the relevant terms with the value of flavour tag:
\begin{equation}
(S,C,a_\perp) \rightarrow (q_T S, q_T C, q_Ta_\perp) \;\; \mathrm{with}\;\; q_T = \left\{\begin{array}{l} +1 \;\;\; \mathrm{if}\; b_\mathrm{tag} \\
                                                                                   -1 \;\;\; \mathrm{if}\; \overline{b}_\mathrm{tag}
\end{array} \right.
\end{equation}

%Note that in case of $B_0\rightarrow J/\psi K^{*0}(K^+\pi^-)$, we have the
%case where $\lambda=0$, or $S=D=0,C=1$. On the other hand, $B_0\rightarrow J/\psi \overline{K^{*0}}(K^-\pi^+)$, 
%we have the case where $1/\lambda=0$, or $S=D=0,C=-1$.

%\begin{equation}
%   A(q_T=-1,q_\perp=\pm,t) =\left( \frac{1-q_T}{2}q_\perp+\frac{1+q_T}{2}\right) e^{-t/\tau} \left[ E_+(t) +q_\perp e^{2q_Ti\beta_s}E_-(t)  \right] a_f 
%\end{equation}
%\begin{eqnarray}
%   {\cal A}_f(t)            &= \frac{1}{1+C} e^{-t/2\tau} \left[+ (1+C) E_+(t) \pm (D+iS) E_-(t)  \right] a_f  &= A(q_T=+1,q_\perp,t)\\
%   {\overline{\cal A}}_f(t) &= \frac{1}{D+iS} e^{-t/2\tau} \left[\pm  (D+iS) E_+(t) + (1+C) E_-(t)  \right] a_f  &= A(q_T=-1,q_\perp,t) 
%\end{eqnarray}

Note that we have not assumed that $|\lambda|=1$, but we did assume that all three $|\lambda_f|$ are the same, i.e.
this implies that there either there is either no, or identical amount of direct CP violation in all three decay amplitudes,
i.e. $|\overline{A}_f/A_f|$ are all the same. The above does allow for CP violation in mixing, as in that case $|q/p|\ne 1$,
and all three $|\lambda_f|$ would be, by definition, affected identically. As a result, {\em assuming the absence of direct CP},
we can relate the value  of $C$ to $A_\mathrm{SL}$ as follows:
\begin{eqnarray}
 A_\mathrm{SL} &=& \frac{ |q/p|^2 - |p/q|^2 }{ |q/p|^2 + |p/q|^2 }  \\
               &=& \frac{ - 2 C } { 1 + C^2 } \approx - 2 C
\end{eqnarray}

\section{Time Evolution}

\begin{equation}
  E_\pm = \frac{1}{2} \left[ e^{+i\omega t } \pm e^{-i\omega t } \right]\;\;\;\;\mathrm{with}\;
   \omega = \frac{\Delta m}{2} + i\frac{\Delta \Gamma}{4} 
\end{equation}
\begin{eqnarray}
 | E_\pm(t) |^2  &=& \frac{1}{2}\left[ \cosh\left(\frac{\Delta\Gamma}{2}t\right) \pm \cos\left(\Delta m t \right) \right] \\
 E_+^*(t)E_-(t)  &=& \frac{1}{2}\left[ -\sinh\left(\frac{\Delta\Gamma}{2}t\right) + i \sin\left(\Delta m t \right) \right]
\end{eqnarray}

%\begin{eqnarray}
%   {\cal N}_\pm  &= & \int_0^\infty dt\; e^{-t/\tau}| E_\pm(t) |^2   
%                  =  \frac{\tau}{2}\left[ \frac{1}{1-(\Delta\Gamma \tau)^2}  \pm \frac{1}{1+\left(\Delta m \tau\right)^2}\right]
%\\ {\cal N}_{+-} &= & \int_0^\infty dt\; e^{-t/\tau} E_+^*(t)E_-(t)  
%                  =  \frac{\tau}{2}\left[ -\frac{\Delta\Gamma\tau}{1-(\Delta\Gamma\tau)^2} + i \frac{\Delta m \tau}{1+\left(\Delta m \tau \right)^2}  \right]
%\end{eqnarray}

\begin{eqnarray}
   |{\cal A}_i(t)|^2  &=& {\cal A}_i^*(t){\cal A}_i(t) \;\; (\mathrm{note: no\; summation!}) \\
%   |{\cal A}_i(t)|^2  &=& e^{-t/\tau}|a_i|^2 \left| E_+(t) +\eta_{i}\frac{D+iS}{1+C} E_-(t)  \right|^2  \\
%                      &=& \frac{e^{-t/\tau}|a_i|^2}{1+C} \left[ (1+C)|E_+(t)|^2 + (1-C)| E_-(t)|^2 + 2 \eta_{i}\left( D\Re\left(E_+^*(t)E_-(t)\right)-S\Im\left(E_+^*(t)E_-(t)\right)\right)  \right] \\
%                      &=& \frac{e^{-t/\tau}|a_i|^2}{1+C} \left[ \cosh\left(\frac{\Delta\Gamma}{2}t\right)
%                                                              + C\cos(\Delta m t ) 
%                                                              - \eta_{i} D\sinh\left(\frac{\Delta\Gamma}{2}t\right)
%                                                              - \eta_{i} S\sin \left(\Delta m t \right)  
%                                                         \right] \\
%   |{\cal\overline{A}}_i(t)|^2 % &=& e^{-t/\tau}|a_i|^2 \left| \eta_i E_+(t) +\frac{D-iS}{1-C} E_-(t)  \right|^2  \\
%                      &=& \frac{e^{-t/\tau}|a_i|^2}{1-C} \left[ \cosh\left(\frac{\Delta\Gamma}{2}t\right)
%                                                              - C\cos(\Delta m t ) 
%                                                              - \eta_{i} D\sinh\left(\frac{\Delta\Gamma}{2}t\right)
%                                                              + \eta_{i} S\sin \left(\Delta m t \right)  
%                                                         \right] \\
  {\cal A}_i^*(t) {\cal A}_j(t) &= \frac{ a_i^* a_j e^{-t/\tau}} {1+C}&\Biggl[ (1+C) |E_+(t)|^2  +\eta_i\eta_j (1-C)|E_-(t)|^2 \nonumber\\
                                                                          && - \eta_j(D+iS)E_+^*E_--\eta_i(D-iS)E_+E_-^*   \Biggr] \\
                                &= \frac{ a_i^* a_j e^{-t/\tau}} {1+C}&\Biggl[ \left( \frac{1+\eta_i\eta_j}{2} +  \frac{1-\eta_i\eta_j}{2}C\right) \cosh\left(\frac{\Delta\Gamma}{2}t\right)\nonumber \\
                                                                          &&  +\left( \frac{1-\eta_i\eta_j}{2} +  \frac{1+\eta_i\eta_j}{2}C\right) \cos(\Delta m t) \nonumber\\
                                                                          &&  +\left( \frac{\eta_i+\eta_j}{2}D - \frac{\eta_i-\eta_j}{2}iS\right) \sinh\left(\frac{\Delta\Gamma}{2}t\right) \nonumber\\
                                                                          &&  +\left( \frac{\eta_i+\eta_j}{2}S + \frac{\eta_i-\eta_j}{2}iD\right) \sin(\Delta m t) \Biggr] 
\end{eqnarray}

The result can be summarized as follows (where we have used that $q_T^2=1$):
\begin{equation}
\begin{array}{| c |c|| c | c | c | c | c |}
 \hline
                        q_T=+1                      &    q_T = -1&                          &  \cosh\left(\frac{\Delta\Gamma}{2}t\right) & q_T \cos(\Delta m t)  & \sinh\left(\frac{\Delta\Gamma}{2}t\right) &q_T\sin(\Delta m t)  \\
 \hline
|{\cal A}_0(t)|^2                             &                                                                &\frac{  | a_0 |^2 e^{-t/\tau}}{1+q_TC}                  & 1 &   C  &   D &  S   \\
|{\cal A}_\parallel(t)|^2                     &                                                                &\frac{  | a_\parallel |^2 e^{-t/\tau}}{1+q_TC}          & 1 &   C  &   D &  S   \\
|{\cal A}_\perp(t)|^2                         &                                                                &\frac{  | a_\perp |^2 e^{-t/\tau}}{1+q_TC}              & 1 &   C  &  -D & -S   \\
\Im({\cal A}_\parallel^*(t){\cal A}_\perp(t)) &                                                                &\frac{  \Re(a_\parallel^* a_\perp)e^{-t/\tau}}{1+q_TC}  & 0 &   0  &  -S &  D   \\                         
                                              &                                                                &\frac{  \Im(a_\parallel^* a_\perp)e^{-t/\tau}}{1+q_TC}  & C &   1  &   0 &  0   \\
\Re({\cal A}_0^*(t){\cal A}_\parallel(t))     &                                                                &\frac{  \Re(a_0^*a_\parallel)e^{-t/\tau}}{1+q_TC}       & 1 &   C  &   D &  S   \\
                                              &                                                                &\frac{  \Im(a_0^*a_\parallel)e^{-t/\tau}}{1+q_TC}       & 0 &   0  &   0 &  0   \\
\Im({\cal A}_0^*(t){\cal A}_\perp(t))         &                                                                &\frac{  \Re(a_0^* a_\perp)e^{-t/\tau}}{1+q_TC}          & 0 &   0  &  -S &  D   \\
                                              &                                                                &\frac{  \Im(a_0^* a_\perp)e^{-t/\tau}}{1+q_TC}          & C &   1  &   0 &  0   \\
|{\cal A}_S(t)|^2                             &                                                                &\frac{  | a_S |^2 e^{-t/\tau}}{1+q_TC}                  & 1 &   C  &  -D & -S   \\
\Re({\cal A}_S^*(t){\cal A}_\parallel(t))     &                                                                &\frac{  \Re(a_S^*a_\parallel)e^{-t/\tau}}{1+q_TC}       & C &   1  &   0 &  0   \\
                                              &                                                                &\frac{  \Im(a_S^*a_\parallel)e^{-t/\tau}}{1+q_TC}       & 0 &   0  &  -S &  D   \\
\Im({\cal A}_S^*(t){\cal A}_\perp(t))         &                                                                &\frac{  \Re(a_S^* a_\perp)e^{-t/\tau}}{1+q_TC}          & 0 &   0  &   0 &  0   \\                         
                                              &                                                                &\frac{  \Im(a_S^* a_\perp)e^{-t/\tau}}{1+q_TC}          & 1 &   C  &  -D & -S   \\
\Re({\cal A}_S^*(t){\cal A}_0(t))             &                                                                &\frac{  \Re(a_S^* a_0)e^{-t/\tau}}{1+q_TC}              & C &   1  &   0 &  0   \\
                                              &                                                                &\frac{  \Im(a_S^* a_0)e^{-t/\tau}}{1+q_TC}              & 0 &   0  &  -S &  D   \\
\hline
\end{array}
\end{equation}

\pagebreak
\subsection{Decay Time Resolution}

By using the observation that $\sin(\Delta m t) = \Im\left( e^{i\Delta m t}\right)$ and $\cos(\Delta m t ) = \Re\left( e^{ i \Delta m t } \right)$, and that both $\cosh$ and $\sinh$ can
be written as the sum and difference of exponentials, i.e. it is just the $\cos$ term with $\Delta m = 0$, it is sufficient to consider the following convolution:
\begin{eqnarray*}
f(t;\Gamma,\Delta m,\sigma) &=& \int_0^{+\infty}\;dt'\;\; e^{-(\Gamma -i \Delta m)t'} \frac{ e^{-\frac{1}{2}\left( \frac{t-t'}{\sigma} \right)^2}}{\sqrt{2\pi}\sigma}
\\                          &=& \frac{1}{\sqrt{2\pi}} \int_0^{+\infty}\;dy \;\;  e^{-\frac{1}{2}\left( x-y \right)^2-zy}
\end{eqnarray*}
where we have defined $z = (\Gamma - i \Delta m ) \sigma$ and $x = t / \sigma$.
Completing the square in the exponent and absorbing the shift in the boundaries of the integral, we find:
\begin{eqnarray*}
f(t;\Gamma,\Delta m,\sigma) &=& e^{-\frac{1}{2}x^2+\frac{1}{2}(z-x)^2}  \frac{1}{\sqrt{2\pi}} \int_0^{+\infty}\;dy \;\; e^{-\left(\frac{ y+(z-x)}{\sqrt{2}} \right)^2}
\\                          &=&  e^{-\frac{1}{2}x^2+\frac{1}{2}(z-x)^2}  \frac{1}{\sqrt{\pi}} \int_{\frac{z-x}{\sqrt{2}}}^{+\infty}\;dy \;\; e^{-y^2 }
\\                          &=&  e^{-\frac{1}{2}x^2+\frac{1}{2}(z-x)^2}  \; \frac{1}{2}\mathrm{erfc}\left(\frac{z-x}{\sqrt{2}} \right)
\end{eqnarray*}
Next we use the Faddeeva function $w$,  defined as 
\begin{equation}
    w(z) = e^{-z^2}\mathrm{erfc}\left(-iz\right)
\end{equation}
and implemented in CERNLIB (see http://wwwasdoc.web.cern.ch/wwwasdoc/shortwrupsdir/c335/top.html) as $\mathrm{cwerf}$, to write the result as:
\begin{eqnarray*}
f(t;\Gamma,\Delta m,\sigma) &=&  e^{-\frac{1}{2}x^2} w\left( \frac{i}{\sqrt{2}} \left( z-x\right) \right)\;\; \mathrm{where} \; x=\frac{t}{\sigma};\; z = \left(\Gamma-i\Delta m\right)\sigma
\\   &=& e^{-\frac{1}{2}\left(\frac{t}{\sigma}\right)^2} w\left( \frac{i}{\sqrt{2}} \left( -\frac{t}{\sigma} + \sigma\left(\Gamma-i\Delta m \right) \right)  \right)
\end{eqnarray*}
The corresponding normaliztion integral is given by:
\begin{equation*}
    I_0(t;\Gamma,\Delta m,\sigma)  = \int dt f(t;\Gamma,\Delta m, \sigma) = \sigma \int dx f(x;z) = \sigma I_0(x;z)
\end{equation*}
where the latter $I_0(x;z)$ is defined as:
\begin{eqnarray*}
   I_0(x_\mathrm{min},x_\mathrm{max};z) &\equiv& \int_{x_\mathrm{min}}^{x_\mathrm{max}} dx\; e^{-\frac{1}{2}x^2} w\left( \frac{i}{\sqrt{2}}(z-x) \right) 
\\             &=&\frac{1}{z}\left[ \mathrm{erf}\left(\frac{x}{\sqrt{2}} \right) -e^{-\frac{1}{2}x^2} w\left( \frac{i}{\sqrt{2}}\left(z-x\right) \right) \right]_{x_\mathrm{min}}^{x_\mathrm{max}}
\end{eqnarray*}

\subsubsection{Describing decaytime efficiency with cubic b-splines}
To describe a non-trivial decaytime efficiency by taking the product with cubic b-splines, 
integrals of form $(t-\alpha)(t-\beta)(t-\gamma) f(t)$ are required for normalization.
These integrals can be written in terms of integrals of $t^nf(t)$ for $n=0,1,2,3$, which can be computed 
using the following method\footnote{Manuel Schiller, private communication}:
\begin{eqnarray*}
    I_n(x_\mathrm{min},x_\mathrm{max};z)&\equiv& \int_{x_\mathrm{min}}^{x_\mathrm{max}} dx\; x^n e^{-\frac{1}{2}x^2} w\left( \frac{i}{\sqrt{2}}(z-x)\right) 
       \\      &  =& \left.\frac{d^n}{d\lambda^n}\right|_{\lambda=0 } \int_{x_\mathrm{min}}^{x_\mathrm{max}} dx\; e^{\lambda x} e^{-\frac{1}{2}x^2} w\left( \frac{i}{\sqrt{2}}(z-x)\right) 
       \\      & \equiv& \left.\frac{d^n}{d\lambda^n}\right|_{\lambda=0 } I(x_\mathrm{min},x_\mathrm{max};z,\lambda)
\end{eqnarray*}
Once again we complete the square, and shift the integrand to obtain:
\begin{eqnarray*}
       I(x_\mathrm{min},x_\mathrm{max};z,\lambda) &=&  \int_{x_\mathrm{min}}^{x_\mathrm{max}} dx\; e^{\lambda x} e^{-\frac{1}{2}x^2} w\left( \frac{i}{\sqrt{2}}(z-x)\right) 
        \\            &=&  e^{\frac{1}{2}\lambda^2}\int_{x_\mathrm{min}-\lambda}^{x_\mathrm{max}-\lambda} dx \;  e^{-\frac{1}{2}x^2} w\left(\frac{i}{\sqrt{2}}\left(z-\lambda-x\right) \right)
        \\            &=& \frac{e^{\frac{1}{2}\lambda^2}}{z-\lambda} \left[ \mathrm{erf}\left( \frac{x}{\sqrt{2}} \right) - e^{-\frac{1}{2}x^2}w\left(\frac{i}{\sqrt{2}}\left(z-\lambda-x \right) \right) \right]_{x_\mathrm{min}-\lambda}^{x_\mathrm{max}-\lambda}
        \\            &\equiv& K(z,\lambda) \left[  J(x_\mathrm{max};\lambda,z) - J(x_\mathrm{min};\lambda,z)  \right]
\end{eqnarray*}
where $J(x;\lambda,z)=\mathrm{erf}\left(\frac{x-\lambda}{\sqrt{2}}\right)-e^{-\frac{1}{2}\left(x-\lambda\right)^2}w\left(\frac{i}{\sqrt{2}}\left(z-x\right) \right)$ and $K(\lambda,z)=\frac{e^{\frac{\lambda^2}{2}}}{z-\lambda}$.

In order to simplify the  computation of  $I_n$, we define the $n^\mathrm{th}$ order derivatives at $\lambda=0$, $K_n(z)$ and $M_n(x;z)$, as follows:
\begin{equation}
\begin{array}{|c|c|c|}
\hline
n   
& \begin{array}{l}  K_n(z)   = \\  \left. \frac{d^n}{d\lambda^n}\right|_{\lambda=0} \frac{e^{\frac{1}{2}\lambda^2}}{z-\lambda}  \end{array}
& \begin{array}{l}  M_n(x;z) = \\  \left. \frac{d^n}{d\lambda^n}\right|_{\lambda=0} \left[ \mathrm{erf}\left(\frac{x-\lambda}{\sqrt{2}}\right)-  e^{-\frac{1}{2}\left(x-\lambda\right)^2}w\left(\frac{i}{\sqrt{2}}\left(z-x\right)\right)  \right] \end{array}
\\ \hline 
   0   & z^{-1} &  \mathrm{erf}\left(\frac{x}{\sqrt{2}}\right)-e^{-\frac{1}{2}x^2}w\left( \frac{i}{\sqrt{2}} \left(z-x\right)\right) 
\\ 1   & z^{-2}          & e^{-\frac{1}{2}x^2} \left[ - \sqrt{\frac{2}{\pi}}   -       x w\left(\frac{i}{\sqrt{2}}(z-x)  \right) \right]
\\ 2   & 2z^{-3}+z^{-1}  & e^{-\frac{1}{2}x^2} \left[ - \sqrt{\frac{2}{\pi}} x - (x^2-1) w\left(\frac{i}{\sqrt{2}}(z-x)\right) \right]
\\ 3   & 6z^{-4}+3z^{-2} & e^{-\frac{1}{2}x^2} \left[ - \sqrt{\frac{2}{\pi}} (x^2-1)-x(x^2-3)w\left(\frac{i}{\sqrt{2}}(z-x)\right)   \right]
\\ \hline
\end{array}
\end{equation}
In terms of these functions, the normalization integrals can now be written as:
\begin{eqnarray*}
     I_0(x_\mathrm{min},x_\mathrm{max};z) &=&  \;\; K_0(z) \left[ M_0(x;z) \right]_{x_\mathrm{min}}^{x_\mathrm{max}}
\\   I_1(x_\mathrm{min},x_\mathrm{max};z) &=&  
       \left(\begin{array}{c} M_0(x;z) \\ M_1(x;z)   \end{array}\right)
\cdot \left(\begin{array}{cc}
                   0  & 1 
               \\  1  & 0 
               \end{array}\right) 
       \left(\begin{array}{c} K_0(z) \\ K_1(z)  \end{array} \right) 
\\   I_2(x_\mathrm{min},x_\mathrm{max};z) &=&  
       \left(\begin{array}{c} M_0(x;z) \\ M_1(x;z) \\ M_2(x;z)  \end{array}\right)
\cdot \left(\begin{array}{ccc}
                   0  & 0 & 1  
               \\  0  & 2 & 0  
               \\  1  & 0 & 0  
               \end{array}\right) 
       \left(\begin{array}{c} K_0(z) \\ K_1(z) \\ K_2(z)  \end{array} \right) 
\\   I_3(x_\mathrm{min},x_\mathrm{max};z) &=&  
       \left(\begin{array}{c} M_0(x;z) \\ M_1(x;z) \\ M_2(x;z) \\ M_3(x;z) \end{array}\right)
\cdot \left(\begin{array}{cccc}
                   0  & 0 & 0  &  1
               \\  0  & 0 & 3  &  0
               \\  0  & 3 & 0  &  0
               \\  1  & 0 & 0  &  0
               \end{array}\right) 
       \left(\begin{array}{c} K_0(z) \\ K_1(z) \\ K_2(z) \\ K_3(z) \end{array} \right) 
\end{eqnarray*}

At this point, we return to the original problem of the normalization of cubic b-spines multiplied by $f(t)$.
In general, the cubic b-splines have the form $(x-\alpha)(x-\beta)(x-\gamma)$, which implies that the corresponding
normalization can be written as a linear combination of $I_n$:
\begin{eqnarray*}
I(x_i,x_{i+1};z,\alpha,\beta,\gamma) &=&
  \int_{x_i}^{x_{i+1}} dx (x-\alpha)(x-\beta)(x-\gamma)  e^{-\frac{1}{2}x^2} w\left( \frac{i}{\sqrt{2}}(z-x) \right) 
\\ &=& I_3(x_i,x_{i+1};z) 
    -  (\alpha+\beta+\gamma) I_2(x_i,x_{i+1};z)
\\ &+& (\alpha\beta+\alpha\gamma+\beta\gamma) I_1(x_i,x_{i+1};z) 
    -  \alpha\beta\gamma I_0(x_i,x_{i+1};z)
\\ &=& 
       \left(\begin{array}{c} M_0(x_{i+1};z)-M_0(x_i;z) \\ M_1(x_{i+1};z)-M_1(x_i;z) \\ M_2(x_{i+1};z)-M_2(x_i;z) \\ M_3(x_{i+1};z)-M_3(x_i;z) \end{array}\right)^T
      \left(\begin{array}{cccc}
                   -\alpha\beta\gamma                     & \alpha\beta+\alpha\gamma+\beta\gamma & -(\alpha+\beta+\gamma)  &  1
               \\  \alpha\beta+\alpha\gamma+\beta\gamma   &              -2(\alpha+\beta+\gamma) &                      3  &  0
               \\  -(\alpha+\beta+\gamma)                 &                                    3 &                      0  &  0
               \\  1                                      &                                    0 &                      0  &  0
               \end{array}\right) 
       \left(\begin{array}{c} K_0(z) \\ K_1(z) \\ K_2(z) \\ K_3(z) \end{array} \right) 
\\ &\equiv& \sum_{j=0}^3 \sum_{k=0}^{3-j} \left( M_j(x_{i+1};z)-M_j(x_i;z)\right) S_{jk}(\alpha,\beta,\gamma) K_k(z)
\end{eqnarray*}
Note that in the end, one needs to evaluate (assuming cubic splines) a sum over four splines in each interval, so one 
needs to take a linear combination of four matrices $S_{jk}(\alpha,\beta,\gamma)$, whose values only depend on the definition 
of the knotvector.  In addition, the upper bound of the $i^\mathrm{th}$ interval is the lower bound of 
the $(i+1)^\mathrm{th}$ interval, so $M_n$ only needs to be evaluated $n+1$ times, where $n$ is the number of knot intervals.

We can go one step further and write $M_n(x;z)$ in terms of its basis functions (taking
care to cancel divergent pieces):
\begin{equation}
     \left( \begin{array}{c} M_0(x;z) \\ M_1(x;z) \\ M_2(x;z) \\ M_3(x;z) \end{array} \right)
     = \left( \begin{array}{ccc}
          1 &  0 & -1
     \\   0 & -\sqrt{\frac{2}{\pi}}& -x 
     \\   0 & -\sqrt{\frac{2}{\pi}}x& -(x^2-1) 
     \\   0 & -\sqrt{\frac{2}{\pi}}(x^2-1)& -x(x^2-3) 
       \end{array}\right) 
       \left(\begin{array}{c} 
                      \mathrm{erf}\left(\frac{x}{\sqrt{2}}\right)
              \\      e^{-\frac{1}{2}x^2}
              \\      e^{-\frac{1}{2}x^2}w\left(\frac{i}{\sqrt{2}}(z-x) \right)
             \end{array} \right)
\end{equation}

\subsubsection{Spline definition}


Given a knot vector $(u_0,u_1,\dots,u_n)$ of length $n+1$, defining $n$ intervals, there are $n+3$ cubic b-splines $\mathcal{S}_i(u)$.

The complete spline function $ \mathcal{S}(u)$ is given by the linear combination $\mathcal{S}(u)= \sum_{i=0}^{n+2} b_i  \mathcal{S}_i(u)$.
For a value $u$ in the interval $u_{i}<  u\le u_{i+1}$, there are four contributing b-splines,$\mathcal{S}_i(u)$ for $i\in \{i,i+1,i+2,i+3\}$.
To compute their contributions, we extend the knot vector at both ends by three entries given by  
\begin{eqnarray*}
   u_{-3}=u_{-2}=u_{-1}&=&u_0
\\ u_{n+1}=u_{n+2}=u_{n+3}&=&u_n
\end{eqnarray*}
The four non-zero contributions can now be written as:
\begin{equation}
    \mathcal{S}(u) = b_i A_i(u) + b_{i+1} B_i(u) + b_{i+2} C_i(u) + b_{i+3} D_i(u)
\end{equation}
\begin{eqnarray*}
     A_i(u) &=& -\frac{(u-u_{i+1})^3}{P_i}  
\\   B_i(u) &=&  \frac{(u-u_{i-2})(u-u_{i+1})^2 }{P_i} + \frac{ (u-u_{i-1})(u-u_{i+1})(u-u_{i+2}) }{Q_i} + \frac{ (u-u_i)(u-u_{i+2})^2 }{R_i}
\\   C_i(u) &=& -\frac{(u-u_{i-1})^2(u-u_{i+1}) }{Q_i} - \frac{ (u-u_{i-1})(u-u_i )(u-u_{i+2}) }{R_i} - \frac{ (u-u_i)^2(u-u_{i+3}) }{S_i}
\\   D_i(u) &=&  \frac{(u-u_i)^3 }{S_i}
\end{eqnarray*}
where the constants $P_i,Q_i,R_i$ and $S_i$ depend on the knot vector only, and are defined as:
\begin{eqnarray*}
    P_i &=&(u_{i+1}-u_{i-2})(u_{i+1}-u_{i-1})(u_{i+1}-u_i)
\\  Q_i &=&(u_{i+2}-u_{i-1})(u_{i+1}-u_{i-1})(u_{i+1}-u_i)
\\  R_i &=&(u_{i+2}-u_{i  })(u_{i+2}-u_{i-1})(u_{i+1}-u_i)
\\  S_i &=&(u_{i+3}-u_{i  })(u_{i+2}-u_{i  })(u_{i+1}-u_i)
\end{eqnarray*}

This implies that the top row of the matrix $S_{jk}$ (which is all that is required to fully define it) for the
interval $[u_i,u_{i+1})$ is given by
\begin{eqnarray*}
            &\rightarrow& S_{1k} += -b_i/P_i  ( -u_{i+1}^3;\; 3u_{i+1}^2 ;\;-3u_{i+1} )
     \\     &\rightarrow& S_{1k} += b_{i+1}/P_i (-u_{i-2}^2u_{i+1};\; 2u_{i-2}u_{i+1}+u_{i+1}^2;\;  -u_{i-1}-2u_{i+1} )
     \\     &           & S_{1k} += b_{i+1}/Q_i (-u_{i-1}u_{i+1}u_{i+2};\; u_{i-1}u_{i+1}+u_{i+1}u_{i+2}+u_{i+1}u_{i+2} )
     \\     &           & S_{1k} += b_{i+1}/R_i (-u_i u_{i+2}^2;\; 2u_iu_{i+2}+u_{i+2}^2;\;-u_i-2u_{i+2} )
     \\     &\rightarrow& S_{1k} += b_{i+2}/Q_i
     \\     &           & S_{1k} += b_{i+2}/R_i
     \\     &           & S_{1k} += b_{i+2}/S_i
     \\     &\rightarrow& 
\end{eqnarray*}

\begin{eqnarray*}
    A_i &=& \{ \{ +1,+1,+1, -P \} \}
 \\ B_i &=& \{ \{ -2,+1,+1, P \}, \{-1,+1,+2, Q \}, \{ 0,2,2,R\} \}
 \\ C_i &=& \{ \{ -1,-1,+1, -Q \}, \{-1,0,+2, -R\},\{0,0,3,-S \}\}
 \\ D_i &=& \{ \{ 0,0,0, S \}
 \\ P_i &=& \{ \{1,-2\},\{1,-1\},\{1,0\} \}
 \\ Q_i &=& \{ \{2,-1\},\{1,-1\},\{1,0\} \}
 \\ R_i &=& \{ \{2,+1\},\{2,-1\},\{1,0\} \}
 \\ S_i &=& \{ \{3,+0\},\{2,0\},\{1,0\} \}
\end{eqnarray*}



\pagebreak
\subsection{Flavour tagging}

Let us start by introducing the following short-hand notation. One can write the 
PDF for 'true' $B$ and $\overline{B}$ as:
\begin{eqnarray}
   P_{B}           &\equiv&\frac{ {\cal E} + {\cal O}}{N_+} \\
   P_{\overline{B}} &\equiv&\frac{ {\cal E} - {\cal O}}{N_-}
\end{eqnarray}
where ${\cal E}$ and ${\cal O}$ are the (propertime and angular dependent) CP-even resp. CP-odd parts of the PDF.
and their ratio ${\cal A} \equiv {\cal E / O }$.
Taking the sum over these, we find:
\begin{equation}
   P_{B} + P_{\overline{B}} =  \frac{ {\cal E} + \nu{\cal O} }{N}  = \frac{\cal E} {N} \left( 1 + \nu \frac{ \cal O }{\cal E} \right) = \frac{\cal E}{N} \left(1 + \nu {\cal A}\right)
   \label{eq:sumbbbar}
\end{equation}
where we have defined
\begin{eqnarray}
   &  N \equiv \left( \frac{1}{N_+}+\frac{1}{N_-}\right)^{-1} = \frac{N_+N_-}{N_+ + N_-};\;\;\;\; &  \frac{1}{N_+} =\frac{1}{2N}(1+\nu) \\
   &\nu \equiv N \left( \frac{1}{N_+}-\frac{1}{N_-}\right) = \frac{N_- - N_+}{N_+ +N_-}; & \frac{1}{N_-} =\frac{1}{2N}(1-\nu) 
\end{eqnarray}
and requiring the total sum of PDFs (\ref{eq:sumbbbar}) to be properly normalized implies
\begin{equation}
   N = I_{\cal E} + \nu I_{\cal O} 
     = I_{\cal E} \left( 1 + \nu \frac{ I_{\cal O}}{I_{\cal E}}   \right)
\end{equation}
where $I_x$ is the integral of $x$ over the propertime and angles.
Note that taking the difference, we find for the asymmetry:
\begin{equation}
  \frac{P_{B} - P_{\overline{B}}}{P_{B} + P_{\overline{B}}} =  \frac{ {\cal A} + \nu}{1+\nu{\cal A}}
   \label{eq:difbbbar}
\end{equation}


Next we introduce the probability of tagging an (initial,"true") $B$ meson  and
the probability to tag an initial $\overline{B}$ meson in some tagging category $i$ (with $i\ne 0$) as $t_i$
respectively $\overline{t}_i$ -- at this point we still do not distinguish between $B$ and $\overline{B}$, i.e.
we marginalize over the tagdecision.  This yields the following PDF for events in category $i$:
\begin{eqnarray}
     P_i & \equiv & t_i P_B + \overline{t}_i P_{\overline B}  \nonumber
\\       &      = & \frac{\epsilon_i}{N} \left[ (1+\nu\delta_i ){\cal E}+(\delta_i + \nu){\cal O} \right] 
%\\       &      = & \epsilon_i\frac{\cal E}{N}(1+\nu\delta_i ) \left[1 +\frac{\delta_i + \nu}{1+\nu\delta_i }{\cal A} \right] 
%\\       &      = & \epsilon_i\frac{\cal E}{2N}\left[ (1+\delta_i)(1+{\cal A})(1+\nu) + (1-\delta_i)(1-{\cal A})(1-\nu) \right]
%\\       &      = & \epsilon_i\frac{\cal E}{2N}(1+\nu\delta_i + \delta_i { \cal A} + \nu {\cal A} )
\\  P_i  &      = & \epsilon_i \left[ (1+\delta_i ) P_B + (1-\delta_i) P_{\overline{B}} \right]
\\       &      = & \epsilon_i \left[ 1+ \delta_i \frac{ P_B -  P_{\overline{B}}}{ P_B +  P_{\overline{B}}} \right]\left[ P_B + P_{\overline{B}}  \right]
%\\       &      = & \epsilon_i \left[1 +\delta_i \frac{{\cal A}+ \nu}{1+\nu{\cal A} } \right] \left[ P_B + P_{\overline{B}} \right] \label{eq:tagsummed}
\\       &      = & \epsilon_i\left[1 +\delta_i \frac{{\cal A}+ \nu}{1+\nu{\cal A} } \right] \frac{\cal E}{N}( 1 + \nu {\cal A} ) 
\\       &      = & \epsilon_i \frac{ (1+\delta_i\nu )}{N} \left[1+{\cal A}\frac{\nu+\delta_i}{1+\delta_i\nu} \right] {\cal E}
\\       & \approx& \epsilon_i \frac{ 1 }{N} \left[1+{\cal A}(\nu+\delta_i) \right] {\cal E}
\end{eqnarray}
where we have defined
\begin{eqnarray}
   &\epsilon_i = \frac{ t_i + \overline{t}_i }{2};\;\;\;\;  & t_i = \epsilon_i( 1 + \delta_i ) \\
   &\delta_i = \frac{t_i - \overline{t}_i}{t_i + \overline{t}_i}; & \overline{t}_i = \epsilon_i( 1 - \delta_i ) 
\end{eqnarray}
and for the remainder, i.e. the untagged events, which we denote with an index $0$:
\begin{eqnarray}
     P_0 & \equiv & \left(1-\sum_i t_i\right) P_B + \left(1-\sum_i \overline{t}_i\right) P_{\overline B}  \nonumber
\\   & = & \frac{1}{N} \left[ \left( 1-\sum_i\epsilon_i\left(1+\nu\delta_i\right) \right) {\cal E} 
                            + \left( \nu - \sum_i \epsilon_i(\delta_i+\nu)   \right) {\cal O} \right] \label{eq:untagged}
\\   P_0 & \equiv & \left( 1 - \sum_i \frac{P_i }{P_{B} + P_{\overline B}} \right)( P_{B} + P_{\overline B}) 
\\       & = & \left( 1  - \sum_i \epsilon_i\left[1 +\delta_i \frac{{\cal A}+ \nu}{1+\nu{\cal A} } \right]\right) \frac{\cal E}{N}( 1 + \nu {\cal A} ) 
%\\ &=& \frac{\cal E}{N} \left(1+\nu{\cal A} - \sum_i \epsilon_i\left[ (1+\delta_i \nu) \left(1+\frac{\nu+\delta_i}{1+\delta_i\nu} {\cal A}\right)\right] \right) 
\\ &=& \frac{\cal E}{N} \left(1- \sum_i \epsilon_i\left(1+\delta_i \nu \right)+ \left(\nu -\sum_i \epsilon_i (\nu+\delta_i) \right){\cal A} \right) 
\\ &=& \frac{1- \sum_i \epsilon_i\left(1+\delta_i \nu \right)}{N} \left(1 + \frac{ \nu -\sum_i \epsilon_i (\nu+\delta_i) }{ 1-\sum_i \epsilon_i\left( 1+\delta_i \nu \right) } {\cal A} \right){\cal E} 
\end{eqnarray}
Note that at this point, by construction, we have that
\begin{equation}
   P_0 + \sum_i P_i = P_{B} + P_{\overline B} 
\end{equation}
and it should be pointed out that the untagged PDF $P_0$ is, in general, {\em not} equal to the tagged (and marginalized
over the tagdecision) PDF $P_i$
with the substitution $\epsilon_i \rightarrow \epsilon_0 \equiv 1-\sum_i \epsilon_i$ due to the term proportional with ${\cal O}$, 
which appears in equation \ref{eq:sumbbbar}. Only if either $\delta_i=0$ or $\nu=0$ (or both) will this 
substitution give the right result.

Inspired by equation \ref{eq:tagsummed}, we can introduce two new parameters:
\begin{eqnarray}
    C_{{\cal E},i} &=& 1 + \nu \delta_i 
\\  C_{{\cal O},i} &=& \delta_i + \nu
\end{eqnarray}
which, in the limit of vanishing asymmetries, would be unity and zero respectively.
In terms of these parameters, the PDFs in equation \ref{eq:tagsummed} and equation \ref{eq:untagged} can be written
as:
\begin{eqnarray}
     P_i    &=&  \frac{\epsilon_i}{N} \left[ C_{{\cal E},i}{\cal E}+ C_{{\cal O},i}{\cal O} \right] \label{eq:tagsummed2} \\
     P_0    &=&  \frac{1}{N}\left[ \left(1-\sum_i \epsilon_i C_{{\cal E},i}\right) {\cal E} + \left( \nu -\sum_i \epsilon_i C_{{\cal O},i} \right) {\cal O} \right]
\end{eqnarray}


At this point we introduce the distinction, for the tagged PDFs $P_i$, between observing a $B$ tag and a $\overline{B}$ tag.
Labelling the probability to tag, in category $i$, a true $B$ as $\overline{B}$ tag with $w_i$ and a true $\overline{B}$ as $B$ tag as $\overline{w}$,
the definition of the tagged PDF becomes\footnote{when summed together, it is obvious that one recovers eq. \ref{eq:tagsummed}}:
\begin{eqnarray}
   P_{B_\mathrm{tag},i} &\equiv t_i (1-w_i) P_{B} + \overline{t}_i\overline{w}_i P_{\overline{ B} }\\
   P_{\overline{B}_\mathrm{tag},i} &\equiv \overline{t}_i (1-\overline{w}_i) P_{\overline{B}} + t_i w_i P_{ B }
\label{eq:tagdef}
\end{eqnarray} 
Once more we replace the new parameters $w_i$ and $\overline{w}_i$ by a CP-even and a CP-odd combination, by introducing
the following parameterization:
\begin{eqnarray}
&w_i            = \frac{1}{2}\left[1-{\cal D}_i\left(1+{\Delta}_i\right) \right]; &  {\cal D}_i =1-w_i - \overline{w}_i \\
&\overline{w}_i = \frac{1}{2}\left[1-{\cal D}_i\left(1-{\Delta}_i\right) \right]; &  {\Delta}_i =  \frac{\overline{w}_i - w_i }{1-w_i-\overline{w}_i}\\
\end{eqnarray}
and thus
\begin{eqnarray}
   &1-w_i            = \frac{1}{2}\left[1+{\cal D}_i\left(1+{\Delta}_i\right) \right]; &
\\ &1-\overline{w}_i = \frac{1}{2}\left[1+{\cal D}_i\left(1-{\Delta}_i\right) \right]; &
\end{eqnarray}

Note that under $(w\rightarrow 1-w_i, \overline{w}_i\rightarrow 1-\overline{w}_i)$ we have $({\cal D}_i \rightarrow -{\cal D}_i,{\Delta}_i\rightarrow{\Delta}_i)$,
and that the quantities $\nu,\delta_i,{\Delta}_i$ are CP odd, whereas $N,\epsilon_i,{\cal D}_i$ are CP even, so the PDFs
for $B$ tags and $\overline{B}$ tags are related by the substitution $({\cal E},{\cal O};N,\nu;\epsilon_i,\delta_i;{\cal D}_i,\Delta_i )\rightarrow({\cal E},-{\cal O};N,-\nu;\epsilon_i,-\delta_i;{\cal D}_i,-\Delta_i )$.


After some tedious algebra, the PDFs for tagged events can be written as:
\begin{eqnarray}
  P_i(q_T\in (-1,+1))  &= \frac{\epsilon_i{\cal E}}{2N}\biggl[& (1+\nu\delta_i)+ (\delta_i+\nu){\cal A}   \nonumber
\\           &&+ q_T {\cal D}_i \left[ \left( (\delta_i+\nu) + {\Delta}_i (1+\nu\delta_i)\right) 
                                     + \left[ (1+\nu\delta_i) + {\Delta}_i (\delta_i +\nu) \right] {\cal A}  \right] \biggr] \nonumber
 \\                    &= \frac{\epsilon_i{\cal E}}{2}\frac{1+\nu\delta_i}{N}\biggl[&   1 + \frac{\delta_i+\nu}{1+\nu\delta_i}{\cal A}    \nonumber
               +  q_T {\cal D}_i \frac{\left[ 1 + {\Delta}_i \frac{\delta_i +\nu}{(1+\nu\delta_i}  \right] {\cal A}+ \left( \frac{\delta_i+\nu}{1+\nu\delta_i } + {\Delta}_i \right)  }{1} \biggr] \nonumber
 \\                    &= \frac{\epsilon_i{\cal E}}{2}\frac{1+\nu\delta_i}{N}\biggl[&   1 + \frac{\delta_i+\nu}{1+\nu\delta_i}{\cal A}    \nonumber
               +  q_T {\cal D}_i \left[ 1 + {\Delta}_i \frac{\delta_i +\nu}{(1+\nu\delta_i}  \right] \left( \frac{{\cal A}+ \left( \frac{\delta_i+\nu}{1+\nu\delta_i } + {\Delta}_i \right)  }{\left[ 1 + {\Delta}_i \frac{\delta_i +\nu}{(1+\nu\delta_i}  \right] }\right) \biggr] \nonumber
\end{eqnarray}
\begin{eqnarray}
  P_i(q_T\in (-1,+1))  &= \frac{\epsilon_i}{2N}\biggl[& C_{{\cal E},i}{\cal E} + C_{{\cal O},i}{\cal O}   \nonumber
\\           &&+ q_T {\cal D}_i \left[ \left( C_{{\cal O},i} + {\Delta}_i C_{{\cal E},i} \right) {\cal E} 
                                     + \left( C_{{\cal E},i} + {\Delta}_i C_{{\cal O},i} \right) {\cal O}  \right] \biggr] \nonumber
% \\ &=& \frac{\epsilon_i}{N} \left[ C_{{\cal E},i}{\cal E} + C_{{\cal O},i}{\cal O} \right] \frac{ 1 + q_T A_{CP}(i)}{2}
\\ &\equiv& P_i \times \left( \frac{ 1}{2} + \frac{1}{2} q_T A_{\mathrm{obs}}(i) \right)
\label{eq:substrule_e}
\end{eqnarray}
where the observed $CP$ asymmetry is given by:
\begin{eqnarray}
  A_{\mathrm{obs}}(i) &=&  {\cal D}_i \frac{\left(C_{{\cal O},i}+{\Delta}_i C_{{\cal E},i}\right) {\cal E}+\left( C_{{\cal E},i} + {\Delta}_i C_{{\cal O},i} \right){\cal O}}
                               { C_{{\cal E},i}{\cal E}+ C_{{\cal O},i}{\cal O} }
\\  &=& {\cal D}_i \left[ \Delta_i + \frac{C_{{\cal E},i}{\cal A}+C_{{\cal O},i} }{ C_{{\cal E},i} + C_{{\cal O},i}{\cal A} }\right]
\\A_{\mathrm{obs}}(i) &=&    {\cal D}_i \left[ \Delta_i + \frac{\delta+\frac{\nu+{\cal A}}{1+\nu{\cal A}}}{1+\delta\frac{\nu+{\cal A}}{1+\nu{\cal A}}}\right] 
%\\          &=&    {\cal D}_i \left[ \Delta_i + \frac{\delta+\nu+{\cal A}+\delta\nu{\cal A}}{1+\nu{\cal A}+\delta\nu+\delta{\cal A}} \right]
%\\          &=&    {\cal D}_i \left[ \Delta_i + \frac{{\cal A}(1+\delta\nu)+\delta+\nu}{1+\delta\nu+{\cal A}(\delta+\nu)} \right] 
%\\          &=&    {\cal D}_i \left[ \Delta_i + \frac{{\cal A}+\frac{\delta+\nu}{1+\delta\nu}}{1+{\cal A}\frac{\delta+\nu}{1+\delta\nu}} \right] \label{eq:ACPintermsofA}
\\          &=&    {\cal D}_i \frac{ {\cal A}+\Delta_i +\delta_i + \nu + \delta_i\nu\Delta_i + {\cal A}\delta_i\Delta_i + {\cal A}\nu\Delta_i + {\cal A} \delta_i\nu }{1 +\delta_i\nu + {\cal A}\delta_i + {\cal A}\nu}
\\          &=&    {\cal D}_i \frac{ {\cal A}\left(1 + \delta_i \nu +\delta_i\Delta_i + \nu\Delta_i \right)  +\delta_i + \nu+\Delta_i + \delta_i\nu\Delta_i  }{1 +\delta_i\nu + {\cal A}( \delta_i +\nu)}
\\       &\approx& {\cal D}_i \left[ {\cal A} + \Delta_i + \delta + \nu \right]
\end{eqnarray}
In the limit where $\nu=\delta_i+{\Delta}=0$, i.e. $C_{{\cal E},i} =1, C_{{\cal O},i}=0, \Delta_i = 0$, this reduces to the expected
\begin{eqnarray}
   P_0 &=&\frac{1}{N} \left[ 1-\sum_i \epsilon_i \right]{\cal E} \\
   P_i(q_T\in(-1,1)) &=& \frac{\epsilon_i{\cal E}}{2N}\left[ 1  +q_T {\cal D}_i {\cal A}\right]  \label{eq:substrule_s}\\
   A_{\mathrm{obs}}(i) &=&   {\cal D}_i {\cal A}
\end{eqnarray}
In addition, solving eq. \ref{eq:ACPintermsofA} for $\cal A$, it is clear that the effect on ${\cal A}$ comes in the form of a (first order) bias on the central value and (second order) on the dilution (i.e. as a scalefactor in the value of ${\cal A}$):
\begin{eqnarray}
    {\cal A} &=& %\frac{ (A_{\mathrm{obs}}/{\cal D}-\Delta)(1+\delta\nu)- (\delta+\nu) }{1+\delta\nu - ( A_{\mathrm{obs}}/{\cal D}-\Delta ) (\delta+\nu) }
% \\     &=& 
%     \frac{ {A_{\mathrm{obs}}(1+\delta\nu)}-{\cal D}\left[ \Delta(1+\delta\nu)  + \delta+\nu \right] }{{\cal D}\left[1+\delta\nu+\Delta(\delta+\nu)\right] -  A_{\mathrm{obs}}(\delta+\nu)   }
%    {\cal A} = \frac{A_{\mathrm{obs}}-{\cal D}\left(\nu+\delta+\Delta-\delta^2\nu+\delta\nu^2\right)}{{\cal D}(1+\delta^2+\nu^2+2\delta\nu+{\cal A}(\nu+\delta))}
%\\ &=& 
\frac{A_{\mathrm{obs}}-{\cal D}(\Delta+\frac{\delta+\nu }{1+\delta\nu})}{ {\cal D}\left[   1+\Delta\frac{\delta+\nu}{1+\delta\nu}\right]-A_{\mathrm{obs}}\frac{\delta+\nu}{1+\delta\nu} } 
%\\ &\approx& \frac{A_{\mathrm{obs}}-{\cal D}(\Delta+\delta+\nu )}{ {\cal D}\left[   1+\Delta\delta+\Delta\nu\right]-A_{\mathrm{obs}}(\delta+\nu) } 
\\ &\approx& \frac{ \dfrac{A_{\mathrm{obs}}}{\cal D}-(\Delta+\delta+\nu )}{   1+\Delta\delta+\Delta\nu- \dfrac{A_{\mathrm{obs}}}{\cal D}(\delta+\nu) } 
\end{eqnarray}

As $\nu$ and $\delta_i$ always appear as sum, i.e. as $C_{{\cal O},i}$ (if we neglect the quadratic terms $\nu\delta_i$), and thus $\nu$ will be fully anticorrelated to the ($\epsilon_i$ weighted) average of $\delta_i$,
it becomes impossible to disentangle the difference between differences in tagging efficiency for $b$ and $\overline{b}$
and a change in the relative normalization. However, when taken into account the fact that the difference in normalization
is due to the apearance of $1/(1+C)$ vs. $1/(1-C)$, i.e. $\nu = -C$, and $C$ also appears as coefficienct in front of the $\cosh\left(\Delta\Gamma t/2\right)$
and $q_T\cos(\Delta m t)$ terms, it is expected that there is some residual power left to disentangle the two effects -- i.e. 
solely due to the time-dependent shape of the distributions, and no longer in their relative normalization, i.e. in the time-integrated yields.


Note that these mistag, tagging efficiency and normalization differences result in a fairly 
straightforward substitution rule, which one can read off by comparing Equations (\ref{eq:substrule_s} - \ref{eq:substrule_e}):
\begin{enumerate}
\item
the initially CP-even terms (i.e. the $\cosh(\Delta\Gamma t/2)$ and $\sinh(\Delta\Gamma t/2)$ terms) obtain an additional
factor 
\begin{equation}
   {\cal E} \rightarrow {\cal E} \left[ C_{{\cal E},i} + q_T {\cal D}_i \left( C_{{\cal O},i}+{\Delta}_i C_{{\cal E},i}) ]\right) \right]
\end{equation}
\item
the initially CP-odd terms (i.e. the $\cos(\Delta m t) $ and $\sin(\Delta m t)$ terms) obtain, { \em instead of the
usual factor} $q_T{\cal D}_i$, the factor
\begin{equation}
   {\cal O} \rightarrow {\cal O} \left[ C_{{\cal O},i} +  q_T {\cal D}_i  \left( C_{{\cal E},i} + {\Delta}_iC_{{\cal O},i} \right) \right]
\end{equation}
\end{enumerate}
Again, in the limit that $\delta_i=\nu={\Delta}_i=0$, i.e. $(C_{{\cal E},i},C_{{\cal O},i},\Delta_i) = (1,0,0)$ this results in the well-known expected behaviour.

It should be noted that the tagged PDFs, i.e. $P_i$, are linear both in $C_{{\cal E},i}$ and $C_{{\cal O},i}$,
and, as a result, scaling these parameters by a common 
factor only changes the total normalization, and hence leaves the problem invariant. As a result, one
may as well decide to fix $C_{{\cal E},i}$ to unity. This choice is also motivated that any deviation of $C_{{\cal E},i}$ from
unity is 2nd order, whereas a deviation of $C_{{\cal O},i}$ from zero is first order. 
This picture changes qualitatively when the untagged events are included. In that case, since
the system is now closed in the sense that events which are 'lost' in the tagged $P_i$ are now 
'gained' in the untagged $P_0$, or, to put it another way, the appearance of a '1' in $P_0$ defines
the overall normalization, it should be possible to determine $C_{{\cal E},i}, C_{{\cal O},i}$ and
as a bonus, also $\nu$, simultaneously.



\vfill
\pagebreak
At this point we introduce the distinction, for the tagged PDFs $P_i$, between observing a $B$ tag and a $\overline{B}$ tag.
Labelling the probability to tag with a true $B$ as $\overline{B}$ tag with $w_i$ and a true $\overline{B}$ as $B$ tag as $\overline{w}$,
where $w$ and $\overline{w}$ depend on an observable $\omega$,
the definition of the tagged PDF becomes\footnote{
when the distribution of $\omega$ is actually a Dirac comb, and the mistag $w(\overline{w})$ is directly given by $\omega(\overline{\omega})$, i.e. $ w(\omega) = \omega, P_\omega = \sum t_i \delta(\omega-w_i)$ 
and $\overline{w} = \omega, \overline{P}_\omega = \sum \overline{t}_i \delta(\omega-\overline{w}_i) $, we recover
eq. \ref{eq:tagdef}}:
\begin{eqnarray}
   P_{B_\mathrm{tag},\omega} &\equiv  (1-w(\omega)) P_{B} P_\omega  + \overline{w}(\omega) P_{\overline{ B} }\overline{P}_\omega \\
   P_{\overline{B}_\mathrm{tag},\omega} &\equiv (1-\overline{w}(\omega)) P_{\overline{B}}\overline{P}_\omega  +  w(\omega) P_{ B } P_\omega
\end{eqnarray} 
Unfortunately, as defined, $P_\omega$ and $\overline{P}_\omega$ are not direclty measurable.
If instead we take the observed distributions of $\omega$ for $B$  and $\overline{B}$, we have:
\begin{eqnarray}
   P_{B_\mathrm{tag},\omega} &\equiv P_\omega \left[ (1-w(\omega)) P_{B} + \overline{w}(\omega) P_{\overline{ B} } \right]\\
   P_{\overline{B}_\mathrm{tag},\omega} &\equiv \overline{P}_\omega\left[ (1-\overline{w}(\omega)) P_{\overline{B}} +  w(\omega) P_{ B } \right]
\end{eqnarray} 

\section{Angular Dependence}

If we write the direction of the positive muon (in which reference frame??) in terms of transversity angles,
\begin{equation}
  \hat{n} = ( \sin\thetatr\cos\phitr, \sin\thetatr\sin\phitr, \cos\thetatr )
\end{equation}
or helicity angles
\begin{equation}
  \hat{n} = ( -\cos\thetaL, -\sin\thetaL\cos\phiL, \sin\thetaL\sin\phiL )
\end{equation}
and define transversity amplitudes
\begin{eqnarray}
   {\bf A}(t)            &=&  ( {\cal A}_0(t) \cos\psi          , -\frac{{\cal A}_\parallel(t) \sin\psi}{\sqrt{2}}         ,i\frac{{\cal A}_\perp(t) \sin\psi}{\sqrt{2}}          )  % \\
%   {\overline{\bf A}}(t) &=&  ( \overline{\cal A}_0(t) \cos\psi), -\frac{\overline{\cal A}_\parallel(t) \sin\psi}{\sqrt{2}},i\frac{\overline{\cal A}_\perp(t) \sin\psi}{\sqrt{2}} )
\end{eqnarray}
then one can write
\begin{eqnarray}
  { \bf A}(t) \wedge \hat{n} &=& \left|  \begin{array}{ccc}  
                                        \hat{e}_x & \hat{e}_y & \hat{e}_z \\
                                        {\cal A}_0(t) \cos\psi & - \frac{{\cal A}_\parallel(t) \sin\psi }{\sqrt{2}} & i\frac{ {\cal A}_\perp(t)\sin\psi}{\sqrt{2}}\\
                                        \sin\theta\cos\phi & \sin\theta\sin\phi & \cos\theta 
                                    \end{array}\right|  \\
                          &=& \left(\begin{array}{c}
                                  \frac{-1}{\sqrt{2}} {\cal A}_\parallel(t) \sin\psi \cos\theta  -  \frac{i }{\sqrt{2}}{\cal A}_\perp(t)\sin\psi\sin\theta\sin\phi  \\
                                   -{\cal A}_0(t) \cos\psi\cos\theta  + \frac{i}{\sqrt{2}} {\cal A}_\perp(t)\sin\psi \sin\theta\cos\phi \\
                                   {\cal A}_0(t) \cos\psi \sin\theta\sin\phi       + \frac{1}{\sqrt{2}}{\cal A}_\parallel(t) \sin\psi \sin\theta\cos\phi 
                              \end{array} \right)
\end{eqnarray}

\begin{eqnarray}
\label{arr:bigarray}
  |{ \bf A}(t) \wedge \hat{n}|^2 &=& 
                                  \frac{1}{2} |{\cal A}_\parallel(t)|^2 \sin^2\psi \cos^2\theta  +  \frac{1 }{2}|{\cal A}_\perp(t)|^2\sin^2\psi\sin^2\theta\sin^2\phi  \\
                              &&+     -\Im({\cal A}_\parallel^*(t){\cal A}_\perp(t))\sin^2\psi\cos\theta\sin\theta\sin\phi  \\
                              &+&     |{\cal A}_0(t)|^2 \cos^2\psi\cos^2\theta  + \frac{1}{2} |{\cal A}_\perp(t)|^2\sin^2\psi \sin^2\theta\cos^2\phi \\
                              &&+     -\sqrt{2}\Im({\cal A}_\perp^*(t) {\cal A}_0(t) )\cos\psi\cos\theta\sin\psi\sin\theta\cos\phi \\
                              &+&     |{\cal A}_0(t)|^2 \cos^2\psi\sin^2\theta\sin^2\phi  + \frac{1}{2}{|\cal A}_\parallel(t)|^2 \sin^2\psi \sin^2\theta\cos^2\phi       \\
                              &&+     \sqrt{2}\Re({\cal A}_0^*(t){\cal A}_\parallel(t)  )\cos\psi\sin\psi\sin^2\theta\sin\phi\cos\phi\\
                              &=&  |{\cal A}_0(t)|^2 \cos^2\psi\left( \cos^2\theta  +  \sin^2\theta\sin^2\phi\right) \\
                              &+& \frac{1}{2}|{\cal A}_\parallel(t)|^2 \sin^2\psi \left( \cos^2\theta + \sin^2\theta\cos^2\phi    \right)  \\
                              &+&\frac{1}{2}|{\cal A}_\perp(t)|^2\sin^2\psi\sin^2\theta \\
                              &+&     -\Im({\cal A}_\parallel^*(t){\cal A}_\perp(t))\sin^2\psi\cos\theta\sin\theta\sin\phi  \\
                              &+&     -\sqrt{2}\Im({\cal A}_\perp^*(t) {\cal A}_0(t) )\cos\psi\cos\theta\sin\psi\sin\theta\cos\phi \\
                              &+&     \sqrt{2}\Re({\cal A}_0^*(t){\cal A}_\parallel(t)  )\cos\psi\sin\psi\sin^2\theta\sin\phi\cos\phi\\
                              &=&\frac{1}{2}|{\cal A}_\parallel(t)|^2 \sin^2\psi \left( 1 -  \sin^2\theta\sin^2\phi    \right)  \\
                              &+&\frac{1}{2}|{\cal A}_\perp(t)|^2\sin^2\psi\sin^2\theta \\
                              &+&   |{\cal A}_0(t)|^2 \cos^2\psi\left( 1  -  \sin^2\theta\cos^2\phi\right) \\
                              &+&     -\frac{1}{2}\Im({\cal A}_\parallel^*(t){\cal A}_\perp(t))\sin^2\psi\sin(2\theta)\sin\phi  \\
                              &+&     \frac{1}{2\sqrt{2}}\Im({\cal A}_0^*(t) {\cal A}_\perp(t) ) \sin(2\psi)\sin(2\theta)\cos\phi \\
                              &+&     \frac{1}{2\sqrt{2}}\Re({\cal A}_0^*(t) {\cal A}_\parallel(t))\sin(2\psi)\sin^2\theta\sin(2\phi)
\end{eqnarray}
Integrating the sum of all these terms we get
\begin{equation}
\frac{16\pi}{9}(|{\cal A}_{0}(t)|^{2}+|{\cal A}_{\parallel}(t)|^{2}+|{\cal A}_{\perp}(t)|^{2}))
\end{equation}
By calculating  $\frac{9}{16\pi}|{ \bf A}(t) \wedge \hat{n}|^2$ instead of $|{ \bf A}(t) \wedge \hat{n}|^2$ we obtain the PDF with normalization $|{\cal A}_{0}(t)|^{2}+|{\cal A}_{\parallel}(t)|^{2}+|{\cal A}_{\perp}(t)|^{2}$. 
\subsection{S-wave and P-S interference angular functions}\label{sec:Swaveangles}
\subsubsection{S-wave angular function}
One can obtain the angular function of a pure S-wave component by calculating a cross-product in the same manner as in the previous section. For a pure S-wave the object to calculate is
\begin{equation}
\frac{3}{16\pi}|{\bf S}(t) \wedge \hat{n}|^2\quad,
\end{equation}
where
\begin{equation}
{\bf S}(t) = ({\cal A}_{S}(t),0,0) \quad ,
\end{equation}
since the S-wave configuration has only vector meson in the decay, and thus must be necessarily longitudinally polarized (see e.g. Rosner, Phys Rev D42, 3732). This yields
\begin{equation}
\frac{3}{16\pi}|{\bf S}(t) \wedge \hat{n}|^{2}=|{\cal A}_{S}(t)|^2 (1-\sin^2\theta\cos^2\phi)
\end{equation}
\subsubsection{P-S interference angular functions}
The P-S interference angular functions are the interference terms between pure P-wave (${ \bf A}(t)$) and S-wave (${ \bf S}(t)$) in the expression
\begin{equation}
\frac{9}{16\pi}\left|[{ \bf A}(t) + \frac{1}{\sqrt{3}}{\bf S}(t)] \wedge \hat{n}\right|^2
\end{equation}
These interference terms are found to be
\begin{eqnarray}
&\frac{9}{16\pi}&[\Re({\cal{A}}_S^*(t){\cal{A}}_\parallel(t))\frac{1}{6}\sqrt{6}\sin\psi\sin^2\theta\sin 2\phi \\
&+&\frac{9}{16\pi}\Im({\cal{A}}_S^*(t){\cal{A}}_\perp(t))\frac{1}{6}\sqrt{6}\sin\psi\sin2\theta\cos\phi \\
&+&\frac{9}{16 \pi}\Re({\cal{A}}_S^*(t){\cal{A}}_0(t))\frac{2}{3}\sqrt{3}\cos\psi(1-\sin^2\theta\cos^2\phi)]
\end{eqnarray}

The pure P-wave, pure S-wave and P-S interference angular functions are summarized in table \ref{tab:canonicalpdf}.
\begin{sidewaystable}[htb]

  \begin{center}
    \renewcommand{\arraystretch}{1.5}
    \begin{tabular}{|c|c|c|c|c|c|}
    %\begin{tabular}{|b{2cm}|b{3cm}|b{4cm}|b{3cm}|}
      \hline
      amplitudes & constant & \multicolumn{2}{|c|}{transversity angular function }  &\multicolumn{2}{|c|}{ helicity angular function}\\
      \hline
      \hline
      $|{\cal{A}}_{0}(t)|^2$ & $\frac{9}{16 \pi}$
        &  $\cos^2\psitr \left(1 - \sin^2\theta_\mathrm{tr} \cos^2\phi_\mathrm{tr}\right)$  
        &  $\cos^2\psitr \left(1 - x_\mathrm{tr}^2 \right) $ 
        &  $\cos^2\thetaK \left(1 - z_\mathrm{hel}^2 \right) $ 
        &  $\cos^2\thetaK \left(1 - \cos^2\theta_\ell \right) $ \\
      $|{\cal{A}}_{\parallel}(t)|^2$ & $\frac{9}{16 \pi}$
        &  $\frac{1}{2}\sin^2\psitr \left(1 - \sin^2\theta_\mathrm{tr} \sin^2\phi_\mathrm{tr}\right)$  
        &  $\frac{1}{2}\sin^2\psitr \left(1 - y_\mathrm{tr}^2 \right)$
        &  $\frac{1}{2}\sin^2\thetaK \left(1 - x_\mathrm{hel}^2 \right)$
        &  $\frac{1}{2}\sin^2\thetaK \left(1 - \sin^2\thetaL\cos^2\phiL \right)$\\
      $|{\cal{A}}_{\perp}(t)|^2$ & $\frac{9}{16 \pi}$
        &  $\frac{1}{2}\sin^2\psitr \sin^2\theta_\mathrm{tr}$  
        &  $\frac{1}{2}\sin^2\psitr \left(1- z_\mathrm{tr}^2\right)$  
        &  $\frac{1}{2}\sin^2\thetaK\left(1- y_\mathrm{hel}^2\right)$  
        &  $\frac{1}{2}\sin^2\thetaK\left(1-\sin^2\thetaL\sin^2\phiL\right)$  \\
      $\Re({\cal{A}}_{0}^{*}(t)\,{\cal{A}}_{\parallel}(t))$ & $\frac{9}{16 \pi}$
        &   $\frac{1}{4}\sqrt{2}$ $\sin2\psitr \sin^2\theta_\mathrm{tr} \sin2\phi_\mathrm{tr}$  
        &   $\frac{1}{2}\sqrt{2}$ $\sin2\psitr\; x_\mathrm{tr}y_\mathrm{tr}$ 
        &   $\frac{1}{2}\sqrt{2}$ $\sin2\thetaK\; x_\mathrm{hel}z_\mathrm{hel}$ 
        &   $\frac{1}{4}\sqrt{2}$ $\sin2\thetaK\sin2\thetaL\cos\phiL$ \\
      $\Im({\cal{A}}_{0}^{*}(t)\,{\cal{A}}_{\perp}(t))$ & $\frac{9}{16 \pi}$
        &  $\frac{1}{4}\sqrt{2}$ $\sin2\psitr \sin2\theta_\mathrm{tr} \cos\phi_\mathrm{tr}$  
        &  $\frac{1}{2}\sqrt{2}$ $\sin2\psitr\; x_\mathrm{tr}z_\mathrm{tr}$  
        &  $-\frac{1}{2}\sqrt{2}$ $\sin2\thetaK\; y_\mathrm{hel}z_\mathrm{hel}$  
        &  $-\frac{1}{4}\sqrt{2}$ $\sin2\thetaK\sin2\thetaL\sin\phiL$\\
      $\Im({\cal{A}}_{\parallel}^{*}(t)\,{\cal{A}}_{\perp}(t))$ & $\frac{9}{16 \pi}$
        &  $-\frac{1}{2}\sin^2\psitr \sin2\theta_\mathrm{tr} \sin\phi_\mathrm{tr}$  
        &  $-\sin^2\psitr \;y_\mathrm{tr}z_\mathrm{tr}$
        &  $+\sin^2\thetaK \;x_\mathrm{hel}y_\mathrm{hel}$
        &  $+\frac{1}{2}\sin^2\thetaK \sin^2\thetaL\sin2\phiL$\\
      $|{\cal{A}}_S(t)|^2$ & $\frac{9}{16 \pi}$                     
        & $\frac{1}{3} (1-\sin^2\theta_\mathrm{tr}\cos^2\phi_\mathrm{tr})$  
        & $\frac{1}{3} (1-x_\mathrm{tr}^2)$  
        & $\frac{1}{3} (1-z_\mathrm{hel}^2)$  
        & $\frac{1}{3} (1-\cos^2\thetaL)$  \\
      $ \Re({\cal{A}}_S^*(t){\cal{A}}_0(t)) $ & $\frac{9}{16 \pi} $
        & $\frac{2}{3}\sqrt{3}\cos\psitr(1-\sin^2\theta_\mathrm{tr}\cos^2\phi_\mathrm{tr})$ 
        & $\frac{2}{3}\sqrt{3}\cos\psitr(1-x_\mathrm{tr}^2)$ 
        & $\frac{2}{3}\sqrt{3}\cos\thetaK(1-z_\mathrm{hel}^2)$ 
        & $\frac{2}{3}\sqrt{3}\cos\thetaK(1-\cos^2\thetaL)$ \\
      $\Re({\cal{A}}_S^*(t){\cal{A}}_\parallel(t))$ & $\frac{9}{16 \pi} $
        &  $\frac{1}{6}\sqrt{6}\sin\psitr\sin^2\theta_\mathrm{tr}\sin 2\phi_\mathrm{tr}$ 
        &  $\frac{1}{3}\sqrt{6}\sin\psitr\; x_\mathrm{tr}y_\mathrm{tr}$ 
        &  $\frac{1}{3}\sqrt{6}\sin\thetaK\; x_\mathrm{hel}z_\mathrm{hel}$ 
        &  $\frac{1}{6}\sqrt{6}\sin\thetaK\sin2\thetaL\cos\phiL$ \\
      $\Im({\cal{A}}_S^*(t){\cal{A}}_\perp(t))$ & $\frac{9}{16 \pi}$ 
        &  $\frac{1}{6}\sqrt{6}\sin\psitr\sin2\theta_\mathrm{tr}\cos\phi_\mathrm{tr}$ 
        &  $\frac{1}{3}\sqrt{6}\sin\psitr\; x_\mathrm{tr} z_\mathrm{tr}$ 
        & -$\frac{1}{3}\sqrt{6}\sin\thetaK\; y_\mathrm{hel} z_\mathrm{hel}$ 
        & -$\frac{1}{3}\sqrt{6}\sin\thetaK\sin2\thetaL\sin\phiL$ \\
       \hline
    \end{tabular}
  \end{center}
  \caption{The pure P-wave, pure S-wave and P-S interference angular functions.}\label{tab:canonicalpdf}
\end{sidewaystable}

\subsection{Reparameterization}
We want to rewrite the PDF in terms of associated Legendre polynomials $P_{i}(\cos\psi)$ and spherical harmonics $Y_{lm}(\cos\theta,\phi)$. First we will give some identities involving these functions:
\begin{eqnarray}
P_l(x) &=& \frac{1}{2^l l!} \frac{d^l}{dx^l}(x^2-1)^l \\
P_l^m(x) &=& \frac{(-1)^m}{2^l l!} (1-x^2)^{\frac{1}{2}m} \frac{d^{l+m}}{dx^{l+m}} (x^2-1)^l\\
P_l^m(x) &=& (-1)^m (1-x^2)^{\frac{1}{2}m} \frac{d^{m}}{dx^{m}} P_l(x)\\
%P_l^{-m}(x) & = & (-1)^m \frac{(l-m)!}{(l+m)!}P_l^m(x) \\
Y_l^m (\theta,\phi) &=& N_{lm} P_l^m(\cos\theta)e^{im\phi}\;\;\mathrm{with}\;N_{lm} =\sqrt{ \frac{2l+1}{4\pi}\frac{(l-m)!}{(l+m)!} } \\
Y_l^{-m}(\theta,\phi) &=& (-1)^m N_{lm}P_l^m(\cos\theta) e^{-im\phi}
\end{eqnarray}

\begin{eqnarray}
Y_{lm}(\theta,\phi) & = & \left\{ \begin{array}{cl} 
                                             Y_l^0(\theta,\phi) & (m=0) \\
                                             \frac{1}{\sqrt{2}} \left( Y_l^m + (-1)^m Y_l^{-m}\right)  & (m>0) \\
                                             \frac{1}{i\sqrt{2}}\left( Y_l^{|m|}-(-1)^{|m|}Y_l^{-|m|} \right) & (m<0)
                                  \end{array}\right.\\
                    & = & \left\{ \begin{array}{cl} 
                                             N_{l0} P_l^0(\cos\theta) & (m=0) \\
                                             \sqrt{2}N_{lm}P_l^m(\cos\theta)\cos(m\phi) & (m>0) \\
                                             \sqrt{2}N_{l|m|} P_l^{|m|}(\cos\theta)\sin(|m|\phi)& (m<0)
                                  \end{array}\right.
\end{eqnarray}

\begin{eqnarray}
  1                       &=& \sqrt{ \frac{4\pi}{9} } \left( 3 Y_{00}(\cos\theta,\phi) \right) \\
  \cos^2\theta            &=& \sqrt{ \frac{4\pi}{9} } \left( Y_{00}(\cos\theta,\phi) + \sqrt{\frac{4}{5}}Y_{20}(\cos\theta,\phi)\right)\\
  \sin^2\theta            &=& \sqrt{ \frac{4\pi}{9} } \left( 2Y_{00}(\cos\theta,\phi) - \sqrt{\frac{4}{5}}Y_{20}(\cos\theta,\phi)\right)\\
  \sin^2\theta \cos^2\phi &=& \sqrt{ \frac{4\pi}{9} } \left( Y_{00}(\cos\theta,\phi) - \sqrt{\frac{1}{5} }Y_{20}(\cos\theta,\phi) +\sqrt{\frac{3}{5}} Y_{2,2}(\cos\theta,\phi)\right) \\
  \sin^2\theta \sin^2\phi &=& \sqrt{ \frac{4\pi}{9} } \left( Y_{00}(\cos\theta,\phi) - \sqrt{\frac{1}{5} }Y_{20}(\cos\theta,\phi) -\sqrt{\frac{3}{5}} Y_{2,2}(\cos\theta,\phi)\right) \\
  \sin^2\theta\cos\phi\sin\phi &=& \sqrt{ \frac{4\pi}{9}} \left(\sqrt{\frac{3}{5}} Y_{2,-2}(\cos\theta,\phi) \right) \\
  \sin\theta\cos\theta\cos\phi &=& \sqrt{ \frac{4\pi}{9}}\left( -\sqrt{\frac{3}{5}}Y_{2,1}(\cos\theta,\phi)\right) \\ 
  \sin\theta\cos\theta\sin\phi &=& \sqrt{ \frac{4\pi}{9}}\left( -\sqrt{\frac{3}{5}}Y_{2,-1}(\cos\theta,\phi) \right) \\
  1 & = & P_0^0(\cos\psi)\\
  \sin \psi & = & -P_1^1(\cos\psi)\\
  \cos \psi & = & P_0^1(\cos\psi)\\
  \sin\psi\cos\psi &=& -\frac{1}{3} P_2^1(\cos\psi) \\
  \sin^2 \psi      &=&  \frac{1}{3} P_2^2(\cos\psi) \\
  \cos^2 \psi      &=&  \frac{1}{3} \left( P_0^0(\cos\psi)+2P_2^0(\cos\psi) \right)
\end{eqnarray}
Using these identities one can rewrite the PDF as 
\begin{eqnarray}
\frac{9 }{\sqrt{4\pi}} \left|[{ \bf A}(t) + \frac{1}{\sqrt{3}}{\bf S}(t)] \wedge \hat{n}\right|^2
                              &=&|{\cal A}_0(t)|^2  (P_0^0+2P_2^0 ) \left( 2 Y_{00}+\sqrt{\frac{1}{5}}Y_{20}-\sqrt{\frac{3}{5}}Y_{22} \right) \\
                              &+&|{\cal A}_\parallel(t)|^2 P_2^2 \left( Y_{00}+\sqrt{\frac{1}{20}}Y_{20}+\sqrt{\frac{3}{20}}Y_{22}  \right)  \\
                              &+&|{\cal A}_\perp(t)|^2  P_2^2 \left( Y_{00} - \sqrt{\frac{1}{5}}Y_{20}\right)\\
                              &+&\Im({\cal A}_\parallel^*(t){\cal A}_\perp(t)) \sqrt{\frac{3}{5}}P_2^2 Y_{2,-1}  \\
                              &-&\Re({\cal A}_0^*(t){\cal A}_\parallel(t)  )   \sqrt{\frac{6}{5}}P_2^1 Y_{2,-2} \\
                              &+&\Im({\cal A}_0^*(t) {\cal A}_\perp(t) )       \sqrt{\frac{6}{5}}P_2^1Y_{2,1} \\
                              &+&|{\cal A}_0(t)|^2 P_0^0\left( 2 Y_{00}+\sqrt{\frac{1}{5}}Y_{20}-\sqrt{\frac{3}{5}}Y_{22} \right)\\
                              &-&\Re({\cal A}_S^*(t){\cal A}_\parallel(t))3\sqrt{\frac{2}{5}}P_1^1Y_{2,-2}\\
                              &+&\Im({\cal A}_S^*(t){\cal A}_\perp(t))3\sqrt{\frac{2}{5}}P_1^1Y_{2,1}\\
                              &+&\Re({\cal A}_S^*(t){\cal A}_0(t)) P_1^0(4\sqrt{3}Y_{00}+2\sqrt{\frac{3}{5}}Y_{20}-6\sqrt{\frac{1}{5}}Y_{22})
\end{eqnarray}
This can be summarized in tabular form as
\begin{equation}\label{tab:pdfinpy}
\begin{array}{|c|| c | c | c | c | c | c |}
 \hline
&\multicolumn{6}{|c|}{ P_i^j(\cos\psitr)Y_{lm}(\cos\thetatr,\phitr) } \\
 \hline
|{\cal A}_0(t)|^2                       &  2 P_0^0Y_{00}  & \sqrt{\frac{1}{5}} P_0^0 Y_{20} & - \sqrt{\frac{3}{5}} P_0^0 Y_{22}
                                        &  4 P_2^0Y_{00}  & 2 \sqrt{\frac{1}{5}} P_2^0 Y_{20} & -2 \sqrt{\frac{3}{5}} P_2^0 Y_{22} \\
|{\cal A}_\parallel(t)|^2               &    P_2^2Y_{00} &  \frac{1}{2}\sqrt{\frac{1}{5}} P_2^2 Y_{20} & \frac{1}{2}\sqrt{\frac{3}{5}} P_2^2Y_{22} & && \\
|{\cal A}_\perp(t)|^2                   &  P_2^2 Y_{00} & -\sqrt{\frac{1}{5}} P_2^2 Y_{20} &                                 &&& \\
\Re({\cal A}_0^*(t){\cal A}_\parallel(t))     & -\sqrt{\frac{6}{5}} P_2^1 Y_{2,-2} &&&&&\\
\Im({\cal A}_0^*(t){\cal A}_\perp(t))         &  \sqrt{\frac{6}{5}} P_2^1 Y_{2,1} &&&&&\\
\Im({\cal A}_\parallel^*(t){\cal A}_\perp(t)) &  \sqrt{\frac{3}{5}} P_2^2 Y_{2,-1} &&&&&\\
|{\cal A}_{s}(t)|^2                   &  2 P_0^0 Y_{00} & \sqrt{\frac{1}{5}} P_0^0 Y_{20} & -\sqrt{\frac{3}{5}} P_0^0 Y_{22}                                &&& \\
\Re({\cal A}_{s}^*(t){\cal A}_0(t))   &  4 \sqrt{3} P_1^0 Y_{00} & 2 \sqrt{\frac{3}{5}} P_1^0 Y_{20} & - 6 \sqrt{\frac{1}{5}} P_1^0 Y_{22}                       &&& \\
\Re({\cal A}_{s}^*(t){\cal A}_\parallel(t))   &  -3\sqrt{\frac{2}{5}} P_1^1 Y_{2,-2} &&&&& \\
\Im({\cal A}_{s}^*(t){\cal A}_\perp(t))     &  3\sqrt{\frac{2}{5}} P_1^1 Y_{2,1} &&&&& \\
 \hline
\end{array}
\end{equation}
Rewriting this in terms of helicity angles instead of transversity angles one obtains
\begin{equation}
\begin{array}{|c|| c | c | c | c | }
 \hline
&\multicolumn{4}{|c|}{ P_i^j(\cos\thetaK)Y_{lm}(\cos\thetaL,\phiL) } \\
 \hline
|{\cal A}_0(t)|^2                       & 2P_0^0Y_{00} & -\sqrt{\frac{4}{5}}P_0^0Y_{20} & 4P_2^0Y_{00} & -\sqrt{\frac{16}{5}} P_2^0Y_{20}  \\
|{\cal A}_\parallel(t)|^2               & P_2^2Y_{00} &\sqrt{\frac{1}{20}} P_2^2Y_{20} &-\sqrt{\frac{3}{20}}P_2^2Y_{22} & \\
|{\cal A}_\perp(t)|^2                   & P_2^2Y_{00} & \sqrt{\frac{1}{20}} P_2^2Y_{20} &\sqrt{\frac{3}{20}}P_2^2Y_{22} & \\
\Re({\cal A}_0^*(t){\cal A}_\parallel(t))     & \sqrt{\frac{6}{5}}P_2^1Y_{21}&&&\\
\Im({\cal A}_0^*(t){\cal A}_\perp(t))         & -\sqrt{\frac{6}{5}}P_2^1Y_{2,-1}&&&\\
\Im({\cal A}_\parallel^*(t){\cal A}_\perp(t)) & \sqrt{\frac{3}{5}}P_2^2Y_{2,-2}&&&\\
|{\cal A}_{s}(t)|^2                   &  2 P_0^0 Y_{00} & -2 \sqrt{\frac{1}{5}} P_0^0 Y_{20} && \\
\Re({\cal A}_{s}^*(t){\cal A}_0(t))   &  4 \sqrt{3} P_1^0 Y_{00} & -4 \sqrt{\frac{3}{5}} P_1^0 Y_{20} && \\
\Re({\cal A}_{s}^*(t){\cal A}_\parallel(t))   &  3\sqrt{\frac{2}{5}} P_1^1 Y_{2,1} &&& \\
\Im({\cal A}_{s}^*(t){\cal A}_\perp(t))     &  -3\sqrt{\frac{2}{5}} P_1^1 Y_{2,-1} &&& \\
 \hline
\end{array}
\end{equation}

or equivalently

\begin{equation}
\begin{array}{|c|| c | c | c | c | }
 \hline
&\multicolumn{4}{|c|}{ f_i = \sum_{jklm} f_i^{jklm} P_j^k(\cos\thetaK)Y_{lm}(\cos\thetaL,\phiL) } \\
 \hline
|{\cal A}_0(t)|^2                       & f_{00}^{0000} = 2 & f_{00}^{0020} =  -\sqrt{\frac{4}{5}} & f_{00}^{2000} = 4 & f_{00}^{2020}= -\sqrt{\frac{16}{5}}   \\
|{\cal A}_\parallel(t)|^2               & f_{\parallel\parallel}^{2200}=1 &f_{\parallel\parallel}^{2220}=\sqrt{\frac{1}{20}}  & f_{\parallel\parallel}^{2222}=-\sqrt{\frac{3}{20}} & \\
|{\cal A}_\perp(t)|^2                   & f_{\perp\perp}^{2200} = 1 & f_{\perp\perp}^{2220}= \sqrt{\frac{1}{20}}  & f_{\perp\perp}^{2222} = \sqrt{\frac{3}{20}} & \\
\Im({\cal A}_\parallel^*(t){\cal A}_\perp(t)) & f_{\parallel\perp}^{222-2}=\sqrt{\frac{3}{5}}&&&\\
\Re({\cal A}_0^*(t){\cal A}_\parallel(t))     & f_{0\parallel}^{2121}= \sqrt{\frac{6}{5}}&&&\\
\Im({\cal A}_0^*(t){\cal A}_\perp(t))         & f_{0\perp}^{212-1}=-\sqrt{\frac{6}{5}}&&&\\
|{\cal A}_{s}(t)|^2                   & f_{SS}^{0000}=  2  & f_{SS}^{0020}= - \sqrt{\frac{4}{5}}  && \\
\Re({\cal A}_{s}^*(t){\cal A}_\parallel(t))   &  f_{S\parallel}^{1121}=3\sqrt{\frac{2}{5}}  &&& \\
\Im({\cal A}_{s}^*(t){\cal A}_\perp(t))     &  f_{S\perp}^{112-1}= -3\sqrt{\frac{2}{5}} &&& \\
\Re({\cal A}_{s}^*(t){\cal A}_0(t))   & f_{S0}^{1000}= 4 \sqrt{3}  & f_{S0}^{1020}= -4 \sqrt{\frac{3}{5}}  && \\
 \hline
\end{array}
\end{equation}

%The correctly normalized PDF is $\frac{9}{16 \pi}|{ \bf A}(t) \wedge \hat{n}|^2$, as was calculated before. Since we coded up the functions as shown in table \ref{tab:pdfinpy}, the correct normalization for the PDF in our code and thus for the terms in table \ref{tab:pdfinpy} is 
%\begin{equation}
%\frac{9}{16 \pi}\frac{\sqrt{4 \pi}}{9} = \frac{1}{8 \sqrt{\pi}} \quad .
%\end{equation}

\pagebreak

\section{Angular acceptance and angular acceptance correction}
We also expand a general angular acceptance function in terms of associated Legendre polynomials and spherical harmonics:
\begin{equation}
   \epsilon(\psi,\theta,\phi) = c^{i}_{jk} P_i(\cos\psi)Y_{jk}(\theta,\phi)
   \label{eq:eps_exp}
\end{equation}
Calculating the normalization integral, i.e. the integral of the product of efficiency and signal:
\begin{eqnarray}
    N(q_T,t)  &=&  \int d\cos\psi\int d\cos\theta \int d\phi \epsilon(\psi,\theta,\phi) \frac{9}{\sqrt{4\pi}}|{\bf A}(t,q_T) \wedge \hat{n}|^2 \\
      &=& c^i_{jk} \int d\cos\psi   P_i(\cos\psi)  \int d\cos\theta d\phi Y_{jk}(\theta,\phi) 
          \biggl[  \\
          &&                  +|{\cal A}_0|^2  (P_0^0(\cos\psi)+2P_2^0(\cos\psi) ) \left( 2Y_{00}+\sqrt{\frac{1}{5}}Y_{20}-\sqrt{\frac{3}{5}}Y_{22} \right) \\
          &&                   |{\cal A}_\parallel|^2 P_2^2(\cos\psi) \left( Y_{00}+\sqrt{\frac{1}{20}}Y_{20}+\sqrt{\frac{3}{20}}Y_{22}  \right)  \\
          &&                  +|{\cal A}_\perp|^2  P_2^2(\cos\psi) \left( Y_{00} - \sqrt{\frac{1}{5}}Y_{20}\right)\\
          &&                  +\Im({\cal A}_\parallel^*{\cal A}_\perp)\sqrt{\frac{3}{5}}P_2^2(\cos\psi) Y_{2,-1}  \\
          &&                  -\Re({\cal A}_0^*{\cal A}_\parallel  )  \sqrt{\frac{6}{5}}P_2^1(\cos\psi) Y_{2,-2} \\
          &&                  +\Im({\cal A}_\perp^* {\cal A}_0 )      \sqrt{\frac{6}{5}}P_2^1(\cos\psi)Y_{2,1} \\
          &&                  +|{\cal A}_{s}(t)|^2  P_0^0 (2Y_{00} + \sqrt{\frac{1}{5}}Y_{20} -\sqrt{\frac{3}{5}} Y_{22}) \\
          &&                  -\Re({\cal A}_{s}^*(t){\cal A}_\parallel(t)) 3\sqrt{\frac{2}{5}} P_1^1 Y_{2,-2}\\
          &&                  +\Im({\cal A}_{s}^*(t){\cal A}_\perp(t)) 3\sqrt{\frac{2}{5}} P_1^1 Y_{2,1}\\
          &&                  +\Re({\cal A}_{s}^*(t){\cal A}_0(t)) P_1^0( 4\sqrt{3}  Y_{00} + 2 \sqrt{\frac{3}{5}} Y_{20} - 6 \sqrt{\frac{1}{5}}Y_{22})
          \biggr] \\
      &=& \int d\cos\psi   P_i(\cos\psi) \biggl[  \\
          &&                  +|{\cal A}_0|^2 2 (\frac{1}{2}P_0^0(\cos\psi)+P_2^0(\cos\psi) ) \left( 2 c^i_{00}+\sqrt{\frac{1}{ 5}}c^i_{20}-\sqrt{\frac{3}{ 5}}c^i_{22} \right) \\
          &&                   |{\cal A}_\parallel|^2 2( P_0^0(\cos\psi) - P_2^0(\cos\psi)  )           \left( c^i_{00}  +\sqrt{\frac{1}{20}}c^i_{20}+\sqrt{\frac{3}{20}}c^i_{22}  \right)  \\
          &&                  +|{\cal A}_\perp|^2 2 ( P_0^0(\cos\psi) - P_2^0(\cos\psi)  )              \left( c^i_{00}  -\sqrt{\frac{1}{ 5}}c^i_{20}\right)\\
          &&                  +\Im({\cal A}_\parallel^*{\cal A}_\perp)2\sqrt{\frac{3}{5}} ( P_0^0(\cos\psi) - P_2^0(\cos\psi)  ) c^i_{2,-1}  \\
          &&                  -\Re({\cal A}_0^*{\cal A}_\parallel  )  \sqrt{\frac{6}{5}}P_2^1(\cos\psi) c^i_{2,-2}\\ 
          &&                  +\Im({\cal A}_0^* {\cal A}_\perp )      \sqrt{\frac{6}{5}}P_2^1(\cos\psi)c^i_{2,1} \\
          &&                  +|{\cal A}_{s}(t)|^2  P_0^0 (2c^i_{00} + \sqrt{\frac{1}{5}}c^i_{20} -\sqrt{\frac{3}{5}} c^i_{22}) \\
          &&                  -\Re({\cal A}_{s}^*(t){\cal A}_\parallel(t)) 3\sqrt{\frac{2}{5}} P_1^1 c^i_{2,-2}\\
          &&                  +\Im({\cal A}_{s}^*(t){\cal A}_\perp(t)) 3\sqrt{\frac{2}{5}} P_1^1 c^i_{2,1}\\
          &&                  +\Re({\cal A}_{s}^*(t){\cal A}_0(t)) P_1^0( 4\sqrt{3}  c^i_{00} + 2 \sqrt{\frac{3}{5}} c^i_{20} - 6 \sqrt{\frac{1}{5}}c^i_{22})
      \biggr] \\
      &=& 4\biggl[ \\
         &&                   +|{\cal A}_0|^2   \left( c^0_{00}+\frac{2}{5}c^2_{00}+\sqrt{\frac{1}{20}}(c^0_{20}+\frac{2}{5}c^2_{20})-\sqrt{\frac{3}{20}}(c^0_{22}+\frac{2}{5}c^2_{22}) \right) \\
         &&                    |{\cal A}_\parallel|^2 \left( c^0_{00}-\frac{1}{5}c^2_{00}+\sqrt{\frac{1}{20}}\left(c^0_{20}-\frac{1}{5}c^2_{20}\right)+\sqrt{\frac{3}{20}}\left(c^0_{22}-\frac{1}{5}c^2_{22}\right)  \right)  \\
         &&                   +|{\cal A}_\perp|^2    \left( c^0_{00}-\frac{1}{5}c^2_{00} - \sqrt{\frac{1}{5}}\left(c^0_{20}-\frac{1}{5}c^2_{20}\right)\right)\\
         &&                   +\Im({\cal A}_\parallel^*{\cal A}_\perp)\sqrt{\frac{3}{5}}\left(c^0_{2,-1}-\frac{1}{5}c^2_{2,-1}\right)  \\
         &&                   +\Re({\cal A}_0^*{\cal A}_\parallel)\sqrt{\frac{6}{5}}\frac{3\pi}{32}\left( +c^1_{2,-2}-\frac{1}{4}c^3_{2,-2}-\frac{5}{128}c^5_{2,-2}-\frac{7}{512}c^7_{2,-2} - \frac{105}{16384}c^9_{2,-2}+... \right) \\
         &&                   -\Im({\cal A}_\perp^* {\cal A}_0 )  \sqrt{\frac{6}{5}}\frac{3\pi}{32}\left( +c^1_{2,1} -\frac{1}{4}c^3_{2,1}- \frac{5}{128}c^5_{2,1} -\frac{7}{512}c^7_{2,1} - \frac{105}{16384}c^9_{2,1}+... \right) \\
          &&                  +|{\cal A}_{s}(t)|^2  \frac{1}{2} \left(2c^0_{00} + \sqrt{\frac{1}{5}}c^0_{20} -\sqrt{\frac{3}{5}} c^0_{22}\right) \\
          &&                  +\Re({\cal A}_{s}^*(t){\cal A}_\parallel(t)) 3 \sqrt{\frac{2}{5}}\frac{\pi}{8} \left(c^0_{2,-2}-\frac{1}{8}c^2_{2,-2}-\frac{1}{64}c^4_{2,-2}-\frac{5\pi}{1024}c^6_{2,-2}-\frac{35\pi}{16384}c^8_{2,-2}-...\right)\\
          &&                  -\Im({\cal A}_{s}^*(t){\cal A}_\perp(t)) 3\sqrt{\frac{2}{5}} \frac{\pi}{8} \left(c^0_{2,1}-\frac{1}{8}c^2_{2,1}-\frac{1}{64}c^4_{2,1}-\frac{5\pi}{1024}c^6_{2,1}-\frac{35\pi}{16384}c^8_{2,1}-...\right)\\
          &&                  +\Re({\cal A}_{s}^*(t){\cal A}_0(t)) \frac{1}{6}\left(4\sqrt{3}  c^1_{00} + 2 \sqrt{\frac{3}{5}} c^1_{20} - 6 \sqrt{\frac{1}{5}}c^1_{22}\right)
      \biggr]
\end{eqnarray}

where we have used 
\begin{eqnarray}
\int_{-1}^1 d\cos\theta \int_0^{2\pi} d\phi \;\;Y_l^m(\theta,\phi) Y_{l'}^{m'*}(\theta,\phi) &=& \delta_{ll'}\delta_{mm'}\\
\int_{-1}^1 dx  P_k^m P_l^m&=& \frac{2}{2l+1}\frac{(l+m)!}{(l-m)!}\delta_{kl} \\
P_2^2(x) &=& 3(1-x^2) = 2(P_0-P_2)\\
P_2^1(x) &=& -3x(1-x^2)^{1/2} = -3x(1-...  x^2 + ... x^4  + ...x^6 + ...) \\
&=& ...P_1 + ...P_3 + ...P_5 +...P_7 + ...\\
P_1^1(x) &=& -1(1-x^2)^{1/2} = ...P_0 + ...P_2 + ...P_4 + ...\\
\end{eqnarray}

In particular, one can explicitly calculate the integrals $\int_{-1}^{1}P_2^1P_i^0\mathrm{d}x$ where  $i = 1,3,5,...$, to find the coefficients in the $\Re({\cal A}_0^*{\cal A}_\parallel)$ and $\Im({\cal A}_\perp^* {\cal A}_0 )$ terms. The same goes for the integrals $\int_{-1}^{1}P_1^1P_i^0\mathrm{d}x$ where $i=0,2,4,...$ for the terms $\Re({\cal A}_{s}^*(t){\cal A}_\parallel(t))$ and $\Im({\cal A}_{s}^*(t){\cal A}_\perp(t))$.

At this point, we recognize the equivalence of the following combinations of Fourier coefficients of the efficiency, and the 'efficiency normalization moments', where we have taken into account the factor $\frac{1}{8\sqrt{\pi}}$ which represents the overall normalization. 

\begin{eqnarray}
    \xi_{00}                 &=& \frac{1}{2\sqrt{\pi}} \left( c^0_{00}+\frac{2}{5}c^2_{00}+\sqrt{\frac{1}{20}}(c^0_{20}+\frac{2}{5}c^2_{20})-\sqrt{\frac{3}{20}}(c^0_{22}+\frac{2}{5}c^2_{22}) \right) \\
    \xi_{\parallel\parallel} &=& \frac{1}{2\sqrt{\pi}} \left( c^0_{00}-\frac{1}{5}c^2_{00}+\sqrt{\frac{1}{20}}\left(c^0_{20}-\frac{1}{5}c^2_{20}\right)+\sqrt{\frac{3}{20}}\left(c^0_{22}-\frac{1}{5}c^2_{22}\right)  \right)  \\
    \xi_{\perp\perp}         &=& \frac{1}{2\sqrt{\pi}} \left( c^0_{00}-\frac{1}{5}c^2_{00} - \sqrt{\frac{1}{5}}\left(c^0_{20}-\frac{1}{5}c^2_{20}\right)\right)\\
    \xi_{\parallel\perp}     &=& \frac{1}{2\sqrt{\pi}} \sqrt{\frac{3}{5}}\left(c^0_{2,-1}-\frac{1}{5}c^2_{2,-1}\right)  \\
    \xi_{0\parallel}         &=& \frac{1}{2\sqrt{\pi}} \sqrt{\frac{6}{5}}\frac{3\pi}{32}\left( +c^1_{2,-2}-\frac{1}{4}c^3_{2,-2}-\frac{5}{128}c^5_{2,-2}-\frac{7}{512}c^7_{2,-2} - \frac{105}{16384}c^9_{2,-2}+... \right) \\
    \xi_{\perp 0}            &=& -\frac{1}{2\sqrt{\pi}} \sqrt{\frac{6}{5}}\frac{3\pi}{32}\left( +c^1_{2,1} -\frac{1}{4}c^3_{2,1}- \frac{5}{128}c^5_{2,1} -\frac{7}{512}c^7_{2,1} - \frac{105}{16384}c^9_{2,1}+... \right) \\
    \xi_{SS}               &=&  \frac{1}{2\sqrt{\pi}}\frac{1}{2} \left(2c^0_{00} + \sqrt{\frac{1}{5}}c^0_{20} -\sqrt{\frac{3}{5}} c^0_{22}\right) \\
    \xi_{S \parallel}        &=&  \frac{1}{2\sqrt{\pi}} 3 \sqrt{\frac{2}{5}}\frac{\pi}{8} \left(c^0_{2,-2}-\frac{1}{8}c^2_{2,-2}-\frac{1}{64}c^4_{2,-2}-\frac{5\pi}{1024}c^6_{2,-2}-\frac{35\pi}{16384}c^8_{2,-2}-...\right)\\
    \xi_{S \perp}           &=& \frac{1}{2\sqrt{\pi}} 3\sqrt{\frac{2}{5}} \frac{\pi}{8} \left(c^0_{2,1}-\frac{1}{8}c^2_{2,1}-\frac{1}{64}c^4_{2,1}-\frac{5\pi}{1024}c^6_{2,1}-\frac{35\pi}{16384}c^8_{2,1}-...\right)\\
    \xi_{S 0}              &=& \frac{1}{2\sqrt{\pi}}\frac{1}{6}\left(4\sqrt{3}  c^1_{00} + 2 \sqrt{\frac{3}{5}} c^1_{20} - 6 \sqrt{\frac{1}{5}}c^1_{22}\right)
    \end{eqnarray}

This implies that, if the efficiency is uniform (and $100\%$) in the angles, i.e.
\begin{equation}
   \epsilon(\psi,\theta,\phi) = 1 \Rightarrow c^0_{00} = 2\sqrt{\pi} ; c^{i}_{jk} = 0 (i\neq 0, j\neq 0, k \neq 0)
\end{equation}
we can read off the moments as:
\begin{eqnarray}
    \xi_{\parallel\parallel} = \xi_{00}= \xi_{\perp\perp} = \xi_{SS} = 1\\
    \xi_{\parallel\perp} =\xi_{\perp 0} = \xi_{0\parallel} =\xi_{S \parallel} = \xi_{S \perp} = \xi_{S 0} =0
\end{eqnarray}
Given that it is known that  in the likelihood fit only the (relative) normalization of the six angular
terms matters, it is clear that only a subset the Fourier components needs to be know to
perform the fit. Of course additional terms will improve the visual results when plotting 
differential distributions.

In case only the $\xi_{ab}$ are known, and an equivalent set of $c^i_{jk}$ is required, the 
following may be used:
\begin{eqnarray}
 c^0_{00}  &=& \frac{3\sqrt{\pi}}{8}( \xi_{\parallel \parallel }+\xi_{00}+\xi_{\perp\perp} )     \\
 c^2_{00}  &=& \frac{15\sqrt{\pi}}{8}( \xi_{00} - \xi_{\parallel \parallel } )    \\
 c^0_{20}  &=& \frac{3\sqrt{5\pi}}{4} ( \xi_{\parallel \parallel } - \xi_{\perp\perp} )    \\
 c^0_{2-1} &=& \frac{9\sqrt{\pi}}{8}\sqrt{\frac{5}{3}}  \xi_{\parallel \perp}    \\
 c^1_{21}  &=& - \frac{12}{\sqrt{\pi}}\sqrt{\frac{5}{6}} \xi_{\perp0}    \\
 c^1_{2-2} &=&   \frac{12}{\sqrt{\pi}}\sqrt{\frac{5}{6}}  \xi_{0\parallel }
\end{eqnarray}
Note that this choice is not unique, but will (should)  reproduce the same minimum as a fit which used
the $\xi_{ab}$.


%TODO: take into account ${\cal D} = 1-2( <w_i> + q_T \frac{\Delta w_i}{2})$, i.e. sum over $q_T$ may not fully cancel some terms...
%and the same for the tagging times reconstruction efficiency....
\pagebreak

We utilize the fact that, for a MC sample generated according to a P.D.F. $g(\vec{x})$,
and an efficience $\epsilon(\vec{x})$, that we can write a sum over accepted events as:
\begin{eqnarray*}
  \frac{1}{N_{\mathrm{gen}}} \sum_{\mathrm{accepted \;\;events}} f(\vec{x}_i) &=& 
  \frac{1}{N_{\mathrm{gen}}} \sum_{\mathrm{generated \;\;events}} \epsilon(\vec{x}_i) f(\vec{x}_i) 
  \\ &\approx&  \int g(\vec{x})dx \; \epsilon(\vec{x}) f(\vec{x})
  \end{eqnarray*}

If we now substitute the expansion of the efficiency in terms of orthonormal basis function,
i.e. equation \ref{eq:eps_exp}, then we have:

\begin{eqnarray*}
  \frac{1}{N_{\mathrm{gen}}} \sum_{\mathrm{accepted\; events}} f(\vec{\Omega}_i) 
    &=& \int g(\vec{\Omega})d\Omega\;  c^{ijk} P_i(\cos\theta_h)Y_{jk}(\cos\theta_\ell,\phi_\ell)  f(\vec{\Omega})
\end{eqnarray*}

Using the orthogonality of the basis functions, we can now determine the coefficients $c^{ijk}$ 
by  choosing $f(\vec{\Omega})$ as follows:
\begin{eqnarray*}
\frac{1}{N_{\mathrm{gen}}} \sum_{\mathrm{accepted \;events}} \frac{2i+1}{2}\frac{ P_i Y_{jk} }{ g }  &=&
    \int g d\vec{\Omega}\; c^{lmn} P_lY_{mn} \left[  \frac{2i+1}{2}\frac{ P_iY_{jk} }{ g } \right] 
\\  &=& c^{lmn} \frac{2i+1}{2} \int d\vec{\Omega}\; P_l P_i \; Y_{mn} Y_{jk}
\\  &=& c^{ijk}
\end{eqnarray*}

where

\begin{equation*}
\vec{\Omega} \equiv (\cos\theta_h,\cos\theta_\ell,\phi_\ell);\;\; \int d\vec{\Omega} \equiv \int d\cos\theta_h d\cos\theta_\ell d\phi_\ell
\end{equation*}

\end{document}



\documentclass[a4paper,9pt,twoside]{article}

\def\thetatr{\theta_\mathrm{tr}}
\def\phitr{\phi_\mathrm{tr}}
\def\psitr{\psi_\mathrm{tr}}
\def\thetaK{\theta_\mathrm{K}}
\def\thetaL{\theta_\mathrm{L}}
\def\phiL{\phi_\mathrm{L}}

\begin{document}


\begin{eqnarray}
   \lambda_f &\equiv& \frac{q}{p}\frac{\overline{\cal A}_f}{{\cal A}_f} = \eta_f\frac{q}{p} \frac{\overline{\cal A}_{\overline{f}}}{{\cal A}_f} \equiv \eta_f \lambda\\
   S &=& \frac{2 \Im \lambda}{1+|\lambda|^2};\;\; D = \frac{2 \Re \lambda}{1+|\lambda|^2};\;\; C = \frac{1-|\lambda|^2 }{1+|\lambda|^2}\\
   D^2+S^2 & = & 1 - C^2 \\
   \lambda & = & \frac{D+iS}{1+C};\;\; \frac{1}{\lambda}  =  \frac{D-iS}{1-C};\;\; |\lambda|^2 = \frac{ 1-C }{ 1+C }
\end{eqnarray}
\begin{equation}
  \eta_f = \left\{ \begin{array}{l} +1 \;\; \mathrm{if}\;\; i=0,\parallel  \\ -1\;\; \mathrm{if}\;\; i=\perp \end{array} \right. \;\;\;;\;\; \frac{1}{\eta_f}=\eta_f
  \label{eq:etadef}
\end{equation}
For $B_s\rightarrow J/\psi\phi$, assuming the Standard Model,  we have $S=\sin(2\beta_s),D=\cos(2\beta_s),C=0$.
In terms of $\phi_s$, we have $S=-\sin(\phi_s),D=\cos(\phi_s),C=0$.

\begin{eqnarray}
   {\cal A}_f(t)            &=& e^{-t/2\tau} \left[ E_+(t) + \eta_f \lambda E_-(t)  \right] a_f  \\
                            &=&     e^{-t/2\tau} \left[ E_+(t) +\eta_f \frac{D+iS}{1+C} E_-(t)  \right] a_f  \\
   {\overline{\cal A}}_f(t) &=& e^{-t/2\tau} \left[  E_+(t) + \frac{1}{\eta_f\lambda} E_-(t)  \right] \eta_f a_f    \\
                            &=&     e^{-t/2\tau} \left[\eta_f E_+(t) + \frac{D-iS}{1-C} E_-(t)  \right] a_f   
\end{eqnarray}
where we have defined that $a_f \equiv {\cal A}_f(t=0)$ and thus that, by definition, $\overline{a}_f = \eta_f a_f$. The above can be
written in a more symmetrical way as:
\begin{eqnarray}
   {\cal A}_f(t)           &=&     \frac{e^{-t/2\tau}}{1+C} \left[ (1+C)E_+(t) +\eta_f (D+iS) E_-(t)  \right] a_f  
%\\                           &=&     \frac{e^{-t/2\tau}}{1-C^2} \left[ (1-C^2)E_+(t)+\eta_f(D+iS)(1-C)E_-(t) \right] a_f\\
\\   {\overline{\cal A}}_f(t) &=&    \frac{e^{-t/2\tau}}{1-C} \left[ (1-C)E_+(t) +\eta_f (D-iS) E_-(t)  \right] \eta_f a_f  
%\\                           &=&     \frac{e^{-t/2\tau}}{1-C^2} \left[ (1-C^2)E_+(t)+\eta_f(D+iS)(1+C)E_-(t) \right] a_f
\end{eqnarray}
where we have used the properties of $\eta_f$ shown in (\ref{eq:etadef}).
%Since the overall normalzation factor $\frac{1}{1-C^2}$ does not depend on any observable, it can be dropped at this point
%(which will simplify the case when we later consider $S=D=0,C=\pm 1$!), resulting in:
%\begin{eqnarray}
%   {\cal A}_f(t)            &=&     e^{-t/2\tau} \left[ (1-C^2)E_+(t)+\eta_f(D+iS)(1-C)E_-(t) \right] a_f\\
%   {\overline{\cal A}}_f(t) &=&     e^{-t/2\tau} \left[ (1-C^2)E_+(t)+\eta_f(D+iS)(1+C)E_-(t) \right] a_f
%\end{eqnarray}
As a consequence, the expressions for $\overline{b}_\mathrm{tag}$ are related to those for $b_\mathrm{tag}$ by the following substitution:
\begin{equation}
b_\mathrm{tag} \leftrightarrow \overline{b}_\mathrm{tag} \;\;\;\Leftrightarrow\;\;\;  ( S \leftrightarrow -S, C\leftrightarrow -C, a_\perp\leftrightarrow -a_\perp ) 
\end{equation}
which can be combined in a single expression by multiplying the relevant terms with the value of flavour tag:
\begin{equation}
(S,C,a_\perp) \rightarrow (q_T S, q_T C, q_Ta_\perp) \;\; \mathrm{with}\;\; q_T = \left\{\begin{array}{l} +1 \;\;\; \mathrm{if}\; b_\mathrm{tag} \\
                                                                                   -1 \;\;\; \mathrm{if}\; \overline{b}_\mathrm{tag}
\end{array} \right.
\end{equation}

%Note that in case of $B_0\rightarrow J/\psi K^{*0}(K^+\pi^-)$, we have the
%case where $\lambda=0$, or $S=D=0,C=1$. On the other hand, $B_0\rightarrow J/\psi \overline{K^{*0}}(K^-\pi^+)$, 
%we have the case where $1/\lambda=0$, or $S=D=0,C=-1$.

%\begin{equation}
%   A(q_T=-1,q_\perp=\pm,t) =\left( \frac{1-q_T}{2}q_\perp+\frac{1+q_T}{2}\right) e^{-t/\tau} \left[ E_+(t) +q_\perp e^{2q_Ti\beta_s}E_-(t)  \right] a_f 
%\end{equation}
%\begin{eqnarray}
%   {\cal A}_f(t)            &= \frac{1}{1+C} e^{-t/2\tau} \left[+ (1+C) E_+(t) \pm (D+iS) E_-(t)  \right] a_f  &= A(q_T=+1,q_\perp,t)\\
%   {\overline{\cal A}}_f(t) &= \frac{1}{D+iS} e^{-t/2\tau} \left[\pm  (D+iS) E_+(t) + (1+C) E_-(t)  \right] a_f  &= A(q_T=-1,q_\perp,t) 
%\end{eqnarray}

Note that we have not assumed that $|\lambda|=1$, but we did assume that all three $|\lambda_f|$ are the same, i.e.
this implies that there either there is either no, or identical amount of direct CP violation in all three decay amplitudes,
i.e. $|\overline{A}_f/A_f|$ are all the same. The above does allow for CP violation in mixing, as in that case $|q/p|\ne 1$,
and all three $|\lambda_f|$ would be, by definition, affected identically. As a result, {\em assuming the absence of direct CP},
we can relate the value  of $C$ to $A_\mathrm{SL}$ as follows:
\begin{eqnarray}
 A_\mathrm{SL} &=& \frac{ |q/p|^2 - |p/q|^2 }{ |q/p|^2 + |p/q|^2 }  \\
               &=& \frac{ - 2 C } { 1 + C^2 } \approx - 2 C
\end{eqnarray}

\section{Time Evolution}

\begin{equation}
  E_\pm = \frac{1}{2} \left[ e^{+i\omega t } \pm e^{-i\omega t } \right]\;\;\;\;\mathrm{with}\;
   \omega = \frac{\Delta m}{2} + i\frac{\Delta \Gamma}{4} 
\end{equation}
\begin{eqnarray}
 | E_\pm(t) |^2  &=& \frac{1}{2}\left[ \cosh\left(\frac{\Delta\Gamma}{2}t\right) \pm \cos\left(\Delta m t \right) \right] \\
 E_+^*(t)E_-(t)  &=& \frac{1}{2}\left[ -\sinh\left(\frac{\Delta\Gamma}{2}t\right) + i \sin\left(\Delta m t \right) \right]
\end{eqnarray}

%\begin{eqnarray}
%   {\cal N}_\pm  &= & \int_0^\infty dt\; e^{-t/\tau}| E_\pm(t) |^2   
%                  =  \frac{\tau}{2}\left[ \frac{1}{1-(\Delta\Gamma \tau)^2}  \pm \frac{1}{1+\left(\Delta m \tau\right)^2}\right]
%\\ {\cal N}_{+-} &= & \int_0^\infty dt\; e^{-t/\tau} E_+^*(t)E_-(t)  
%                  =  \frac{\tau}{2}\left[ -\frac{\Delta\Gamma\tau}{1-(\Delta\Gamma\tau)^2} + i \frac{\Delta m \tau}{1+\left(\Delta m \tau \right)^2}  \right]
%\end{eqnarray}

\begin{eqnarray}
   |{\cal A}_i(t)|^2  &=& {\cal A}_i^*(t){\cal A}_i(t) \;\; (\mathrm{note: no\; summation!}) \\
%   |{\cal A}_i(t)|^2  &=& e^{-t/\tau}|a_i|^2 \left| E_+(t) +\eta_{i}\frac{D+iS}{1+C} E_-(t)  \right|^2  \\
%                      &=& \frac{e^{-t/\tau}|a_i|^2}{1+C} \left[ (1+C)|E_+(t)|^2 + (1-C)| E_-(t)|^2 + 2 \eta_{i}\left( D\Re\left(E_+^*(t)E_-(t)\right)-S\Im\left(E_+^*(t)E_-(t)\right)\right)  \right] \\
%                      &=& \frac{e^{-t/\tau}|a_i|^2}{1+C} \left[ \cosh\left(\frac{\Delta\Gamma}{2}t\right)
%                                                              + C\cos(\Delta m t ) 
%                                                              - \eta_{i} D\sinh\left(\frac{\Delta\Gamma}{2}t\right)
%                                                              - \eta_{i} S\sin \left(\Delta m t \right)  
%                                                         \right] \\
%   |{\cal\overline{A}}_i(t)|^2 % &=& e^{-t/\tau}|a_i|^2 \left| \eta_i E_+(t) +\frac{D-iS}{1-C} E_-(t)  \right|^2  \\
%                      &=& \frac{e^{-t/\tau}|a_i|^2}{1-C} \left[ \cosh\left(\frac{\Delta\Gamma}{2}t\right)
%                                                              - C\cos(\Delta m t ) 
%                                                              - \eta_{i} D\sinh\left(\frac{\Delta\Gamma}{2}t\right)
%                                                              + \eta_{i} S\sin \left(\Delta m t \right)  
%                                                         \right] \\
  {\cal A}_i^*(t) {\cal A}_j(t) &= \frac{ a_i^* a_j e^{-t/\tau}} {1+C}&\left[ (1+C) |E_+(t)|^2  +\eta_i\eta_j (1-C)|E_-(t)|^2 + \eta_j(D+iS)E_+^*E_-+\eta_i(D-iS)E_+E_-^*   \right] \\
                                &= \frac{ a_i^* a_j e^{-t/\tau}} {1+C}&\Biggl[ \left( \frac{1+\eta_i\eta_j}{2} +  \frac{1-\eta_i\eta_j}{2}C\right) \cosh\left(\frac{\Delta\Gamma}{2}t\right) \\
                                                                          &&  +\left( \frac{1-\eta_i\eta_j}{2} +  \frac{1+\eta_i\eta_j}{2}C\right) \cos(\Delta m t) \\
                                                                          &&  +\left(-\frac{\eta_i+\eta_j}{2}D + \frac{\eta_i-\eta_j}{2}iS\right) \sinh\left(\frac{\Delta\Gamma}{2}t\right) \\
                                                                          &&  +\left(-\frac{\eta_i+\eta_j}{2}S - \frac{\eta_i-\eta_j}{2}iD\right) \sin(\Delta m t) \Biggr] 
\end{eqnarray}

The result can be summarized as follows (where we have used that $q_T^2=1$):
\begin{equation}
\begin{array}{|c|| c | c | c | c | c |}
 \hline
                                             &                          &  \cosh\left(\frac{\Delta\Gamma}{2}t\right) & q_T \cos(\Delta m t)  & \sinh\left(\frac{\Delta\Gamma}{2}t\right) &q_T\sin(\Delta m t)  \\
 \hline
|{\cal A}_0(t)|^2                           &\frac{  | a_0 |^2 e^{-t/\tau}}{1+q_TC}                  & 1 &   C  &  -D & -S   \\
|{\cal A}_\parallel(t)|^2                   &\frac{  | a_\parallel |^2 e^{-t/\tau}}{1+q_TC}          & 1 &   C  &  -D & -S   \\
|{\cal A}_\perp(t)|^2                       &\frac{  | a_\perp |^2 e^{-t/\tau}}{1+q_TC}              & 1 &   C  &  +D & +S   \\
\Im({\cal A}_\parallel^*(t){\cal A}_\perp(t)) &\frac{  \Re(a_\parallel^* a_\perp)e^{-t/\tau}}{1+q_TC}& 0 &   0  &   S & -D   \\                         
                                            &\frac{  \Im(a_\parallel^* a_\perp)e^{-t/\tau}}{1+q_TC}  & C &   1  &   0 &  0   \\
\Im({\cal A}_0^*(t){\cal A}_\perp(t))         &\frac{  \Re(a_0^* a_\perp)e^{-t/\tau}}{1+q_TC}        & 0 &   0  &   S & -D   \\
                                            &\frac{  \Im(a_0^* a_\perp)e^{-t/\tau}}{1+q_TC}          & C &   1  &   0 &  0   \\
\Re({\cal A}_0^*(t){\cal A}_\parallel(t))     &\frac{  \Re(a_0^*a_\parallel)e^{-t/\tau}}{1+q_TC}     & 1 &   C  &  -D & -S   \\
                                            &\frac{  \Im(a_0^*a_\parallel)e^{-t/\tau}}{1+q_TC}       & 0 &   0  &   0 &  0   \\
\hline
\end{array}
\end{equation}

If we introduce the probability of tagging an (initial) $B$ meson as $\overline{B}$ as $w$, and
the probability to tag an initial $\overline{B}$ meson as $B$ as $\overline{w}$, and the probability
(efficiency) to tag an $B$ ($\overline{B}$) meson in the $i^\mathrm{th}$ tagging category as $t_i$ ($\overline{t}_i$), then the
PDF for candidates, tagged as $B$ respectively $\overline{B}$, in tagging category $i$ will be:
\begin{eqnarray}
   P_{B_\mathrm{tag},i} &\equiv t_i (1-w_i) P_{B} + \overline{t}_i\overline{w}_i P_{\overline{ B} }\\
   P_{\overline{B}_\mathrm{tag},i} &\equiv \overline{t}_i (1-\overline{w}_i) P_{\overline{B}} + t_i w_i P_{ B }
\end{eqnarray} 
In addition, there will be those events which are untagged, which will be distributed as:
\begin{equation}
   P_{\mathrm{untagged}} \equiv \left( 1-\sum_i \overline{t}_i \right)  P_{\overline{B}} + \left( 1 - \sum_i t_i \right)  P_{ B }
\end{equation}
where we have assumed that the {\em combination} of $P_{B}+P_{\overline{B}}$ is properly normalized.
Taking into account that $P_{B}$ and $P_{\overline{B}}$ differ in normalization by
$N_+=1+C$ and $N_-=1-C$, and labelling for the time being the CP even 
resp. CP odd terms in $P_{B}$ and $P_{\overline{B}}$ as ${\cal E}$
and ${\cal O}$, we find that:
\begin{eqnarray}
   P(q_T=1,i) &=& t_i(1-w_i)\frac{{\cal E} + {\cal O}}{N_+} + \overline{t}_i\overline{w}_i\frac{{\cal E}-{\cal O}}{N_-}  \\
   P(q_T=-1,i) &=& \overline{t}_i(1-\overline{w}_i)\frac{{\cal E} - {\cal O}}{N_-} + t_i w_i\frac{{\cal E}+{\cal O}}{N_+} 
   \\ P(q_T=0) &=& \left(1-\sum_i t_i\right)  \frac{ { \cal E } + {\cal O} }{N_+} + \left(1-\sum_i \overline{t}_i\right)  \frac{ { \cal E} - {\cal O} } { N_- }
%   \\
%     &=& \frac{  \left[ 1-{\cal D}({\cal W}+C) \right]{\cal E}
%                               +\left[ {\cal D}(1 + {\cal W} C) - C \right] {\cal O}
%                     }{1-C^2} \\
%   P(q_T=-1) &=& \frac{  \left[ 1+{\cal D}({\cal W} + C)  \right]{\cal E}
%                               -\left[ {\cal D}(1+{\cal W}C) +  C \right] {\cal O}
%                        }{1-C^2}\ \\
\end{eqnarray} 
In order to restore some symmetry to these equations, and to be able to easily reduce these
equations to the familiar ones where $t_i=\overline{t}_i$ and $w_i=\overline{w}_i$, we introduce 
the following parameterization:
\begin{eqnarray}
\frac{1}{N_\pm} &=& \frac{1}{N}\left( 1 \pm \nu \right) \\
&w_i            = \frac{1}{2}\left[1-{\cal D}_i\left(1-{\cal W}_i\right) \right]; &  {\cal D}_i =1-w_i - \overline{w}_i \\
&\overline{w}_i = \frac{1}{2}\left[1-{\cal D}_i\left(1+{\cal W}_i\right) \right]; &  {\cal W}_i = \frac{ w_i - \overline{w}_i}{1-w_i-\overline{w}_i}\\
&t_i            = \epsilon_i(1+\delta_i);                              &  \epsilon_i = \frac{t_i + \overline{t}_i}{2}\\
&\overline{t}_i = \epsilon_i(1-\delta_i);                              &  \delta_i = \frac{t_i - \overline{t}_i}{t_i+\overline{t}_i}
\end{eqnarray}
Note that under $(w\rightarrow 1-w_i, \overline{w}_i\rightarrow 1-\overline{w}_i)$ we have $({\cal D}_i \rightarrow -{\cal D}_i,{\cal W}_i\rightarrow{\cal W}_i)$,
and that the quantities $\nu,\delta_i,{\cal W}_i$ are CP odd, whereas $N,\epsilon_i,{\cal D}_i$ are CP even.

\begin{eqnarray}
   P(q_T=1,i)  &=&\frac{1}{2} \epsilon_i(1+\nu)(1+\delta_i)\left(1+{\cal D}_i(1-{\cal W}_i)\right)\left({\cal E}+{\cal O}\right)  
\\             &+&\frac{1}{2} \epsilon_i(1-\nu)(1-\delta_i)\left(1-{\cal D}_i(1+{\cal W}_i)\right)\left({\cal E}-{\cal O}\right)  
\\ P(q_T=-1,i) &=&\frac{1}{2} \epsilon_i(1-\nu)(1-\delta_i)\left(1+{\cal D}_i(1+{\cal W}_i)\right)\left({\cal E}-{\cal O}\right) 
\\             &+&\frac{1}{2} \epsilon_i(1+\nu)(1+\delta_i)\left(1-{\cal D}_i(1-{\cal W}_i)\right)\left({\cal E}+{\cal O}\right)
\end{eqnarray}
where we have eliminated the overall normalization constant $N$ which is shared by each term. 
Combining the factors in front of ${\cal E}$ 
and ${\cal O}$ results after some tedious algebra in:
\begin{eqnarray}
     P(q_T=1,i)  
           &=& \epsilon_i\left[ (1+\delta_i\nu)  (1-{\cal D}_i {\cal W}_i)  +  (\delta_i+\nu)     {\cal D}_i       \right]{\cal E}
\\         &+& \epsilon_i\left[ (1+\delta_i\nu)     {\cal D}_i              +  (\delta_i+\nu)  (1-{\cal D}_i{\cal W}_i ) \right]{\cal O}
\\   P(q_T=-1,i) 
           &=& \epsilon_i\left[ (1+\delta_i\nu)  (1+{\cal D}_i {\cal W}_i)  -  (\delta_i+\nu)     {\cal D}_i       \right]{\cal E}
\\         &-& \epsilon_i\left[ (1+\delta_i\nu)     {\cal D}_i              -  (\delta_i+\nu)  (1+{\cal D}_i{\cal W}_i ) \right]{\cal O}
\\   P(q_T=0)    &=&
\end{eqnarray}
If instead we collect all CP even and CP odd terms together, we find:
\begin{eqnarray}
     P(q_T,i)  &=& \epsilon_i\left[ (1+\delta_i\nu){\cal E} + (\delta_i+\nu){\cal O} \right]
\label{eq:substrule_s}
\\             &+& \epsilon_i q_T {\cal D}_i \left[ \left( \delta_i + \nu -{\cal W}_i(1+\delta_i\nu) ]\right) { \cal E} + \left( 1+\delta_i\nu - {\cal W}_i(\delta_i+\nu) \right){\cal O}  \right]
\label{eq:substrule_e}
\end{eqnarray}
and the CP asymmetry can be written as:
\begin{equation}
  A_{CP}(i) =  {\cal D}_i \frac{\left( 1+\delta_i\nu - {\cal W}_i(\delta_i+\nu) \right){\cal O}+(\nu+\delta_i-{\cal W}_i\left(1+\delta_i\nu\right) ){\cal E}}
                               {(1+\delta_i\nu){\cal E}+(\nu+\delta_i){\cal O} }
\end{equation}

In the limit where $\nu=\delta_i={\cal W}=0$ this reduces to the expected
\begin{eqnarray}
   P(q_T,i) &=& \epsilon_i\left[  {\cal E} +q_T {\cal D} {\cal O}\right]  \\
   A_{CP}(i) &=&  {\cal D}_i\frac{\cal O }{\cal E}
\end{eqnarray}

As $\nu$ and $\delta_i$ always appear as sum (if we neglect the quadratic terms $\nu\delta_i$), and thus $\nu$ will be fully anticorrelated to the ($\epsilon_i$ weighted) average of $\delta_i$,
it becomes impossible to disentangle the difference between differences in tagging efficiency for $b$ and $\overline{b}$
and a change in the relative normalization. However, when taken into account the fact that the difference in normalization
is due to the apearance of $1/(1+C)$ vs. $1/(1-C)$, i.e. $\nu = -C$, and $C$ also appears as coefficienct in front of the $\cosh\left(\Delta\Gamma t/2\right)$
and $q_T\cos(\Delta m t)$ terms, it is expected that there is some residual power left to disentangle the two effects -- i.e. 
solely due to the time-dependent shape of the distributions, and no longer in their relative normalization, i.e. in the time-integrated yields.


Note that these mistag, tagging efficiency and normalization differences result in a fairly 
straightforward substitution rule, which one can read off from Equations (\ref{eq:substrule_s} - \ref{eq:substrule_e}):
\begin{enumerate}
\item
the initially CP-even terms (i.e. the $\cosh(\Delta\Gamma t/2)$ and $\sinh(\Delta\Gamma t/2)$ terms) obtain an additional
factor 
\begin{equation}
   {\cal E} \rightarrow {\cal E} \left[ (1+\delta_i\nu)+ q_T {\cal D}_i \left( \delta_i + \nu -{\cal W}_i(1+\delta_i\nu) ]\right) \right]
\end{equation}
\item
the initially CP-odd terms (i.e. the $\cos(\Delta m t) $ and $\sin(\Delta m t)$ terms) obtain, { \em instead of the
usual factor} $q_T{\cal D}_i$, the factor
\begin{equation}
   {\cal O} \rightarrow {\cal O} \left[ (\delta_i+\nu) +  q_T {\cal D}_i  \left( 1+\delta_i\nu - {\cal W}_i(\delta_i+\nu) \right) \right]
\end{equation}
\end{enumerate}
Again, in the limit that $\delta_i=\nu={\cal W}_i=0$, this results in the well-known expected behaviour.

\vfill
\pagebreak

\section{Angular Dependence}

If we write the direction of the positive muon (in which reference frame??) in terms of transversity angles,
\begin{equation}
  \hat{n} = ( \sin\thetatr\cos\phitr, \sin\thetatr\sin\phitr, \cos\thetatr )
\end{equation}
or helicity angles
\begin{equation}
  \hat{n} = ( -\cos\thetaL, -\sin\thetaL\cos\phiL, \sin\thetaL\sin\phiL )
\end{equation}
%%% TODO: can we trivially connect this to helicity angles at this point??
%%% i.e. can we write \hat{n} in terms of helicity angles???
and define transversity amplitudes
\begin{eqnarray}
   {\bf A}(t)            &=&  ( {\cal A}_0(t) \cos\psi          , -\frac{{\cal A}_\parallel(t) \sin\psi}{\sqrt{2}}         ,i\frac{{\cal A}_\perp(t) \sin\psi}{\sqrt{2}}          )  % \\
%   {\overline{\bf A}}(t) &=&  ( \overline{\cal A}_0(t) \cos\psi), -\frac{\overline{\cal A}_\parallel(t) \sin\psi}{\sqrt{2}},i\frac{\overline{\cal A}_\perp(t) \sin\psi}{\sqrt{2}} )
\end{eqnarray}
then one can write
\begin{eqnarray}
  { \bf A}(t) \wedge \hat{n} &=& \left|  \begin{array}{ccc}  
                                        \hat{e}_x & \hat{e}_y & \hat{e}_z \\
                                        {\cal A}_0(t) \cos\psi & - \frac{{\cal A}_\parallel(t) \sin\psi }{\sqrt{2}} & i\frac{ {\cal A}_\perp(t)\sin\psi}{\sqrt{2}}\\
                                        \sin\theta\cos\phi & \sin\theta\sin\phi & \cos\theta 
                                    \end{array}\right|  \\
                          &=& \left(\begin{array}{c}
                                  \frac{-1}{\sqrt{2}} {\cal A}_\parallel(t) \sin\psi \cos\theta  -  \frac{i }{\sqrt{2}}{\cal A}_\perp(t)\sin\psi\sin\theta\sin\phi  \\
                                   -{\cal A}_0(t) \cos\psi\cos\theta  + \frac{i}{\sqrt{2}} {\cal A}_\perp(t)\sin\psi \sin\theta\cos\phi \\
                                   {\cal A}_0(t) \cos\psi \sin\theta\sin\phi       + \frac{1}{\sqrt{2}}{\cal A}_\parallel(t) \sin\psi \sin\theta\cos\phi 
                              \end{array} \right)
\end{eqnarray}

\begin{eqnarray}
  |{ \bf A}(t) \wedge \hat{n}|^2 &=& 
                                  \frac{1}{2} |{\cal A}_\parallel(t)|^2 \sin^2\psi \cos^2\theta  +  \frac{1 }{2}|{\cal A}_\perp(t)|^2\sin^2\psi\sin^2\theta\sin^2\phi  \\
                              &&+     -\Im({\cal A}_\parallel^*{\cal A}_\perp)\sin^2\psi\cos\theta\sin\theta\sin\phi  \\
                              &+&     |{\cal A}_0|^2 \cos^2\psi\cos^2\theta  + \frac{1}{2} |{\cal A}_\perp(t)|^2\sin^2\psi \sin^2\theta\cos^2\phi \\
                              &&+     -\sqrt{2}\Im({\cal A}_\perp^* {\cal A}_0 )\cos\psi\cos\theta\sin\psi\sin\theta\cos\phi \\
                              &+&     |{\cal A}_0|^2 \cos^2\psi\sin^2\theta\sin^2\phi  + \frac{1}{2}{|\cal A}_\parallel(t)|^2 \sin^2\psi \sin^2\theta\cos^2\phi       \\
                              &&+     \sqrt{2}\Re({\cal A}_0^*{\cal A}_\parallel  )\cos\psi\sin\psi\sin^2\theta\sin\phi\cos\phi\\
                              &=&  |{\cal A}_0(t)|^2 \cos^2\psi\left( \cos^2\theta  +  \sin^2\theta\sin^2\phi\right) \\
                              &+& \frac{1}{2}|{\cal A}_\parallel(t)|^2 \sin^2\psi \left( \cos^2\theta + \sin^2\theta\cos^2\phi    \right)  \\
                              &+&\frac{1}{2}|{\cal A}_\perp(t)|^2\sin^2\psi\sin^2\theta \\
                              &+&     -\Im({\cal A}_\parallel^*(t){\cal A}_\perp(t))\sin^2\psi\cos\theta\sin\theta\sin\phi  \\
                              &+&     -\sqrt{2}\Im({\cal A}_\perp^*(t) {\cal A}_0(t) )\cos\psi\cos\theta\sin\psi\sin\theta\cos\phi \\
                              &+&     \sqrt{2}\Re({\cal A}_0^*(t){\cal A}_\parallel(t)  )\cos\psi\sin\psi\sin^2\theta\sin\phi\cos\phi\\
                              &=&\frac{1}{2}|{\cal A}_\parallel(t)|^2 \sin^2\psi \left( 1 -  \sin^2\theta\sin^2\phi    \right)  \\
                              &+&\frac{1}{2}|{\cal A}_\perp(t)|^2\sin^2\psi\sin^2\theta \\
                              &+&   |{\cal A}_0(t)|^2 \cos^2\psi\left( 1  -  \sin^2\theta\cos^2\phi\right) \\
                              &+&     -\frac{1}{2}\Im({\cal A}_\parallel^*(t){\cal A}_\perp(t))\sin^2\psi\sin(2\theta)\sin\phi  \\
                              &+&     \frac{1}{2\sqrt{2}}\Im({\cal A}_0^*(t) {\cal A}_\perp(t) ) \sin(2\psi)\sin(2\theta)\cos\phi \\
                              &+&     \frac{1}{2\sqrt{2}}\Re({\cal A}_0^*(t) {\cal A}_\parallel(t))\sin(2\psi)\sin^2\theta\sin(2\phi)
\end{eqnarray}


\begin{eqnarray}
P_l(x) &=& \frac{1}{2^l l!} \frac{d^l}{dx^l}(x^2-1)^l \\
P_l^m(x) &=& \frac{(-1)^m}{2^l l!} (1-x^2)^{\frac{1}{2}m} \frac{d^{l+m}}{dx^{l+m}} (x^2-1)^l\\
P_l^{-m}(x) & = & (-1)^m \frac{(l-m)!}{(l+m)!}P_l^m(x) \\
Y_l^m (\theta,\phi) &=& N_{lm} P_l^m(\cos\theta)e^{im\phi}\;\;\mathrm{with}\;N_{lm} =\sqrt{ \frac{2l+1}{4\pi}\frac{(l-m)!}{(l+m)!} } \\
Y_l^{-m}(\theta,\phi) &=& (-1)^m N_{lm}P_l^m(\cos\theta) e^{-im\phi}
\end{eqnarray}

\begin{eqnarray}
Y_{lm}(\theta,\phi) & = & \left\{ \begin{array}{cl} 
                                             Y_l^0(\theta,\phi) & (m=0) \\
                                             \frac{1}{\sqrt{2}} \left( Y_l^m + (-1)^m Y_l^{-m}\right)  & (m>0) \\
                                             \frac{1}{i\sqrt{2}}\left( Y_l^{|m|}-(-1)^{|m|}Y_l^{-|m|} \right) & (m<0)
                                  \end{array}\right.\\
                    & = & \left\{ \begin{array}{cl} 
                                             N_{l0} P_l^0(\cos\theta) & (m=0) \\
                                             \sqrt{2}N_{lm}P_l^m(\cos\theta)\cos(m\phi) & (m>0) \\
                                             \sqrt{2}N_{l|m|} P_l^{|m|}(\cos\theta)\sin(|m|\phi)& (m<0)
                                  \end{array}\right.
\end{eqnarray}

\begin{eqnarray}
  1                       &=& \sqrt{ \frac{4\pi}{9} } \left( 3 Y_{00}(\cos\theta,\phi) \right) \\
  \cos^2\theta            &=& \sqrt{ \frac{4\pi}{9} } \left( Y_{00}(\cos\theta,\phi) + \sqrt{\frac{4}{5}}Y_{20}(\cos\theta,\phi)\right)\\
  \sin^2\theta            &=& \sqrt{ \frac{4\pi}{9} } \left( 2Y_{00}(\cos\theta,\phi) - \sqrt{\frac{4}{5}}Y_{20}(\cos\theta,\phi)\right)\\
  \sin^2\theta \cos^2\phi &=& \sqrt{ \frac{4\pi}{9} } \left( Y_{00}(\cos\theta,\phi) - \sqrt{\frac{1}{5} }Y_{20}(\cos\theta,\phi) +\sqrt{\frac{3}{5}} Y_{2,2}(\cos\theta,\phi)\right) \\
  \sin^2\theta \sin^2\phi &=& \sqrt{ \frac{4\pi}{9} } \left( Y_{00}(\cos\theta,\phi) - \sqrt{\frac{1}{5} }Y_{20}(\cos\theta,\phi) -\sqrt{\frac{3}{5}} Y_{2,2}(\cos\theta,\phi)\right) \\
  \sin^2\theta\cos\phi\sin\phi &=& \sqrt{ \frac{4\pi}{9}} \left(\sqrt{\frac{3}{5}} Y_{2,-2}(\cos\theta,\phi) \right) \\
  \sin\theta\cos\theta\cos\phi &=& \sqrt{ \frac{4\pi}{9}}\left( -\sqrt{\frac{3}{5}}Y_{2,1}(\cos\theta,\phi)\right) \\ 
  \sin\theta\cos\theta\sin\phi &=& \sqrt{ \frac{4\pi}{9}}\left( -\sqrt{\frac{3}{5}}Y_{2,-1}(\cos\theta,\phi) \right) \\
  \sin\psi\cos\psi &=& -\frac{1}{3} P_2^1(\cos\psi) \\
  \sin^2 \psi      &=&  \frac{1}{3} P_2^2(\cos\psi) \\
  \cos^2 \psi      &=&  \frac{1}{3} \left( P_0^0(\cos\psi)+2P_2^0(\cos\psi) \right)
\end{eqnarray}

\begin{eqnarray}
\frac{9 }{\sqrt{4\pi}} |{ \bf A}(t) \wedge \hat{n}|^2 
                              &=&|{\cal A}_0|^2  (P_0^0+2P_2^0 ) \left( 2 Y_{00}+\sqrt{\frac{1}{5}}Y_{20}-\sqrt{\frac{3}{5}}Y_{22} \right) \\
                              &+&|{\cal A}_\parallel|^2 P_2^2 \left( Y_{00}+\sqrt{\frac{1}{20}}Y_{20}+\sqrt{\frac{3}{20}}Y_{22}  \right)  \\
                              &+&|{\cal A}_\perp|^2  P_2^2 \left( Y_{00} - \sqrt{\frac{1}{5}}Y_{20}\right)\\
                              &+&\Im({\cal A}_\parallel^*{\cal A}_\perp) \sqrt{\frac{3}{5}}P_2^2 Y_{2,-1}  \\
                              &+&\Im({\cal A}_0^* {\cal A}_\perp )       \sqrt{\frac{6}{5}}P_2^1Y_{2,1} \\
                              &-&\Re({\cal A}_0^*{\cal A}_\parallel  )   \sqrt{\frac{6}{5}}P_2^1 Y_{2,-2} 
\end{eqnarray}

\begin{equation}
\begin{array}{|c|| c | c | c | c | c | c |}
 \hline
&\multicolumn{6}{|c|}{ P_i^j(\cos\psitr)Y_{lm}(\cos\thetatr,\phitr) \times \frac{9}{\sqrt{4\pi}} } \\
 \hline
|{\cal A}_0(t)|^2                       &  2 P_0^0Y_{00}  & \sqrt{\frac{1}{5}} P_0^0 Y_{20} & -\sqrt{\frac{3}{5}} P_0^0 Y_{22}
                                        &  4 P_2^0Y_{00}  & \sqrt{\frac{4}{5}} P_2^0 Y_{20} & -\sqrt{\frac{12}{5}} P_2^0 Y_{22} \\
|{\cal A}_\parallel(t)|^2               &    P_2^2Y_{00} &  \sqrt{\frac{1}{20}} P_2^2 Y_{20} & +\sqrt{\frac{3}{20}} P_2^2Y_{22} & && \\
|{\cal A}_\perp(t)|^2                   &  P_2^2 Y_{00} & -\sqrt{\frac{1}{5}} P_2^2 Y_{20} &                                 &&& \\
\Im({\cal A}_\parallel^*(t){\cal A}_\perp(t)) &  \sqrt{\frac{3}{5}} P_2^2 Y_{2,-1} &&&&&\\
\Im({\cal A}_0^*(t){\cal A}_\perp(t))         &  \sqrt{\frac{6}{5}} P_2^1 Y_{2,1} &&&&&\\
\Re({\cal A}_0^*(t){\cal A}_\parallel(t))     & -\sqrt{\frac{6}{5}} P_2^1 Y_{2,-2} &&&&&\\
 \hline
\end{array}
\end{equation}
\begin{equation}
\begin{array}{|c|| c | c | c | c | }
 \hline
&\multicolumn{4}{|c|}{ P_i^j(\cos\thetaK)Y_{lm}(\cos\thetaL,\phiL) \times \frac{9}{\sqrt{4\pi}} } \\
 \hline
|{\cal A}_0(t)|^2                       & 2P_0^0Y_{00} & -\sqrt{\frac{4}{5}}P_0^0Y_{20} & 4P_2^0Y_{00} & -\sqrt{\frac{16}{5}} P_2^0Y_{20}  \\
|{\cal A}_\parallel(t)|^2               & P_2^2Y_{00} &\sqrt{\frac{1}{20}} P_2^2Y_{20} &-\sqrt{\frac{3}{20}}P_2^2Y_{22} & \\
|{\cal A}_\perp(t)|^2                   & P_2^2Y_{00} & \sqrt{\frac{1}{20}} P_2^2Y_{20} &\sqrt{\frac{3}{20}}P_2^2Y_{22} & \\
\Im({\cal A}_\parallel^*(t){\cal A}_\perp(t)) & \sqrt{\frac{3}{5}}P_2^2Y_{2,-2}&&&\\
\Im({\cal A}_0^*(t){\cal A}_\perp(t))         & -\sqrt{\frac{6}{5}}P_2^1Y_{2,-1}&&&\\
\Re({\cal A}_0^*(t){\cal A}_\parallel(t))     & \sqrt{\frac{6}{5}}P_2^1Y_{21}&&&\\
 \hline
\end{array}
\end{equation}



\pagebreak
\begin{equation}
   \epsilon(\psi,\theta,\phi) = c^{i}_{jk} P_i(\cos\psi)Y_{jk}(\theta,\phi)
   \label{eq:eps_exp}
\end{equation}

\begin{eqnarray}
    N(q_T,t)  &=&  \int d\cos\psi\int d\cos\theta \int d\phi \epsilon(\psi,\theta,\phi) |{\bf A}(t,q_T) \wedge \hat{n}|^2 \\
      &=& c^i_{jk}\frac{\sqrt{4\pi}}{9} \int d\cos\psi   P_i(\cos\psi)  \int d\cos\theta d\phi Y_{jk}(\theta,\phi) 
          \biggl[  \\
          &&                   |{\cal A}_\parallel|^2 P_2^2(\cos\psi) \left( Y_{00}+\sqrt{\frac{1}{20}}Y_{20}+\sqrt{\frac{3}{20}}Y_{22}  \right)  \\
          &&                  +|{\cal A}_0|^2  (P_0^0(\cos\psi)+2P_2^0(\cos\psi) ) \left( 2Y_{00}+\sqrt{\frac{1}{5}}Y_{20}-\sqrt{\frac{3}{5}}Y_{22} \right) \\
          &&                  +|{\cal A}_\perp|^2  P_2^2(\cos\psi) \left( Y_{00} - \sqrt{\frac{1}{5}}Y_{20}\right)\\
          &&                  +\Im({\cal A}_\parallel^*{\cal A}_\perp)\sqrt{\frac{3}{5}}P_2^2(\cos\psi) Y_{2,-1}  \\
          &&                  +\Im({\cal A}_\perp^* {\cal A}_0 )      \sqrt{\frac{6}{5}}P_2^1(\cos\psi)Y_{2,1} \\
          &&                  -\Re({\cal A}_0^*{\cal A}_\parallel  )  \sqrt{\frac{6}{5}}P_2^1(\cos\psi) Y_{2,-2} 
          \biggr] \\
      &=& \frac{\sqrt{4\pi}}{9}  \int d\cos\psi   P_i(\cos\psi) \biggl[  \\
          &&                   |{\cal A}_\parallel|^2 2( P_0^0(\cos\psi) - P_2^0(\cos\psi)  )           \left( c^i_{00}  +\sqrt{\frac{1}{20}}c^i_{20}+\sqrt{\frac{3}{20}}c^i_{22}  \right)  \\
          &&                  +|{\cal A}_0|^2 2 (\frac{1}{2}P_0^0(\cos\psi)+P_2^0(\cos\psi) ) \left( 2 c^i_{00}+\sqrt{\frac{1}{ 5}}c^i_{20}-\sqrt{\frac{3}{ 5}}c^i_{22} \right) \\
          &&                  +|{\cal A}_\perp|^2 2 ( P_0^0(\cos\psi) - P_2^0(\cos\psi)  )              \left( c^i_{00}  -\sqrt{\frac{1}{ 5}}c^i_{20}\right)\\
          &&                  +\Im({\cal A}_\parallel^*{\cal A}_\perp)2\sqrt{\frac{3}{5}} ( P_0^0(\cos\psi) - P_2^0(\cos\psi)  ) c^i_{2,-1}  \\
          &&                  +\Im({\cal A}_0^* {\cal A}_\perp )      \sqrt{\frac{6}{5}}P_2^1(\cos\psi)c^i_{2,1} \\
          &&                  -\Re({\cal A}_0^*{\cal A}_\parallel  )  \sqrt{\frac{6}{5}}P_2^1(\cos\psi) c^i_{2,-2} 
      \biggr] \\
      &=& \frac{8\sqrt{\pi}}{9}  \biggl[ \\
         &&                    |{\cal A}_\parallel|^2 \left( c^0_{00}-\frac{1}{5}c^2_{00}+\sqrt{\frac{1}{20}}\left(c^0_{20}-\frac{1}{5}c^2_{20}\right)+\sqrt{\frac{3}{20}}\left(c^0_{22}-\frac{1}{5}c^2_{22}\right)  \right)  \\
         &&                   +|{\cal A}_0|^2   \left( c^0_{00}+\frac{2}{5}c^2_{00}+\sqrt{\frac{1}{20}}(c^0_{20}+\frac{2}{5}c^2_{20})-\sqrt{\frac{3}{20}}(c^0_{22}+\frac{2}{5}c^2_{22}) \right) \\
         &&                   +|{\cal A}_\perp|^2    \left( c^0_{00}-\frac{1}{5}c^2_{00} - \sqrt{\frac{1}{5}}\left(c^0_{20}-\frac{1}{5}c^2_{20}\right)\right)\\
         &&                   +\Im({\cal A}_\parallel^*{\cal A}_\perp)\sqrt{\frac{3}{5}}\left(c^0_{2,-1}-\frac{1}{5}c^2_{2,-1}\right)  \\
         &&                   -\Im({\cal A}_\perp^* {\cal A}_0 )  \sqrt{\frac{6}{5}}\frac{3\pi}{32}\left( +c^1_{2,1} -\frac{1}{4}c^3_{2,1}- \frac{5}{128}c^5_{2,1} -\frac{7}{512}c^7_{2,1} - \frac{105}{16384}c^9_{2,1}+... \right) \\
         &&                   +\Re({\cal A}_0^*{\cal A}_\parallel)\sqrt{\frac{6}{5}}\frac{3\pi}{32}\left( +c^1_{2,-2}-\frac{1}{4}c^3_{2,-2}-\frac{5}{128}c^5_{2,-2}-\frac{7}{512}c^7_{2,-2} - \frac{105}{16384}c^9_{2,-2}+... \right) \\
      \biggr]
\end{eqnarray}

where we have used 
\begin{eqnarray}
\int_{-1}^1 d\cos\theta \int_0^{2\pi} d\phi \;\;Y_l^m(\theta,\phi) Y_{l'}^{m'*}(\theta,\phi) &=& \delta_{ll'}\delta_{mm'}\\
\int_{-1}^1 dx  P_k^m P_l^m&=& \frac{2}{2l+1}\frac{(l+m)!}{(l-m)!}\delta_{kl} \\
P_2^1(x) = -3x(1-x^2)^{1/2} &=& -3x(1-...  x^2 + ... x^4  + ...x^6 + ...x^8  + ... ) \\
&=& ...P_1 + ...P_3 + ...P_5 +...P_7 + ...
\end{eqnarray}

At this point, we recognize the equivalence of the following combinations of Fourier coefficients of the efficiency, and the 'efficiency moments',
i.e.

\begin{eqnarray}
    \xi_{\parallel\parallel} &=& \frac{8 \sqrt{\pi}}{9} \left( c^0_{00}-\frac{1}{5}c^2_{00}+\sqrt{\frac{1}{20}}\left(c^0_{20}-\frac{1}{5}c^2_{20}\right)+\sqrt{\frac{3}{20}}\left(c^0_{22}-\frac{1}{5}c^2_{22}\right)  \right)  \\
    \xi_{00}                 &=& \frac{8 \sqrt{\pi}}{9} \left( c^0_{00}+\frac{2}{5}c^2_{00}+\sqrt{\frac{1}{20}}(c^0_{20}+\frac{2}{5}c^2_{20})-\sqrt{\frac{3}{20}}(c^0_{22}+\frac{2}{5}c^2_{22}) \right) \\
    \xi_{\perp\perp}         &=& \frac{8 \sqrt{\pi}}{9} \left( c^0_{00}-\frac{1}{5}c^2_{00} - \sqrt{\frac{1}{5}}\left(c^0_{20}-\frac{1}{5}c^2_{20}\right)\right)\\
    \xi_{\parallel\perp}     &=& \frac{8 \sqrt{\pi}}{9} \sqrt{\frac{3}{5}}\left(c^0_{2,-1}-\frac{1}{5}c^2_{2,-1}\right)  \\
    \xi_{\perp 0}            &=& -\frac{8 \sqrt{\pi}}{9} \sqrt{\frac{6}{5}}\frac{3\pi}{32}\left( +c^1_{2,1} -\frac{1}{4}c^3_{2,1}- \frac{5}{128}c^5_{2,1} -\frac{7}{512}c^7_{2,1} - \frac{105}{16384}c^9_{2,1}+... \right) \\
    \xi_{0\parallel}         &=& \frac{8 \sqrt{\pi}}{9} \sqrt{\frac{6}{5}}\frac{3\pi}{32}\left( +c^1_{2,-2}-\frac{1}{4}c^3_{2,-2}-\frac{5}{128}c^5_{2,-2}-\frac{7}{512}c^7_{2,-2} - \frac{105}{16384}c^9_{2,-2}+... \right) 
\end{eqnarray}

This implies that, if the efficiency is uniform (and $100\%$) in the angles, i.e.
\begin{equation}
   \epsilon(\psi,\theta,\phi) = 1 \Rightarrow c^0_{00} = 2\sqrt{\pi} ; c^{i}_{jk} = 0 (i\neq 0, j\neq 0, k \neq 0)
\end{equation}
we can read off the moments as:
\begin{equation}
    \xi_{\parallel\parallel}= \xi_{00}= \xi_{\perp\perp}  = \frac{16\pi}{9};\;\; \xi_{\parallel\perp} =\xi_{\perp 0} = \xi_{0\parallel} = 0
\end{equation}
Given that it is known that  in the likelihood fit only the (relative) normalization of the six angular
terms matters, it is clear that only a subset the Fourier components needs to be know to
perform the fit. Of course additional terms will improve the visual results when plotting 
differential distributions.

In case only the $\xi_{ab}$ are known, and an equivalent set of $c^i_{jk}$ is required, the 
following may be used:
\begin{eqnarray}
 c^0_{00}  &=&  \frac{1}{3}( \xi_{\parallel \parallel }+\xi_{00}+\xi_{\perp\perp} )     \\
 c^2_{00}  &=& \frac{5}{3} ( \xi_{00} - \xi_{\parallel \parallel } )    \\
 c^0_{20}  &=& \frac{2}{3}\sqrt{5} ( \xi_{\parallel \parallel } - \xi_{\perp\perp} )    \\
 c^0_{2-1} &=& \sqrt{\frac{5}{3}}  \xi_{\parallel \perp}    \\
 c^1_{21}  &=& - \frac{32}{3\pi}\sqrt{\frac{5}{6}} \xi_{\perp0}    \\
 c^1_{2-2} &=&   \frac{32}{3\pi}\sqrt{\frac{5}{6}}  \xi_{0\parallel }
\end{eqnarray}
Note that this choice is not unique, but will (should)  reproduce the same minimum as a fit which used
the $\xi_{ab}$.


%TODO: take into account ${\cal D} = 1-2( <w_i> + q_T \frac{\Delta w_i}{2})$, i.e. sum over $q_T$ may not fully cancel some terms...
%and the same for the tagging times reconstruction efficiency....


We utilize the fact that, for a MC sample generated according to a P.D.F. $g(\vec{x})$,
and an efficience $\epsilon(\vec{x})$, that we can write a sum over accepted events as:
\begin{equation}
  \frac{1}{N_{\mathrm{gen}}} \sum_{\mathrm{accepted events}} f(\vec{x}_i) = \frac{1}{N_{\mathrm{gen}}} \sum_{\mathrm{generated events}} \epsilon(\vec{x}_i) f(\vec{x}_i) =  \int g(\vec{x})dx \epsilon(\vec{x}) f(\vec{x})
\end{equation}

If we now substitute the expansion of the efficiency in terms of orthonormal basis function,
i.e. equation \ref{eq:eps_exp}, then we have:

\begin{equation}
  \frac{1}{N_{\mathrm{gen}}} \sum_{\mathrm{accepted events}} f(\vec{x}_i) = c^i_{jk}  \int g(\vec{x})dx P_i(\cos\psi)Y_{jk}(\cos\theta,\phi)  f(\vec{x})
\end{equation}

Using the orthonormality of the basis functions, we can now determine the coefficients $c^i_{jk}$ 
by  choosing $f(\vec{x})$ as follows:
\begin{eqnarray}
\frac{1}{N_{\mathrm{gen}}} \sum_{\mathrm{accepted events}} \frac{2i+1}{2}\frac{ P_i(\cos\psi)Y_{jk}(\cos\theta,\phi) }{ g(\vec{\Omega}) }  &=&\\
    \int g(\vec{\Omega})d\vec{\Omega}\; c^l_{mn} P_l(\cos\psi)Y_{mn}(\cos\theta,\phi) \left[  \frac{2i+1}{2}\frac{ P_i(\cos\psi)Y_{jk}(\cos\theta,\phi) }{ g(\vec{\Omega}) } \right] &=&\\
    &=& c^i_{jk}
\end{eqnarray}

where

\begin{equation}
\vec{\Omega} \equiv (\cos\psi,\cos\theta,\phi);\;\; \int d\vec{\Omega} \equiv \int d\cos\psi d\cos\theta d\phi
\end{equation}

\end{document}


